\section{Introduction}

Machine learning and control theory are two foundational but disjoint communities. Machine learning requires data to produce models, and control systems require models to provide stability, safety or other performance guarantees. Machine learning is widely used for regression or classification, but thus far data-driven models have not been suitable for closed-loop control of physical plants. The challenge now, with using data-driven approaches, is to close the loop for real-time control and decision making.

Consider a multivariable dynamical system subject to external disturbances. The first and foremost requirement for making any decision is to obtain the underlying control-oriented predictive model of the system. With a reasonable forecast of the external disturbances, these models should predict the state of the system in the future and thus a predictive controller based on Model Predictive Control (MPC) can act preemptively to provide a desired behavior. In particular, MPC has been proven to be very powerful for multivariable systems in the presence of input and output constraints, and forecast of the disturbances. The caveat is that MPC requires a reasonably accurate physical representation of the system. This makes MPC unsuitable for control of complex plants such as natural gas processing, oil refineries, boilers, manufacturing plants, and buildings where the user expertise, time, and associated sensor costs required to develop a model are very high \cite{Sturzenegger2016,vzavcekova2014}.

There are two main reasons for model complexity. 
(1) The prime contributor is the change in model properties over time. Even if the model is identified once via an expensive route, as the model changes with time, the system identification must be repeated to update the model. Thus, model adaptability or adaptive control is desirable for such systems. 
(2) A secondary reason is the model heterogeneity which further prohibits the use of model-based control. For example, unlike the automobile or the aircraft industry, each building is designed and used in a different way. Therefore, this modeling process must be repeated for every new building. 
Due to aforementioned reasons, the control strategies in such systems are often limited to fuzzy logic rules that are based on best practices. 

The question now is, can we employ data-driven techniques to reduce the cost of modeling, and still exploit the benefits that MPC has to offer? We therefore look for automatic and data-driven approaches to control that are also adaptive, scalable and interpretable. We solve this problem with \textit{Data Predictive Control (DPC)} by bridging the gap between Machine Learning and Predictive Control.

\subsection{Practical challenges in bridging machine learning and controls}

The central idea behind DPC is to obtain control-oriented models using machine learning or black-box modeling, and formulate the control problem in a way that receding horizon control (RHC) can still be applied and the optimization problem can be solved efficiently.

At this point, it's important to note that the standard machine learning regression used for prediction is fundamentally different from using machine learning for control synthesis. In the former, all the inputs to the model (also called regressors or features) are known, while in the latter some of the inputs that are the control variables must be optimized in real-time for a desired performance. We next discuss the challenges in using machine learning algorithms for control.

\noindent \textbf{(1) Computational complexity:} Depending upon the learning algorithm, the output from a learned model is a non-linear, non-convex and sometimes non-differentiable (eg.~Random Forests \cite{Friedman2001}) function of the inputs with no closed-form expression. Using such models for control synthesis where some of the inputs must be optimized can lead to computationally intractable optimization. Our previous work on DPC uses an adaptation of Random Forests which overcomes this problem by separation of variables to derive a linearized input-output mapping at each time step \cite{JainACC2017,JainCDC2017}.
This paper uses Gaussian Processes (GP) where the output mean and variance are analytical functions of the inputs, albeit non-convex.

\noindent \textbf{(2) Data quality and quantity:} Most of the historical data that is available from complex systems like buildings are based on some rule-based controllers. Therefore, the data may not be sufficient to explain the relationship between the inputs and the outputs. To obtain richer data with enough excitation in the inputs, new experiments must be done either by sampling the inputs randomly or by a procedure for optimal experiment design (OED) \cite{Emery1998,Fedorov2010}. This paper proposes the use of GP i.e.~the estimate of variance in GP predictions to recommend control strategies for OED.

\noindent \textbf{(3) Performance guarantees and robustness:} An essential characteristic for closed-loop control is to provide performance guarantees. This becomes hard with a black-box is used to replace a physical model. However, it is possible to provide probabilistic guarantees with a learning algorithm based on Gaussian Processes. DPC based on Gaussian Processes bounds the performance errors within 95\% confidence intervals. Handling disturbance uncertainties or robustness to sensor failures in the DPC framework is part of our on-going work and is thus excluded from this paper.

\noindent \textbf{(4) Interpretability of control decisions:} Besides the accuracy of synthesizing control strategies with machine learning in the loop, we are also interested in solutions that are interpretable and trustworthy. Thus, the DPC recommendations should have traceability so they can be verified to be stable and safe. This direction of research also forms part of our on-going work.

\subsection{Overcoming practical challenges}
To address these challenges, we can take two different approaches based on what level machine learning is used to learn the models.

\noindent \textbf{(1) Mix of black-box and physics-based models:} This approach uses machine learning only to learn the dynamics of a sub-system or to model uncertainties in the dynamics. An example of former is the use of machine learning for perception and model-based control for low-level control in self-driving cars. Another example is to learn uncertainties in the models \cite{Berkenkamp2015,Desaraju2016}.

\noindent \textbf{(2) Fully black-box models:} Here, the full dynamical model is obtained using only machine learning algorithms. This deviates from the traditional idea of system identification where a physics-based structure is assumed to begin with. An example would be fully autonomous control using camera \cite{Bojarski2016}. 
%The catch here is that, prior to learning, sufficiently large data could be generated by running the car in simulations.

For the application to building control in context of Demand Response, this paper explores the latter route to bypass the modeling cost and difficulties as summarized in \cite{Sturzenegger2016}.

\subsection{Contributions}

\todo[inline]{revisit when OED and CONTROL sections are ready}

This paper answers the following questions:
\begin{enumerate}
	\item Optimal experiment design for batch updates
	\item Optimal experiment design for sequential updates in real-time
	\item Data Predictive Control using Gaussian Processes for real-time control
	\item Online learning with a Data Predictive Controller in closed-loop
\end{enumerate}

\begin{figure*}[t!]
	\centering
	\missingfigure[figwidth=42pc]{Paper organization and contributions}
	\caption{Paper organization and contributions}
	\captionsetup{justification=centering}
	\label{F:intro}
\end{figure*}

