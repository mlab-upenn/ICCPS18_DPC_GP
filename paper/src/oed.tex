\section{Optimal Experiment Design}
\label{S:oed}

In this section, we address the practical challenge of ``Data quality" listed in Sec.~\ref{SS:practical_challenges}.

In general, the more data we have, a better model we can learn using machine learning algorithms. When sufficient training data is not available for learning the behavior of the dynamical system, we resort to \textit{optimal experiment design} (OED) or \textit{functional testing}, a method of exciting the inputs of the dynamical system and measuring its response. For example, in the control literature, a popular technique, especially for linear systems, is measuring the \textit{step response} of the system to estimate the time-constants, and further for designing of controllers. In context of buildings as we discuss in Sec.~\ref{S:casestudy}, it is often the case that very limited data from the installed sensors and multimeters are available. Hence, we need to design a mechanism to recommend control strategies to sample new data.

For optimal experiment design with GP as the learned model, we follow the \textit{information theoretic} approach to estimate how well the training samples explain the behavior underlying physical system.

\subsection{Information theoretic approach to OED}
\label{SS:information-theory}

In this section, we show how the variance in the predictions in GP regression \eqref{E:gp-regression} can be exploited for experiment design.
The goal here is to update the parameters \(\theta\) in the model \(y \sim \mathcal{GP}(\mu(x), k(x); \theta)\) as new samples are observed sequentially. One popular metric of selecting the next sample is the point of Maximum Variance (MV), which is also widely used for Bayesian Optimization using GPs \cite{Snoek2012}. Since, we can calculate the variance in \(y\) for any \(x\), OED based on MV can be directly computed using \eqref{E:gp-regression}. However, another metric which has proven to more powerful in learning parameters \(\theta\) is the Information Gain (IG) \cite{Krause2008}. 

IG metric selects the sample which adds maximum information to the model, i.e.~reduces the maximum uncertainty in \(\theta\). If we denote the existing data before sampling by \(\D\), then the goal is to select \(x\) that maximizes the information gain defined as
\begin{align}
\argmax_x H(\theta|\D) - \EE_{y \sim \GaussianDist{\bar{y}(x)}{\sigma^2(x)}}H(\theta|\D,x,y),
\label{E:ig:theta}
\end{align}
where, \(H\) is the Shanon's Entropy given by
\begin{align}
H(\theta|\D) = -\int p(\theta|\D) \log (p(\theta|\D))d\theta.
\end{align}
Since \(y|x \sim \GaussianDist{\bar{y}(x)}{\sigma^2(x)}\), we need to take an expectation over \(y\). When the dimension of \(\theta\) is large, computing entropies is typically computationally intractable. Using equivalence of the expressions \(H(\theta) - H(\theta|y) = H(y) - H(y|\theta)\),
we can transform \eqref{E:ig:theta} as
\begin{align}
\argmax_x H(y|x,\D) - \EE_{\theta \sim p(\theta|\D)}H(y|x,\theta).
\label{E:ig:y}
\end{align}
In this case, the expectation is defined over \(\theta\) the expression \eqref{E:ig:y} is much easier to compute because \(y\) is now single dimension. For further details, we refer the reader to \cite{Houlsby2011}.
The first term in \eqref{E:ig:y} can be calculated by marginalizing over the distribution of \(\theta|\D\):
\begin{align}
p(y|x,\D) =& \EE_{\theta \sim p(\theta|\D)}p(y|x,\theta,\D) \nonumber\\
=& \int p(y|x,\theta, \D)p(\theta|\D)d\theta
\end{align}
for which the exact solution is difficult to compute. We therefore use an approximation described in \cite{Garnett2013}. It is shown that for \(\theta|\D \sim \GaussianDist{\bar{\theta}}{\Sigma}\), we can find a linear approximation to \(\bar{y}(x) = a^T(x)\theta+b(x)\) such that
\begin{align}
\EE_{\theta \sim p(\theta|\D)}p(y|x,\theta,\D) \sim \GaussianDist{a^T\bar{\theta}+b}{\sigma^2+a^T\Sigma a}.
\end{align}
Under the same approximation, the second term in \eqref{E:ig:y} can be written as \(H(y|x,\hat{\theta})\). 
Finally, using the relation for the differential entropy for a Gaussian distribution, the information gain in \eqref{E:ig:y} is approximated as
%\begin{align}
%H(y|x,\D) = \frac{1}{2}\log(2\pi e \sigma^2(x)).
%\end{align}
\begin{align}
\text{IG} = \frac{1}{2}\log\left(\frac{\sigma^2(x)+a^T(x)\Sigma a(x)}{\sigma^2(x)}\right).
\label{E:ig:final}
\end{align}
Next, we apply this result for sequential optimal experiment design.

\subsection{Sequential sampling: recommending control strategies for experiment design }

%As said before, when the available data is limited, we need a procedure to sample new data. 
The goal here is to update the model parameters \(\theta\) of the GP efficiently as new data is observed. 
To begin the experiment design, we assume that we only know about which features \(x\) have an influence on the output \(y\). This is often known in practice. For example, for the case study in Sec.~\ref{S:casestudy}, the output of interest if the building power consumption, and the features we consider include outside air temperature and humidity, time of day to account for occupancy, control set points and lagged terms for the output. Then a covariance structure of GP must be selected. For the example above, we chose a squared exponential kernel.
If samples \(\D := (X,Y)\) are available, we can assign the prior distribution on \(\theta\) based on the MLE estimate \( \argmax_\theta \Pr(Y \vert X, \theta)\), i.e.~\(\theta_{\mathrm{0}} \sim \GaussianDist{\theta_{\mathrm{MLE}}}{\sigma^2_{\mathrm{init}}}\) where a suitable value of \(\sigma^2_{\mathrm{init}}\) is chosen, otherwise, the mean of the Gaussian priors \(\theta_{\mathrm{0}} \sim \GaussianDist{\mu_{\mathrm{{init}}}}{\sigma^2_\mathrm{init}}\) is also initialized manually.

\begin{algorithm}[!tb]
	\caption{Sequential sampling for OED}
	\label{A:oed:sequential}
	\begin{algorithmic}[1]
		\Procedure{Initialization}{}
		\If{initial \(\D := (X,Y)\)}
		\State Compute \( \theta_{\mathrm{MLE}} = \argmax_{\theta^{\mathrm{MLE}}} \Pr(Y \vert X, \theta)\)
		\State Assign priors \(\theta_{\mathrm{0}} \sim \GaussianDist{\theta_{\mathrm{MLE}}}{\sigma^2_{\mathrm{init}}}\)
		\Else 
		\State Assign priors \(\theta_{\mathrm{0}} \sim \GaussianDist{\mu_{\mathrm{{init}}}}{\sigma^2_\mathrm{init}}\)
		\EndIf
		\EndProcedure
		\Procedure{Sampling}{}
		\While{\(t<t_{\mathrm{max}}\)}
		\State Calculate features \(x_t\) in \eqref{E:GP:features} as a function of \(u_t\)
		\State Solve \eqref{E:oed:sampling} to calculate optimal \(u^*_t\)
		\State Apply \(u^*_t\) to the system and measure \(y_t\)
		\State \(\D = \D \cup (x_t,y_t) \)
		\State Update \( \theta_{\mathrm{t}} = \argmax_{\theta^{\mathrm{MAP}}} \Pr(Y \vert X, \theta_{\mathrm{t-1}})\)
		\EndWhile
		\EndProcedure
	\end{algorithmic}
\end{algorithm}

Now, consider a dynamical GP model introduced in Sec.~\ref{S:gp},
\begin{math}
y_{t} = f(x_t;\theta)
\end{math}
where
\begin{align}
x_{t}\!=\![y_{t-l}, \dots, y_{t-1}, u_{t-m}, \dots, u_t, w_{t-p}, \dots, w_{t-1}, w_t] \text.
\label{E:GP:features}
\end{align}
At time \(t\), the current disturbance, and the lagged terms of the output, control input and disturbance are all known. The current control input \(u_t \in \RR^p \) is the only unknown feature for experiment design. For physical systems, very often, we must operate under strict actuation or operation constraints. Therefore, the new sampled inputs must lie within these constraints. To this end, we solve the following optimization problem to compute optimal control set point recommendations \(u^*_t\) for experiment design
\begin{align}
\label{E:oed:sampling}
\maximize_{u_{t}} & \ \ \ \frac{1}{2}\log\left(\frac{\sigma^2(x_t)+a^T(x_t)\Sigma a(x_t)}{\sigma^2(x_t)}\right) \\
\st &  \ \ \ \   u_t \in \mathcal{U} \nonumber
\end{align}
The new control input \(u^*_t\) is applied to the physical system to generate the output \(y_t\), update the parameters \(\theta\) using MAP estimate \cite{Garnett2013}, and we proceed to time \(t+1\). 
The algorithm for OED is summarized in Algo.~\ref{A:oed:sequential}.
In Sec.~\ref{S:casestudy}, for a dynamical model of a building, we add operation constraints on a Chiller system to optimally sample the chilled water temperature, supply air temperature and the zone level cooling set point. For a building model, an example of experiment design is shown in Fig.~\ref{F:oed:example}. Initially the OED is much better. As the experiment is run in closed loop with the building for more and more days, the performance of OED approaches to that of random sampling. 

\begin{figure}[!tb]
	\centering
	\setlength\fwidth{0.44\textwidth}
	\setlength\hwidth{0.2\textwidth}	
	% This file was created by matlab2tikz.
%
%The latest updates can be retrieved from
%  http://www.mathworks.com/matlabcentral/fileexchange/22022-matlab2tikz-matlab2tikz
%where you can also make suggestions and rate matlab2tikz.
%
\begin{tikzpicture}

\begin{axis}[%
width=0.951\fwidth,
height=\hwidth,
at={(0\fwidth,0\hwidth)},
scale only axis,
unbounded coords=jump,
xmin=0.75,
xmax=4.25,
xtick={1,2,3,4},
xticklabels={{3},{7},{10},{14}},
xlabel style={font=\color{white!15!black}},
xlabel={no of days},
ymin=10,
ymax=65,
ylabel style={font=\color{white!15!black}},
ylabel={error [kW]},
axis background/.style={fill=white},
legend style={legend cell align=left, align=left, fill=none, draw=none},
xlabel style={font=\footnotesize},ylabel style={font=\footnotesize},legend style={font=\footnotesize},ticklabel style={font=\footnotesize},ylabel shift = -5 pt,xlabel shift = -5 pt,
]
\addplot [color=black, line width=0.8pt, mark=o, mark options={solid, black}]
  table[row sep=crcr]{%
1	38.3577768196538\\
2	23.5762131941091\\
3	22.2379994094524\\
4	22.0269257238242\\
5	nan\\
};
\addlegendentry{OED IG: RMSE}

\addplot [color=black, dashed, line width=0.8pt, mark=o, mark options={solid, black}]
  table[row sep=crcr]{%
1	60.5953168967942\\
2	33.7867759715775\\
3	25.1122218421038\\
4	23.931187395704\\
5	nan\\
};
\addlegendentry{Random: RMSE}

\addplot [color=black, line width=0.8pt, mark=diamond, mark options={solid, black}]
  table[row sep=crcr]{%
1	29.3576068831179\\
2	17.6386666958257\\
3	16.0967115004186\\
4	15.9556477539858\\
5	nan\\
};
\addlegendentry{OED IG: AE}

\addplot [color=black, dashed, line width=0.8pt, mark=diamond, mark options={solid, black}]
  table[row sep=crcr]{%
1	46.4640980105609\\
2	25.1777970816288\\
3	18.7235731225753\\
4	17.3664780029757\\
5	nan\\
};
\addlegendentry{Random: AE}

\end{axis}
\end{tikzpicture}%
	\caption{Error in prediction of power consumption on test data set. RMSE: root mean square error. AE: Absolute error. The errors are much lower with optimal experiment design initially. After one week of data is generated, the accuracy of OED is similar to random sampling.}
	\captionsetup{justification=centering}
	\label{F:oed:example}
\end{figure}
