\section{Optimal Experiment Design}
\label{S:oed}

In this section, we address the practical challenge of ``Data quality'' listed in Sec.~\ref{SS:practical_challenges}.

In general, the more data we have, the better we can learn a model using machine learning algorithms.
These data are often obtained by running tests, called \emph{functional tests}, on the real system.
However, in many applications, the amount of training data we can practically obtain is usually limited due to many factors, such as a short permitted duration for functional tests and operational or safety constraints of the physical system.
For example, in the context of buildings as we will discuss in Sec.~\ref{S:casestudy}, a functional test typically involves changing various set-points of the building energy control system in order to excite the different components and operation modes of the building, so that the obtained data will reflect their behaviors.
It is often the case that a functional test in a building is limited by the short time window during which the set-points are allowed to change, and by the maximum allowable rates of change of these set-points.
Subject to these constraints, it is desirable to optimally design the functional tests so that the data quality is maximized, in the sense that the model obtained from the data with a specific learning technique likely has the best quality possible.
This practice is known as \emph{optimal experiment design} (OED).


\subsection{Information theoretic approach to OED}
\label{SS:information-theory}

In this section, we present an \emph{information theoretic} approach for OED to incrementally design or select the best data points for explaining the behavior of the underlying physical system with GP.
This is achieved by exploiting the predictive variance in GP regression \eqref{E:gp-regression}.
The goal here is to update the hyperparameters \(\theta\) in the model \(y \sim \mathcal{GP}(\mu(x), k(x); \theta)\) as new samples are observed sequentially.
One popular method for selecting the next sample is the point of Maximum Variance (MV), which is also widely used for Bayesian Optimization using GPs \cite{Snoek2012}.
Since we can calculate the variance in \(y\) for any \(x\), OED based on MV can be directly computed using \eqref{E:gp-regression}.
However, another approach which has been shown to result in better samples for learning the hyperparameters \(\theta\) is maximizing the Information Gain (IG) \cite{Krause2008}. In Sec.~\ref{S:casestudy}, we will compare both approaches in a case study.

The IG approach selects the sample which adds the maximum information to the model, i.e.~which reduces the maximum uncertainty in \(\theta\). If we denote the existing data before sampling by \(\D\), then the goal is to select \(x\) that maximizes the information gain defined as
\begin{align}
\argmax_x H(\theta|\D) - \EE_{y \sim \GaussianDist{\bar{y}(x)}{\sigma^2(x)}}H(\theta|\D,x,y),
\label{E:ig:theta}
\end{align}
where \(H\) is the Shannon's Entropy given by
\begin{align}
H(\theta|\D) = -\int p(\theta|\D) \log (p(\theta|\D))d\theta.
\end{align}
Since \(y|x \sim \GaussianDist{\bar{y}(x)}{\sigma^2(x);\theta}\), we need to take an expectation over \(y\).
When the dimension of \(\theta\) is large, computing entropies is typically computationally intractable.
Using the equality \(H(\theta) - H(\theta|y) = H(y) - H(y|\theta)\), we can rewrite \eqref{E:ig:theta} equivalently as
\begin{align}
\argmax_x H(y|x,\D) - \EE_{\theta \sim p(\theta|\D)}H(y|x,\theta).
\label{E:ig:y}
\end{align}
In this case, as the expectation is defined over \(\theta\), \eqref{E:ig:y} is much easier to compute because \(y\) is a scaler.
For further details, we refer the reader to \cite{Houlsby2011}.
The first term in \eqref{E:ig:y} can be calculated by marginalizing over the distribution of \(\theta|\D\):
\begin{align}
p(y|x,\D) &= \EE_{\theta \sim p(\theta|\D)}p(y|x,\theta,\D) \nonumber\\
&= \int p(y|x,\theta, \D)p(\theta|\D)d\theta
\end{align}
for which the exact solution is difficult to compute. We therefore use an approximation described in \cite{Garnett2013}. It is shown that for \(\theta|\D \sim \GaussianDist{\bar{\theta}}{\Sigma}\), we can find a linear approximation to \(\bar{y}(x) = a^T(x)\theta+b(x)\) such that
\begin{align}
p(y|x,\D) \sim \GaussianDist{a^T\bar{\theta}+b}{\sigma^2+a^T\Sigma a}
\end{align}
in the neighborhood of \(\bar{\theta}\).
Under the same approximation, the variance \(p(y|x,\D)\) is approximated to be
\begin{align}
\tilde{\sigma}^2(x) = \frac{4}{3}\sigma^2(x) + \frac{\partial \bar{y}(x)}{\partial \theta}^T \Sigma \frac{\partial \bar{y}(x)}{\partial \theta} + \nonumber \\
\qquad\qquad \frac{1}{3\sigma^2(x)}\frac{\partial \sigma^2(x)}{\partial \theta}^T \Sigma \frac{\partial \sigma^2(x)}{\partial \theta}
\end{align}
evaluated at \(\bar{\theta}\) while the second term in \eqref{E:ig:y} can be written as \(H(y|x,\bar{\theta})\). 
Finally, maximizing the information gain in \eqref{E:ig:theta} is equivalent to maximizing \(\tilde{\sigma}^2(x)/{\sigma}^2(x)\).
Next, we apply this result for sequential optimal experiment design.

% \subsection{Sequential sampling: recommending control strategies for experiment design }
\subsection{Sequential experiment design with Gaussian Processes}
\label{SS:oed:sequential}

%As said before, when the available data is limited, we need a procedure to sample new data. 
Our goal is to update the hyperparameters \(\theta\) of the GP efficiently as new data is observed. 
To begin the experiment design, we assume that we only know about which features \(x\) have an influence on the output \(y\). This is often known in practice. For example, for the case study in Sec.~\ref{S:casestudy}, the output of interest is the building power consumption, and the features we consider include outside air temperature and humidity, time of day to account for occupancy, control setpoints and lagged terms for the output. Then a covariance structure of GP must be selected. For the example above, we chose a squared exponential kernel.
If samples \(\D := (X,Y)\) are available, we can assign the prior distribution on \(\theta\) based on the MLE estimate \( \argmax_\theta \Pr(Y \vert X, \theta)\), i.e.~\(\theta_{\mathrm{0}} \sim \GaussianDist{\theta_{\mathrm{MLE}}}{\sigma^2_{\mathrm{init}}}\) where a suitable value of \(\sigma^2_{\mathrm{init}}\) is chosen.
Otherwise, the Gaussian priors \(\theta_{\mathrm{0}} \sim \GaussianDist{\mu_{\mathrm{{init}}}}{\sigma^2_\mathrm{init}}\) are initialized manually.

\begin{algorithm}[!tb]
	\caption{Sequential sampling for OED based on IG}
	\label{A:oed:sequential}
	\begin{algorithmic}[1]
		\Procedure{Initialization}{}
		\If{initial \(\D := (X,Y)\)}
		\State Compute \( \theta_{\mathrm{MLE}} = \argmax_{\theta^{\mathrm{MLE}}} \Pr(Y \vert X, \theta)\)
		\State Assign priors \(\theta_{\mathrm{0}} \sim \GaussianDist{\theta_{\mathrm{MLE}}}{\sigma^2_{\mathrm{init}}}\)
		\Else 
		\State Assign priors \(\theta_{\mathrm{0}} \sim \GaussianDist{\mu_{\mathrm{{init}}}}{\sigma^2_\mathrm{init}}\)
		\EndIf
		\EndProcedure
		\Procedure{Sampling}{}
		\While{\(t<t_{\mathrm{max}}\)}
		\State Calculate features \(x_t\) in \eqref{E:GP:features} as a function of \(u_t\)
		\State Solve \eqref{E:oed:sampling} to calculate optimal \(u^*_t\)
		\State Apply \(u^*_t\) to the system and measure \(y_t\)
		\State \(\D = \D \cup (x_t,y_t) \)
		\State Update \( \theta_{\mathrm{t}} = \argmax_{\theta^{\mathrm{MAP}}} \Pr(Y \vert X, \theta_{\mathrm{t-1}})\)
		\EndWhile
		\EndProcedure
	\end{algorithmic}
\end{algorithm}

Now, consider a dynamical GP model introduced in Sec.~\ref{S:gp},
\begin{math}
y_{t} = f(x_t;\theta)
\end{math}
where
\begin{align}
x_{t}\!=\![y_{t-l}, \dots, y_{t-1}, u_{t-m}, \dots, u_t, w_{t-p}, \dots, w_{t-1}, w_t] \text.
\label{E:GP:features}
\end{align}
At time \(t\), the current disturbance, and the lagged terms of the output, the control inputs and the disturbance are all known. The current control input \(u_t \in \RR^u \) is the only unknown feature for experiment design, which we aim to select optimally.
For physical systems, very often, we must operate under strict actuation or operation constraints. Therefore, the new sampled inputs must lie within these constraints. To this end, we solve the following optimization problem to compute optimal control set point recommendations \(u^*_t\) for experiment design
\begin{align}
\label{E:oed:sampling}
\maximize_{u_{t}} & \ \ \ \tilde{\sigma}^2(x_t)/{\sigma}^2(x_t) \\
\st &  \ \ \ \   u_t \in \mathcal{U} \nonumber
\end{align}
where \(x_t\) is a function of \(u_t\). The new control input \(u^*_t\) is applied to the physical system to generate the output \(y_t\), update the parameters \(\theta\) using maximum a posteriori (MAP) estimate \cite{Garnett2013}, and we proceed to time \(t+1\). 
The algorithm for OED is summarized in Algo.~\ref{A:oed:sequential}.

As an example, in Sec.~\ref{S:casestudy} where we learn a dynamical model of a building, the proposed OED method is used to optimally sample the chilled water temperature, supply air temperature and zone-level cooling set-points, subject to operation constraints on the chiller system.
The result is illustrated in Fig.~\ref{F:oed:example}, which shows the changes in model accuracy between several experiment design methods, for various durations of functional tests.
For short functional test durations, our OED methods achieve much more accurate models compared to random sampling methods. The random sampling requires more than 300 hours to reach the same accuracy. Since the historical data is unperturbed, it would require even longer to provide similar accuracy.

\begin{figure}[!tb]
  \centering
  \setlength\fwidth{0.4\textwidth}
  \setlength\hwidth{0.18\textwidth}	
  % This file was created by matlab2tikz.
%
%The latest updates can be retrieved from
%  http://www.mathworks.com/matlabcentral/fileexchange/22022-matlab2tikz-matlab2tikz
%where you can also make suggestions and rate matlab2tikz.
%
\begin{tikzpicture}

\begin{axis}[%
width=0.951\fwidth,
height=\hwidth,
at={(0\fwidth,0\hwidth)},
scale only axis,
unbounded coords=jump,
xmin=0.75,
xmax=4.25,
xtick={1,2,3,4},
xticklabels={{3},{7},{10},{14}},
xlabel style={font=\color{white!15!black}},
xlabel={no of days},
ymin=10,
ymax=65,
ylabel style={font=\color{white!15!black}},
ylabel={error [kW]},
axis background/.style={fill=white},
legend style={legend cell align=left, align=left, fill=none, draw=none},
xlabel style={font=\footnotesize},ylabel style={font=\footnotesize},legend style={font=\footnotesize},ticklabel style={font=\footnotesize},ylabel shift = -5 pt,xlabel shift = -5 pt,
]
\addplot [color=black, line width=0.8pt, mark=o, mark options={solid, black}]
  table[row sep=crcr]{%
1	38.3577768196538\\
2	23.5762131941091\\
3	22.2379994094524\\
4	22.0269257238242\\
5	nan\\
};
\addlegendentry{OED IG: RMSE}

\addplot [color=black, dashed, line width=0.8pt, mark=o, mark options={solid, black}]
  table[row sep=crcr]{%
1	60.5953168967942\\
2	33.7867759715775\\
3	25.1122218421038\\
4	23.931187395704\\
5	nan\\
};
\addlegendentry{Random: RMSE}

\addplot [color=black, line width=0.8pt, mark=diamond, mark options={solid, black}]
  table[row sep=crcr]{%
1	29.3576068831179\\
2	17.6386666958257\\
3	16.0967115004186\\
4	15.9556477539858\\
5	nan\\
};
\addlegendentry{OED IG: AE}

\addplot [color=black, dashed, line width=0.8pt, mark=diamond, mark options={solid, black}]
  table[row sep=crcr]{%
1	46.4640980105609\\
2	25.1777970816288\\
3	18.7235731225753\\
4	17.3664780029757\\
5	nan\\
};
\addlegendentry{Random: AE}

\end{axis}
\end{tikzpicture}%
  \caption{Error in prediction of power consumption on test data set. RMSE: root mean square error. AE: Absolute error. The errors are much lower with optimal experiment design based on information gain. The random sampling requires more than 300 hours to reach the same accuracy}
  \captionsetup{justification=centering}
  \label{F:oed:example}
\end{figure}


%%% Local Variables:
%%% mode: latex
%%% TeX-master: "main"
%%% End:
