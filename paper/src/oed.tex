\section{Optimal Experiment Design}
\label{S:oed}

In this section, we address the practical challenge of ``Data quality and quantity" and also touch upon ``Model adaptability" listed in Sec.~\ref{SS:practical_challenges}.
\\

For practical applications, we come across two kinds of situations:
\begin{enumerate}
	\item \textbf{Insufficient data:} In general, the more data we have, a better model we can learn using machine learning algorithms. When sufficient training data is not available for learning the behavior of the dynamical system, we resort to \textit{optimal experiment design} (OED) or \textit{functional testing}, a method of exciting the inputs of the dynamical system and measuring its response. For example, in the control literature, a popular technique, especially for linear systems, is measuring the \textit{step response} of the system to estimate the time-constants, and further for designing of controllers. In context of buildings as we discuss in Sec.~\ref{S:casestudy}, it is often the case that very limited data from the installed sensors and multimeters are available. Hence, we need to design a mechanism to recommend control strategies to sample new data.
	
	\item \textbf{Computational complexity:} Even if we have sufficient data, it may not be directly suitable for learning because of noisy measurements/outliers or using the entire data set may be not advisable due to computational complexity as is the case with GPs. The solution to this problem lies in \textit{selecting the most informative batch} (from the available data) that best explains the system behavior or dynamics. Another application of this in periodic update of the learned model as the system properties change over time. For example, the same GP model may be not be suitable to control a building in both Summer and Winter season, so we must select the most informative data from year around data. We discuss active learning in detail in Sec.~\ref{S:active}.
\end{enumerate}

For \textit{optimal experiment design} and \textit{selecting the most informative batch} with GP as the learned model, we follow the \textit{information theoretic} approach to estimate how well the training samples explain the behavior underlying physical system.

\subsection{Information theoretic approach to OED}

In this section, we show how the prediction variance \eqref{E:gp-regression} in GPs can be exploited for experiment design.
The goal here is to update the parameters \(\theta\) in the model \(y \sim \mathcal{GP}(\mu(x), k(x); \theta)\) as new samples are observed sequentially. One popular metric of selecting the next sample is the point of Maximum Variance (MV), which is also widely used for Bayesian Optimization using GPs \cite{Snoek2012}. Since, we can calculate the variance in \(y\) for any \(x\), OED based on MV is straight forward to compute. However, another metric which has proven to more powerful in learning parameters \(\theta\) is the Information Gain (IG) \cite{Krause2008}. 

IG metric selects the sample which adds maximum information to the model, i.e.~reduces the uncertainty in \(\theta\) the most. If we denote the existing data before sampling by \(\D\), then the goal is to select \(x\) that maximizes the information gain defined as
\begin{align}
\argmax_x H(\theta|\D) - \EE_{y \sim \GaussianDist{\bar{y}(x)}{\sigma^2(x)}}H(\theta|\D,x,y),
\label{E:ig:theta}
\end{align}
where, \(H\) is the Shanon's Entropy given by
\begin{align}
H(\theta|\D) = -\int p(\theta|\D) \log (p(\theta|\D))d\theta.
\end{align}
Since \(y|x \sim \GaussianDist{\bar{y}(x)}{\sigma^2(x)}\), we need to take an expectation over \(y\). When the dimension of \(\theta\) is large, computing entropies is typically computationally intractable. Using equivalence of the expressions \(H(\theta) - H(\theta|y) = H(y) - H(y|\theta)\),
we can transform \eqref{E:ig:theta} as
\begin{align}
\argmax_x H(y|x,\D) - \EE_{\theta \sim p(\theta|\D)}H(y|x,\theta).
\label{E:ig:y}
\end{align}
In this case, the expectation is defined over \(\theta\) the expression \eqref{E:ig:y} is much easier to compute because \(y\) is now single dimension. For further details, we refer the reader to \cite{Houlsby2011}.
The first term in \eqref{E:ig:y} can be calculated by marginalizing over the distribution of \(\theta|\D\):
\begin{align}
p(y|x,\D) =& \EE_{\theta \sim p(\theta|\D)}p(y|x,\theta,\D) \nonumber\\
=& \int p(y|x,\theta, \D)p(\theta|\D)d\theta
\end{align}
for which the exact solution is difficult to compute. We therefore use an approximation described in \cite{Garnett2013}. It is shown that for \(\theta|\D \sim \GaussianDist{\bar{\theta}}{\Sigma}\), we can find a linear approximation to \(\bar{y}(x) = a^T(x)\theta+b(x)\) such that
\begin{align}
\EE_{\theta \sim p(\theta|\D)}p(y|x,\theta,\D) \sim \GaussianDist{a^T\bar{\theta}+b}{\sigma^2+a^T\Sigma a}.
\end{align}
Under the same approximation, the second term in \eqref{E:ig:y} can be written as \(H(y|x,\hat{\theta})\). 
Finally, using the relation for the differential entropy for a Gaussian distribution, the information gain in \eqref{E:ig:y} is approximated as
%\begin{align}
%H(y|x,\D) = \frac{1}{2}\log(2\pi e \sigma^2(x)).
%\end{align}
\begin{align}
\text{IG} = \frac{1}{2}\log\left(\frac{\sigma^2(x)+a^T(x)\Sigma a(x)}{\sigma^2(x)}\right).
\label{E:ig:final}
\end{align}
Next, we apply this result for sequential optimal experiment design and best batch selection.

\subsection{Sequential sampling: recommending control strategies for experiment design }

%As said before, when the available data is limited, we need a procedure to sample new data. 
The goal here is to update the model parameters \(\theta\) of the GP efficiently as new data is observed. 
To begin the experiment design, we assume that we only know about which features \(x\) have an influence on the output \(y\). This is often known in practice. For example, for the case study in Sec.~\ref{S:casestudy}, the output of interest if the building power consumption, and the features we consider include outside air temperature and humidity, time of day to account for occupancy, control set points and lagged terms for the output. Then a covariance structure of GP must be selected. For the example above, we chose a squared exponential kernel.
If samples \(\D := (X,Y)\) are available, we can assign the prior distribution on \(\theta\) based on the MLE estimate \( \argmax_\theta \Pr(Y \vert X, \theta)\), i.e.~\(\theta_{\mathrm{0}} \sim \GaussianDist{\theta_{\mathrm{MLE}}}{\sigma^2_{\mathrm{init}}}\) where a suitable value of \(\sigma^2_{\mathrm{init}}\) is chosen, otherwise, the mean of the Gaussian priors \(\theta_{\mathrm{0}} \sim \GaussianDist{\mu_{\mathrm{{init}}}}{\sigma^2_\mathrm{init}}\) is also initialized manually.

Now, consider a dynamical GP model introduced in Sec.~\ref{S:intro-gp:control},
\begin{math}
y_{t} = f(x_t;\theta)
\end{math}
where
\begin{align}
x_{t}\!=\![y_{t-l}, \dots, y_{t-1}, u_{t-m}, \dots, u_t, w_{t-p}, \dots, w_{t-1}, w_t] \text.
\label{E:GP:features}
\end{align}
At time \(t\), the current disturbance, and the lagged terms of the output, control input and disturbance are all known. The current control input \(u_t \in \RR^p \) is the only unknown feature for experiment design. For physical systems, very often, we must operate under strict actuation or operation constraints. Therefore, the new sampled inputs must lie within these constraints. To this end, we solve the following optimization problem to compute optimal control set point recommendations \(u^*_t\) for experiment design
\begin{align}
\label{E:oed:sampling}
\maximize_{u_{t}} & \ \ \ \frac{1}{2}\log\left(\frac{\sigma^2(x_t)+a^T(x_t)\Sigma a(x_t)}{\sigma^2(x_t)}\right) \\
\st &  \ \ \ \     u_{\mathrm{min}}  \leq u_t \leq u_{\mathrm{max}} \nonumber
\end{align}
The new control input \(u^*_t\) is applied to the physical system to generate the output \(y_t\), update the parameters \(\theta\) using MAP estimate \cite{Garnett2013}, and we proceed to time \(t+1\). 
The algorithm for OED is summarized in Algo.~\ref{A:oed:sequential}.
In Sec.~\ref{S:casestudy}, for a dynamical model of a building, we add operation constraints on a Chiller system to optimally sample the chilled water temperature, supply air temperature and the zone level cooling set point. 

\begin{algorithm}[!tb]
	\caption{Sequential sampling for OED}
	\label{A:oed:sequential}
	\begin{algorithmic}[1]
		\Procedure{Initialization}{}
		\If{initial \(\D := (X,Y)\)}
		\State Compute \( \theta_{\mathrm{MLE}} = \argmax_{\theta^{\mathrm{MLE}}} \Pr(Y \vert X, \theta)\)
		\State Assign priors \(\theta_{\mathrm{0}} \sim \GaussianDist{\theta_{\mathrm{MLE}}}{\sigma^2_{\mathrm{init}}}\)
		\Else 
		\State Assign priors \(\theta_{\mathrm{0}} \sim \GaussianDist{\mu_{\mathrm{{init}}}}{\sigma^2_\mathrm{init}}\)
		\EndIf
		\EndProcedure
		\Procedure{Sampling}{}
		\While{\(t<t_{\mathrm{max}}\)}
		\State Calculate features \(x_t\) in \eqref{E:GP:features} as a function of \(u_t\)
		\State Solve \eqref{E:oed:sampling} to calculate optimal \(u^*_t\)
		\State Apply \(u^*_t\) to the system and measure \(y_t\)
		\State \(\D = \D \cup (x_t,y_t) \)
		\State Update \( \theta_{\mathrm{t}} = \argmax_{\theta^{\mathrm{MAP}}} \Pr(Y \vert X, \theta_{\mathrm{t-1}})\)
		\EndWhile
		\EndProcedure
	\end{algorithmic}
\end{algorithm}

\begin{figure}[!tb]
	\centering
	\missingfigure[figwidth=20pc]{comparison of model accuracy b/w  IG, MV, Uniform, PRBS}
	\caption{}
	\captionsetup{justification=centering}
	\label{F:OED}
\end{figure}


\subsection{Batch selection: selecting most informative data for periodic model update}

The computational complexity of training Gaussian Processes is $\bigO(n^3)$, where $n$ is number of training samples. 
Further, the data from a real system are often noisy, contain outliers. 
Obtaining the best GP model with least data is highly desired.
It is therefore essential to filter the most informative subset of data that best explain the dynamics.
In this section, we outline a systematic procedure to select best $k$ samples from given $n$ observations.
The main differences between the problem of selecting the best or the most informative subset of data and the sequential sampling for OED are that in the former, (1) all the features are variable, and (2) the decision has to be made only from the available data rather than sampling. 

Starting with a set \(\mathcal{S}\) consisting of single sample, we loop through the full data set \(\D\) to identify which sample maximizes the information gain defined in \eqref{E:ig:final}. Then, we add this sample to \(\mathcal{S}\) until \(|\mathcal{S}|=k\) or the model performance meets some desired metric. In this setup, we solve the following optimization problem
\begin{align}
\label{E:oed:batch}
\maximize_{x_{j}|(x_j,y_j) \in \mathcal{D} \setminus \mathcal{S}} & \ \ \ \frac{1}{2}\log\left(\frac{\sigma^2(x_j)+a^T(x_j)\Sigma a(x_j)}{\sigma^2(x_j)}\right)
\end{align}

The algorithm for OED is summarized in Algo.~\ref{A:oed:batch}.

\begin{algorithm}[!tb]
	\caption{Batch selection for OED}
	\label{A:oed:batch}
	\begin{algorithmic}[1]
		\Procedure{Initialization}{}
		\State Sample with replacement \(k\) integers \( \in \{1,\dots,n\} \)
		\State Compute \( \theta_{\mathrm{MLE}} = \argmax_{\theta^{\mathrm{MLE}}} \Pr(Y \vert X, \theta)\)
		\State Assign priors \(\theta_{\mathrm{0}} \sim \GaussianDist{\theta_{\mathrm{MLE}}}{\sigma^2_{\mathrm{init}}}\)
		\EndProcedure
		\State Define \(\mathcal{S} = \varnothing\)
		\Procedure{Sampling}{}
		\While{\( j < k \)}
		\State Solve \eqref{E:oed:batch} for optimal \({x_{j} \vert (x_j,y_j) \in \mathcal{D} \setminus \mathcal{S}} \)
		\State \(\mathcal{S} = \mathcal{S} \cup (x_j,y_j) \)
		\State Update \( \theta_{\mathrm{j}} = \argmax_{\theta^{\mathrm{MAP}}} \Pr(Y \vert X, \theta_{\mathrm{j-1}})\)
		\EndWhile
		\EndProcedure
	\end{algorithmic}
\end{algorithm}

%\begin{figure}[!tb]
%	\centering
%	\setlength\fwidth{0.4\textwidth}
%	\setlength\hwidth{0.3\textwidth}	
%	% This file was created by matlab2tikz.
%
%The latest updates can be retrieved from
%  http://www.mathworks.com/matlabcentral/fileexchange/22022-matlab2tikz-matlab2tikz
%where you can also make suggestions and rate matlab2tikz.
%
\definecolor{mycolor1}{rgb}{0.97647,0.89804,1.00000}%
\definecolor{mycolor2}{rgb}{0.85000,0.32500,0.09800}%
\definecolor{mycolor3}{rgb}{0.92900,0.69400,0.12500}%
%
\begin{tikzpicture}

\begin{axis}[%
width=\fwidth,
height=0.362\hwidth,
at={(0\fwidth,0.611\hwidth)},
scale only axis,
xmin=0,
xmax=100,
ymin=0,
ymax=450,
ylabel style={font=\color{white!15!black}},
ylabel={power [kW]},
axis background/.style={fill=white},
xmajorgrids,
ymajorgrids,
legend style={at={(0.5,1.03)}, anchor=south, legend columns=3, legend cell align=left, align=left, fill=none, draw=none}
]

\addplot[area legend, draw=mycolor1, fill=mycolor1]
table[row sep=crcr] {%
x	y\\
0	254.189473869388\\
1	267.755337204084\\
2	297.889548747117\\
3	257.052229060928\\
4	247.912765314504\\
5	232.483745847208\\
6	283.468895356103\\
7	260.337891183406\\
8	304.819544461068\\
9	225.774411512617\\
10	215.353033748236\\
11	222.409673756027\\
12	236.578167554865\\
13	211.706670355527\\
14	234.781221224438\\
15	279.261828184355\\
16	291.972949524987\\
17	236.266907083932\\
18	243.123686192491\\
19	230.839632424542\\
20	326.961019580528\\
21	317.82988291468\\
22	258.228238034178\\
23	280.227312657035\\
24	348.977123174\\
25	360.137300505161\\
26	296.207849361431\\
27	319.6318261712\\
28	342.677050785344\\
29	333.38510275672\\
30	395.987285516545\\
31	363.891416237357\\
32	415.557395386526\\
33	361.169497128139\\
34	418.649882855963\\
35	416.005192752497\\
36	402.243296810863\\
37	408.080480276701\\
38	359.985748189533\\
39	348.024351886284\\
40	332.551391580914\\
41	310.449244504022\\
42	301.337564228733\\
43	292.751339998192\\
44	276.185752459887\\
45	230.156990209853\\
46	206.439371824891\\
47	216.907740333905\\
48	205.790967890616\\
49	175.951202550186\\
50	167.783711244293\\
51	173.813292839088\\
52	176.218978905865\\
53	196.268173965069\\
54	164.690743597717\\
55	150.423925452778\\
56	147.309883113831\\
57	149.329992458247\\
58	136.245055456347\\
59	148.936846084171\\
60	151.099391645956\\
61	148.264015759748\\
62	168.306003395956\\
63	143.613599338115\\
64	143.89420921351\\
65	163.645326208772\\
66	169.855667472313\\
67	145.172665577423\\
68	149.045783303546\\
69	155.574202461034\\
70	165.87539539057\\
71	150.215014499578\\
72	161.457232682825\\
73	179.700889661888\\
74	182.148047013125\\
75	217.382523795121\\
76	246.463674896232\\
77	276.302151176476\\
78	309.508848930545\\
79	330.039923935275\\
80	338.551036224447\\
81	363.848361582792\\
82	350.481992047997\\
83	337.78819766102\\
84	322.684225733555\\
85	289.600332804177\\
86	296.779489521859\\
87	282.431770973139\\
88	265.621229682816\\
89	227.962708636176\\
90	245.817980506291\\
91	230.816095890899\\
92	240.077879428845\\
93	208.185344822642\\
94	215.329211810536\\
95	218.583529643087\\
96	213.100947574563\\
97	265.569905168207\\
98	248.380985625169\\
99	221.221513029373\\
100	227.841950452206\\
101	227.323889115851\\
102	257.174127428956\\
103	220.045590271869\\
104	226.514634448886\\
105	210.475624960173\\
106	261.199424133464\\
107	198.982632890027\\
108	229.080657540444\\
109	275.714297735107\\
110	220.418826187145\\
111	214.054807957089\\
112	275.383564216621\\
113	253.606206764727\\
114	237.400100380716\\
115	230.217575526456\\
116	276.606681265482\\
117	286.871764591931\\
118	311.864230183834\\
119	298.624813182405\\
120	275.341356856921\\
121	330.727569894252\\
122	343.444811472541\\
123	349.513629283739\\
124	327.051198984832\\
125	368.454705271404\\
126	326.886884128061\\
127	372.095417183608\\
128	383.12289980533\\
129	393.338410688367\\
130	362.401764945576\\
131	389.15947328937\\
132	392.040528686597\\
133	367.755024748574\\
134	363.110298810696\\
135	343.302767483347\\
136	336.599344285134\\
137	319.066441915709\\
138	293.54001013358\\
139	267.388423666701\\
140	250.115038596218\\
141	222.68167172282\\
142	208.176858933397\\
143	199.537697074823\\
144	183.970303938336\\
145	166.222278438224\\
146	179.442025294444\\
147	170.107177247921\\
148	197.217325223931\\
149	197.3462122268\\
150	163.933991570702\\
151	149.36035680523\\
152	149.876305357128\\
153	141.709186067618\\
154	142.636328780992\\
155	142.712846224849\\
156	147.398240008309\\
157	159.966936460706\\
158	147.573543963436\\
159	146.247042225991\\
160	177.24021522959\\
161	188.570175375976\\
162	152.751934553025\\
163	181.445436488432\\
164	153.803258377037\\
165	185.309889954248\\
166	160.209558709526\\
167	169.990639511012\\
168	179.933501983538\\
169	180.77071606084\\
170	223.445506336859\\
171	212.357707770177\\
172	264.728869596623\\
173	297.075746538356\\
174	304.094220217167\\
175	349.924492538935\\
176	364.999139594235\\
177	362.446245579325\\
178	366.322418121052\\
179	364.505008541922\\
180	345.514719660209\\
181	345.923427678685\\
182	307.090386410734\\
183	289.778699121849\\
184	296.725864918052\\
185	272.087035494803\\
186	267.03644458899\\
187	234.590960920218\\
188	229.249829019191\\
189	225.472790270685\\
190	247.684046023917\\
191	232.620361534192\\
192	220.04737920909\\
193	291.684683411734\\
194	267.541825981169\\
195	239.038240398447\\
196	277.508276157222\\
197	248.414932553056\\
198	306.259123466855\\
199	274.71609925262\\
200	262.419245604962\\
201	274.96226481606\\
202	276.900785289693\\
203	267.452715482256\\
204	275.067880218319\\
205	203.644840902097\\
206	258.129968226866\\
207	215.355183447878\\
208	278.043175630421\\
209	270.19320092846\\
210	230.648486591831\\
211	282.665164411278\\
212	289.723049648832\\
213	315.094944693849\\
214	301.705070817634\\
215	287.406082949487\\
216	303.530546615969\\
217	304.160027022854\\
218	355.653374027654\\
219	382.069270068747\\
220	326.941262213156\\
221	354.394218712236\\
222	370.056206256065\\
223	406.148268510527\\
224	379.739038313518\\
225	417.53802837844\\
226	409.433855258363\\
227	414.972211080169\\
228	398.621023394411\\
229	410.774556198913\\
230	353.69659555338\\
231	405.801594823466\\
232	372.028762317501\\
233	330.288942917509\\
234	298.120340157069\\
235	290.565799316558\\
236	314.376828234004\\
237	275.503943636942\\
238	246.130174049697\\
239	231.329431144799\\
240	220.138384865189\\
241	204.410998498415\\
242	238.088466493948\\
243	196.043035531191\\
244	203.911078976327\\
245	215.278152527422\\
246	178.343489203083\\
247	169.553540489023\\
248	152.053315041282\\
249	143.357664413341\\
250	144.657083861502\\
251	167.205162474864\\
252	154.930686678771\\
253	156.292812783376\\
254	181.685134582931\\
255	176.011638661418\\
256	151.549504717757\\
257	164.962537801846\\
258	170.266682846284\\
259	179.624594122004\\
260	180.426345630737\\
261	178.051792091184\\
262	172.899641562454\\
263	183.94515667613\\
264	177.173897874327\\
265	185.813604414809\\
266	239.983060851441\\
267	214.978879654088\\
268	271.245280165826\\
269	274.89967503092\\
270	331.366282477478\\
271	352.111119547628\\
272	366.975375997009\\
273	379.859819946761\\
274	390.80876387155\\
275	343.161944839632\\
276	372.057999517246\\
277	367.384101270417\\
278	324.772974670237\\
279	312.943157215813\\
280	255.379420895561\\
281	300.639949650055\\
282	264.714832478947\\
283	236.61717067409\\
284	273.525039410975\\
285	266.221986275821\\
286	228.903527246463\\
287	230.97433501778\\
288	276.776911012824\\
289	309.538299811083\\
290	282.637703928072\\
291	307.622061360336\\
292	304.833033643089\\
293	332.250906774937\\
294	314.875069551701\\
295	313.343504912068\\
296	254.930910944102\\
297	277.675327373466\\
298	334.585428722204\\
299	299.043492244536\\
300	263.985773983039\\
301	344.260742223876\\
302	355.578083686549\\
303	236.286848400552\\
304	288.67705489546\\
305	279.400239027314\\
306	321.802911623341\\
307	248.420697759442\\
308	343.998037425823\\
309	267.943323591844\\
310	352.842908541554\\
311	295.415392793917\\
312	363.596159868197\\
313	353.179291767717\\
314	388.204112682951\\
315	350.988671964482\\
316	347.217223225801\\
317	365.007955222581\\
318	417.665480443858\\
319	395.44913176198\\
320	403.548008804797\\
321	393.967754573835\\
322	438.970339238177\\
323	403.134702844458\\
324	439.287811020793\\
325	403.386399080808\\
326	420.526005266073\\
327	385.436482502422\\
328	347.34079277596\\
329	388.230924292144\\
330	327.383738922246\\
331	302.350939997539\\
332	299.263507917095\\
333	299.395549340639\\
334	305.241589163726\\
335	233.286095560167\\
336	279.350029913585\\
337	269.209573330048\\
338	219.455746453866\\
339	207.779952868281\\
340	233.328108829174\\
341	244.338367175343\\
342	229.188954011427\\
343	213.401424295249\\
344	192.37610081857\\
345	196.100957342729\\
346	194.310075256548\\
347	211.615617545283\\
348	141.993808386943\\
349	198.210794980122\\
350	215.57218570044\\
351	142.172843779654\\
352	232.448913517029\\
353	241.599925950222\\
354	146.467186983148\\
355	188.33359540081\\
356	210.297971738084\\
357	223.279615685626\\
358	200.906112299081\\
359	206.249482329653\\
360	223.950586708702\\
361	213.239209260056\\
362	253.228554607289\\
363	260.67562421448\\
364	292.739043286842\\
365	320.60610734196\\
366	345.149324183131\\
367	359.731143524055\\
368	395.6245398589\\
369	397.357067878771\\
370	409.442145446388\\
371	397.296641231854\\
372	372.047851198427\\
373	372.9032129801\\
374	304.923528528196\\
375	335.937525193965\\
376	338.314575868388\\
377	312.602666803989\\
378	283.600628079258\\
379	252.181569955069\\
380	315.850777267468\\
381	266.643900076145\\
382	269.212603873241\\
383	313.32344874278\\
384	269.38198251219\\
385	278.601147189172\\
386	329.620575599183\\
387	359.107623921404\\
388	300.166994166139\\
389	281.291731974726\\
390	268.560810644043\\
391	257.209826482261\\
392	282.636357420861\\
393	336.040151682212\\
394	326.32950300528\\
395	325.693294269414\\
396	242.82404363906\\
397	265.638415830683\\
398	359.726623173868\\
399	345.393549127586\\
400	291.631807127412\\
401	304.973266522462\\
402	357.495182140434\\
403	336.880900850978\\
404	357.186720358359\\
405	300.621956549519\\
406	373.609464142071\\
407	324.216131091108\\
408	375.637730126977\\
409	382.816498725388\\
410	370.193050082327\\
411	379.699210308316\\
412	354.187026418108\\
413	418.586095064517\\
414	411.058162737891\\
415	429.570512706399\\
416	385.886343699872\\
417	401.877962713666\\
418	375.668419645741\\
419	443.602431850835\\
420	441.920967055405\\
421	431.692638484051\\
422	358.337895318904\\
423	367.676545889624\\
424	368.411761077796\\
425	326.511223219364\\
426	338.560368434138\\
427	311.758052680349\\
428	281.851937517257\\
429	274.155208249614\\
430	248.705122731176\\
431	239.919930514407\\
432	223.03672915388\\
433	218.550521932034\\
434	250.764908458227\\
435	246.1930794089\\
436	206.759137808947\\
437	251.762013007389\\
438	232.412953455432\\
439	204.4793168051\\
440	191.521808605462\\
441	183.831723041167\\
442	154.138630172198\\
443	199.138716972178\\
444	192.040850001142\\
445	155.511129299568\\
446	234.423049686921\\
447	220.673414121676\\
448	209.050279567114\\
449	188.354273415483\\
450	210.765923785396\\
451	191.274329535097\\
452	187.24648744889\\
453	201.301582656423\\
454	194.305772356326\\
455	205.717723021685\\
456	205.960765427511\\
457	220.333449282045\\
458	236.572862144735\\
459	234.702830122613\\
460	277.977853376737\\
461	311.458935135984\\
462	322.23238078394\\
463	374.785886553298\\
464	371.945616081257\\
465	391.661232818461\\
466	407.168400703863\\
467	392.414193068445\\
468	388.758494941752\\
469	353.547447997216\\
470	354.306472071908\\
471	327.574385900417\\
472	264.757747565813\\
473	308.99899682069\\
474	304.702322320035\\
475	276.791444082868\\
476	321.530295997522\\
477	252.685896644492\\
478	313.494503569673\\
479	246.840448014099\\
480	247.995285311568\\
481	233.439874185696\\
482	288.766888226318\\
483	343.141417640418\\
484	289.174891651939\\
485	249.902004201553\\
486	258.832716429631\\
487	304.926924102959\\
488	351.557000991286\\
489	348.920708115644\\
490	313.068884196831\\
491	324.790464373248\\
492	303.767069648985\\
493	301.981045342775\\
494	276.927983518149\\
495	248.458578663303\\
496	277.152859717988\\
497	313.789106474854\\
498	293.235800127915\\
499	336.425887424834\\
500	278.678726249039\\
501	265.927373989769\\
502	372.191084845726\\
503	351.874653703528\\
504	317.575087167611\\
505	372.242703979638\\
506	365.563830981085\\
507	348.626515207231\\
508	365.947666894017\\
509	346.60377665042\\
510	343.961603818691\\
511	382.28525533991\\
512	394.713261498702\\
513	391.122283875168\\
514	441.027851569636\\
515	383.326947914707\\
516	435.327132521297\\
517	397.98495560932\\
518	389.542282374782\\
519	408.732966716765\\
520	348.419833302577\\
521	378.934381336599\\
522	367.999677635783\\
523	356.36893697596\\
524	298.364836198538\\
525	325.202338004424\\
526	255.903946569033\\
527	271.675336831697\\
528	276.25126907034\\
529	245.000168325281\\
530	215.947108352612\\
531	262.697110918715\\
532	263.249414368165\\
533	210.165859154956\\
534	235.515046317387\\
535	218.263957136088\\
536	200.520992350381\\
537	184.770152104525\\
538	188.987552896118\\
539	174.164836830295\\
540	202.673733645212\\
541	184.24976837449\\
542	223.666079998128\\
543	137.133295558627\\
544	193.741836873391\\
545	167.480328254931\\
546	176.231917608598\\
547	236.711002910225\\
548	183.992147048595\\
549	212.459092626944\\
550	184.702021775041\\
551	208.667081154688\\
552	220.839548854739\\
553	223.401353761384\\
554	237.431061559028\\
555	255.981244876152\\
556	295.253845028945\\
557	300.94859650329\\
558	337.747121203303\\
559	357.574200945341\\
560	386.916447934324\\
561	395.105687829865\\
562	395.350976497878\\
563	366.361375243233\\
564	392.84965190704\\
565	334.923505849709\\
566	355.67477808661\\
567	323.732511123643\\
568	313.883253216281\\
569	306.181668186015\\
570	263.666614114442\\
571	229.839885449193\\
572	229.598433741902\\
573	225.737556175852\\
574	211.54530809109\\
575	232.667320066557\\
576	255.872562672754\\
577	250.224688663561\\
578	262.014400913987\\
579	289.671305777363\\
580	258.972680304709\\
581	285.940774787226\\
582	253.679148318042\\
583	232.413539915065\\
584	266.984882622477\\
585	249.01768660116\\
586	245.322553858899\\
587	269.229885414868\\
588	241.655730977351\\
589	261.474160350513\\
590	227.320672951482\\
591	268.875284223589\\
592	288.780127034319\\
593	259.40814849582\\
594	274.720760424164\\
595	241.448757081503\\
596	247.221511263063\\
597	307.105624681376\\
598	263.852071003795\\
599	295.37357402069\\
600	292.316530011252\\
601	329.275928043163\\
602	299.546137002317\\
603	351.436452321845\\
604	335.343760745392\\
605	341.009778777036\\
606	360.453422606062\\
607	371.951548561241\\
608	366.902504092638\\
609	379.387080732364\\
610	365.5256296952\\
611	374.443203248885\\
612	366.594200898393\\
613	367.080189682017\\
614	363.452149564057\\
615	359.024525593873\\
616	334.858526055132\\
617	305.60568818413\\
618	314.499502738825\\
619	267.462074297351\\
620	241.898336633045\\
621	242.426330308996\\
622	231.797002835225\\
623	224.495839978972\\
624	190.151299539758\\
625	170.482607453173\\
626	172.72688982816\\
627	175.043855137972\\
628	180.224467635586\\
629	192.14871443986\\
630	161.561282327527\\
631	155.585446143918\\
632	139.344662336763\\
633	146.943234763308\\
634	139.520045123749\\
635	153.359424207135\\
636	147.5100410775\\
637	149.729748352204\\
638	153.571813336845\\
639	178.192023489058\\
640	168.672788741417\\
641	179.384034149256\\
642	149.318889875494\\
643	183.690721981789\\
644	177.794523997578\\
645	150.983285508728\\
646	161.933703766763\\
647	157.274619839951\\
648	170.776913179198\\
649	172.216322127308\\
650	220.491092858713\\
651	219.331546174457\\
652	261.825147311558\\
653	274.827382834852\\
654	294.059576030649\\
655	332.522758695611\\
656	359.651072510239\\
657	355.130101441137\\
658	359.178675012792\\
659	347.805529028533\\
660	329.135201820177\\
661	306.426864202396\\
662	280.091842262905\\
663	275.216409874451\\
664	255.186466511654\\
665	247.746944328932\\
666	229.640441211148\\
667	220.676998443608\\
668	237.640610890632\\
669	198.23401493993\\
670	197.438682529869\\
671	208.162020797613\\
672	222.100854737106\\
673	226.783569965487\\
674	230.293007923984\\
675	217.128061005209\\
676	240.067545972732\\
677	246.507729848048\\
678	247.30285907434\\
679	231.160298954232\\
680	233.167421633468\\
681	230.348510422546\\
682	216.430723529243\\
683	230.406195767979\\
684	228.152741009194\\
685	224.737788889101\\
686	236.458164835806\\
687	226.177412454239\\
688	252.734815939487\\
689	243.796802654995\\
690	222.812746269184\\
691	224.473060475355\\
692	271.78164368436\\
693	252.539091733214\\
694	263.904147436302\\
695	266.44500945683\\
696	316.818542330818\\
697	330.367281146623\\
698	315.352006591351\\
699	345.957885221136\\
700	317.768992081404\\
701	342.522110083279\\
702	331.592524754784\\
703	349.209311622609\\
704	349.192314807556\\
705	360.900063602359\\
706	359.138235684731\\
707	377.420902777891\\
708	363.157387976365\\
709	345.711909787242\\
710	359.174303803041\\
711	343.393533657201\\
712	331.512617365852\\
713	311.370840519373\\
714	287.05964644944\\
715	260.45277315849\\
716	248.093455776511\\
717	228.766243273574\\
718	199.086440757996\\
719	189.052855821044\\
720	196.507429545197\\
721	174.47524881568\\
722	169.41198564303\\
723	166.504847001334\\
724	185.213752265041\\
725	193.428441152468\\
726	159.916947967417\\
727	151.759124691572\\
728	153.807901273325\\
729	133.346347367094\\
730	144.915090736411\\
731	149.320244557967\\
732	144.991934461794\\
733	162.814581135367\\
734	168.225843604575\\
735	175.328085616857\\
736	177.358798936249\\
737	202.887919039505\\
738	175.784357273062\\
739	193.285586445437\\
740	166.049233194602\\
741	168.845414191451\\
742	169.170846288215\\
743	178.974062850836\\
744	180.319629312981\\
745	189.190718948537\\
746	224.331522014182\\
747	212.362029606222\\
748	264.193294232767\\
749	298.305018492363\\
750	329.51143795821\\
751	329.094772044316\\
752	352.796102400325\\
753	369.808822174357\\
754	363.566920345417\\
755	358.183157184677\\
756	351.1737405209\\
757	323.453729295846\\
758	321.34669934048\\
759	309.168585928841\\
760	266.098298901241\\
761	235.000055930624\\
762	237.26346736066\\
763	236.866039008232\\
764	223.952127543516\\
765	250.915044623501\\
766	229.17936656329\\
767	258.221942230698\\
768	287.965156279758\\
769	286.952873437096\\
770	240.839513058378\\
771	234.684181434473\\
772	298.383880129311\\
773	327.853538889452\\
774	323.741802885027\\
775	288.674232909585\\
776	258.424901466994\\
777	313.500948802208\\
778	343.726918356087\\
779	284.206526564573\\
780	340.401166790263\\
781	255.878104168219\\
782	326.954479017045\\
783	314.504107411888\\
784	296.689018951565\\
785	248.312890670116\\
786	340.205453674288\\
787	280.050176231879\\
788	277.353177307231\\
789	277.248514740225\\
790	355.497413682368\\
791	336.595709553656\\
792	362.993286238028\\
793	364.538405809272\\
794	377.508391144426\\
795	325.65250146188\\
796	372.093871419284\\
797	352.119844427774\\
798	385.233218211862\\
799	351.642992181507\\
800	441.761321408933\\
801	410.342834915077\\
802	376.322094570998\\
803	446.749444890996\\
804	436.282753501902\\
805	418.973175873143\\
806	380.433584243508\\
807	407.107377994262\\
808	350.405010327838\\
809	396.405161436964\\
810	334.310098125412\\
811	307.209045535312\\
812	347.983817079947\\
813	263.801766436919\\
814	339.263836685331\\
815	289.294721721183\\
816	270.070787013005\\
817	264.490829269418\\
818	218.023384068989\\
819	215.874721361747\\
820	251.885563794024\\
821	230.752151821587\\
822	217.551866152653\\
823	199.222274137681\\
824	200.315629643839\\
825	168.379231832107\\
826	183.785059254213\\
827	201.559092673922\\
828	145.815299464231\\
829	164.227257616321\\
830	181.305461759697\\
831	207.674986462191\\
832	163.004632182045\\
833	265.944047258233\\
834	170.517606102598\\
835	209.077917934429\\
836	253.138916637431\\
837	197.977746468852\\
838	195.529921334227\\
839	220.375932663362\\
840	217.085621379445\\
841	231.781872252619\\
842	246.992467614642\\
843	274.071130030694\\
844	276.395629951914\\
845	323.056873321183\\
846	331.698596751969\\
847	375.392340478506\\
848	389.502463069001\\
849	410.690587905262\\
850	350.469585756701\\
851	410.189984487399\\
852	401.344722007223\\
853	390.132247877271\\
854	328.445129618721\\
855	311.912545367855\\
856	338.292842248498\\
857	308.201293570177\\
858	245.250198832121\\
859	279.343514924731\\
860	301.02985745303\\
861	319.241150792595\\
862	244.006795063165\\
863	294.792065999251\\
864	329.356349716398\\
865	252.035902859221\\
866	276.474220246999\\
867	279.937219434251\\
868	339.213462190222\\
869	317.821828744182\\
870	366.645328584158\\
871	371.535235702878\\
872	358.620578280698\\
873	348.827408551612\\
874	272.233289195488\\
875	341.276992741515\\
876	318.90643603798\\
877	347.367891270174\\
878	361.908758167731\\
879	259.288197520925\\
880	271.141944847755\\
881	333.749614206455\\
882	377.349584586714\\
883	370.715119573755\\
884	306.172648494322\\
885	339.549316520075\\
886	355.299594092395\\
887	287.013675585035\\
888	358.790077979792\\
889	366.304700889305\\
890	368.093029692115\\
891	359.41186068618\\
892	406.362251754709\\
893	343.032517074838\\
894	358.56310513524\\
895	422.027691728083\\
896	377.81158384567\\
897	402.248376805179\\
898	389.108596424902\\
899	385.692949942068\\
900	378.829536181961\\
901	411.32066713796\\
902	412.118175510091\\
903	380.070644346977\\
904	392.458303922324\\
905	368.670816180503\\
906	349.802940886597\\
907	296.893732604937\\
908	267.952914364926\\
909	314.441691256439\\
910	262.001229366074\\
911	218.919152694708\\
912	278.159453125995\\
913	261.834988797297\\
914	241.936806662956\\
915	228.537379262686\\
916	205.778290251634\\
917	208.412227408956\\
918	193.561490047949\\
919	174.853339678108\\
920	176.445936505525\\
921	154.369022172852\\
922	137.529691182416\\
923	170.903257575169\\
924	149.253392028197\\
925	180.868372159889\\
926	200.923618754499\\
927	166.804669163963\\
928	216.926015601158\\
929	217.552665776794\\
930	162.743054221604\\
931	221.009550884298\\
932	210.383576275604\\
933	197.096036491022\\
934	172.750329562159\\
935	198.857021995947\\
936	204.59304281942\\
937	206.644021689509\\
938	251.887064646862\\
939	246.558207861101\\
940	287.885454009913\\
941	317.974856078826\\
942	333.817253010068\\
943	371.012789271346\\
944	387.020345273908\\
945	381.494518212727\\
946	380.386930264064\\
947	362.776751812134\\
948	363.937935820019\\
949	340.535173197347\\
950	360.644832632746\\
951	329.936253429852\\
952	256.025526693487\\
953	287.887245822034\\
954	263.278577943502\\
955	307.016644769722\\
956	321.142063785644\\
957	281.61028503397\\
958	283.504460734894\\
959	299.923340005825\\
960	282.590366527916\\
961	333.03022753604\\
962	335.597642762945\\
963	276.139772280543\\
964	307.901308804241\\
965	323.707589946105\\
966	277.089596873158\\
967	347.552961508904\\
968	291.940544051264\\
969	245.193033555164\\
970	286.341778868552\\
971	342.515304137003\\
972	289.03092871395\\
973	249.554216937097\\
974	332.716933283288\\
975	224.130449966879\\
976	325.461623653118\\
977	346.251714965542\\
978	283.937742585236\\
979	255.940472312072\\
980	339.770758219431\\
981	341.740901229446\\
982	341.995132605457\\
983	342.752471898959\\
984	334.731521473408\\
985	353.424132073614\\
986	383.926050610085\\
987	371.90123375284\\
988	374.450076624778\\
989	402.540896137603\\
990	373.632857793767\\
991	422.974670324882\\
992	370.299370792674\\
993	405.777613762291\\
994	398.391039212963\\
995	382.90742662054\\
996	374.330035594456\\
997	387.279780227638\\
998	403.594238744367\\
999	403.875043108485\\
1000	362.512072757987\\
1001	349.869882450483\\
1002	346.339837648239\\
1003	285.900124966148\\
1004	283.938992013731\\
1005	258.003379871872\\
1006	292.450176099074\\
1007	244.26292561068\\
1008	259.779104347829\\
1009	191.510421254075\\
1010	203.043056230854\\
1011	230.569737748965\\
1012	191.149913243097\\
1013	219.97896898829\\
1014	184.069896491353\\
1015	175.974654911101\\
1016	152.860296804779\\
1017	153.378212952072\\
1018	141.709446363632\\
1019	156.422081329323\\
1020	151.822533977017\\
1021	141.890570205693\\
1022	144.108625911663\\
1023	188.234357113807\\
1024	150.244369055935\\
1025	152.731327804808\\
1026	189.37479665115\\
1027	154.647809204964\\
1028	171.339659361701\\
1029	168.713853070293\\
1030	164.039564460993\\
1031	170.722041667236\\
1032	168.612799307645\\
1033	185.339032983855\\
1034	230.010080769032\\
1035	214.033970323606\\
1036	270.146012489506\\
1037	304.184381600748\\
1038	300.701307339358\\
1039	353.764714812116\\
1040	368.012171010852\\
1041	376.592010428251\\
1042	386.891210657315\\
1043	370.163159786209\\
1044	351.224897115612\\
1045	353.351969936772\\
1046	345.622895686651\\
1047	301.341460326245\\
1048	306.014071517772\\
1049	247.434443870057\\
1050	288.281376080058\\
1051	291.468676665214\\
1052	277.109918304181\\
1053	234.548051051303\\
1054	239.033002535988\\
1055	260.450588426201\\
1056	251.420488996826\\
1057	244.660804329325\\
1058	258.364357758364\\
1059	296.199129826107\\
1060	321.27406025301\\
1061	304.532435934505\\
1062	239.95687961159\\
1063	297.67525575808\\
1064	319.550032160895\\
1065	239.488276164828\\
1066	219.556491993769\\
1067	270.268288172611\\
1068	245.377750860464\\
1069	272.180194498009\\
1070	228.005572624136\\
1071	242.133758954701\\
1072	200.678915284188\\
1073	221.723682967758\\
1074	292.548783248989\\
1075	283.220460827717\\
1076	239.919973556541\\
1077	308.757932887219\\
1078	318.673094471136\\
1079	312.639350818327\\
1080	332.570759705923\\
1081	320.062769677598\\
1082	323.264622328873\\
1083	347.987662101314\\
1084	312.955255050373\\
1085	376.134486733381\\
1086	377.121167342639\\
1087	345.934043735789\\
1088	378.421581363688\\
1089	379.326666755514\\
1090	373.391445935558\\
1091	380.052844679287\\
1092	371.288220309121\\
1093	363.077310088233\\
1094	357.133659379866\\
1095	330.088208679071\\
1096	324.207440806791\\
1097	325.564058540017\\
1098	325.938361128069\\
1099	266.179471777175\\
1100	258.181959327533\\
1101	270.075115287865\\
1102	218.615632449712\\
1103	206.373502417489\\
1104	190.667580947646\\
1105	175.392945644155\\
1106	177.812659562072\\
1107	169.924924003054\\
1108	176.890390696054\\
1109	175.51800703194\\
1110	167.517975414173\\
1111	152.672402172055\\
1112	151.516401347754\\
1113	134.38994512817\\
1114	149.427499796966\\
1115	147.123808602101\\
1116	142.955843048046\\
1117	142.780281710286\\
1118	162.657963383036\\
1119	164.258794245701\\
1120	153.970648517976\\
1121	170.463671124798\\
1122	157.064029595168\\
1123	141.178162969064\\
1124	145.616943605901\\
1125	147.247592641373\\
1126	151.679324096107\\
1127	149.796047285568\\
1128	168.945997786711\\
1129	163.106070391351\\
1130	177.529833860102\\
1131	235.825318497494\\
1132	249.785676915345\\
1133	262.106201539456\\
1134	284.697823140909\\
1135	311.316725341753\\
1136	332.976718915547\\
1137	335.723776430292\\
1138	334.210004782471\\
1139	323.609664951779\\
1140	303.80439918181\\
1141	289.54577470262\\
1142	274.797659188764\\
1143	259.92514987157\\
1144	244.904073065658\\
1145	223.703003059231\\
1146	219.608447258425\\
1147	203.376883616063\\
1148	205.543258546877\\
1149	204.264298390648\\
1150	202.444685031338\\
1151	214.171458597523\\
1152	227.308750299963\\
1153	211.341570439145\\
1154	229.073504351002\\
1155	230.616993098051\\
1156	222.782437479736\\
1157	225.848843062031\\
1158	235.468190756076\\
1159	229.030201247867\\
1160	229.28142733136\\
1161	229.178392988292\\
1162	209.289180971354\\
1163	189.704100646362\\
1164	192.483913144291\\
1165	183.035313165816\\
1166	204.372212025296\\
1167	208.755893666357\\
1168	217.631975683155\\
1169	237.93307070197\\
1170	208.810433424577\\
1171	220.436163648439\\
1172	256.364109904169\\
1173	259.478783980628\\
1174	257.641872310412\\
1175	263.232403952177\\
1176	308.325029359399\\
1177	308.154843483053\\
1178	314.281833381557\\
1179	313.710204034966\\
1180	320.144526219889\\
1181	320.877805320357\\
1182	335.009284994537\\
1183	335.075700654776\\
1184	353.152246938521\\
1185	346.290281198343\\
1186	371.305248002725\\
1187	356.736576592336\\
1188	365.506762324735\\
1189	345.17536832938\\
1190	338.767073407301\\
1191	337.617033727679\\
1192	320.886046558618\\
1193	295.737972129759\\
1194	277.420772534833\\
1195	252.246797910266\\
1196	230.949503868577\\
1197	222.249305246504\\
1198	204.159368628748\\
1199	193.529533767323\\
1200	173.287991126387\\
1201	161.333366995302\\
1202	177.186459029043\\
1203	183.524370973043\\
1204	167.187482314843\\
1205	205.463728541855\\
1206	162.610044611846\\
1207	157.399501088071\\
1208	154.457891094002\\
1209	159.962479689511\\
1210	143.49403813247\\
1211	146.881769852314\\
1212	150.126661686246\\
1213	146.716027663418\\
1214	154.068314355175\\
1215	135.488524332459\\
1216	158.236670490297\\
1217	175.202100498231\\
1218	161.720288630545\\
1219	145.402922184124\\
1220	160.775107638915\\
1221	152.26477901461\\
1222	150.79426237729\\
1223	148.719640626359\\
1224	158.056714789514\\
1225	174.303402713354\\
1226	201.960524906792\\
1227	212.172875482549\\
1228	241.490747751138\\
1229	263.25450539137\\
1230	299.761509411405\\
1231	326.182765560113\\
1232	341.019673624204\\
1233	343.191140548597\\
1234	353.522531742964\\
1235	353.694739259989\\
1236	313.67290473098\\
1237	309.706587673915\\
1238	287.513048856347\\
1239	279.429618075108\\
1240	247.315479846442\\
1241	227.07412486193\\
1242	237.606780237257\\
1243	248.939420881337\\
1244	203.430814110315\\
1245	205.942458485729\\
1246	234.293058858654\\
1247	240.885856998347\\
1248	237.845237189093\\
1249	265.386568939926\\
1250	245.230948186457\\
1251	267.39122499508\\
1252	273.308768606637\\
1253	229.449450912071\\
1254	226.480815992258\\
1255	259.428491024438\\
1256	225.174982104862\\
1257	265.960519590923\\
1258	273.383441499879\\
1259	252.625246490041\\
1260	252.364743199126\\
1261	241.7930389099\\
1262	178.906680690695\\
1263	224.616256905683\\
1264	208.309052676421\\
1265	269.337645422664\\
1266	267.669464358291\\
1267	284.969773329492\\
1268	224.916009455693\\
1269	245.205999991407\\
1270	296.587253497943\\
1271	283.866466897924\\
1272	289.062548275197\\
1273	326.951221428739\\
1274	329.938924153092\\
1275	321.867207686518\\
1276	365.66939760806\\
1277	361.69999697487\\
1278	366.909057159553\\
1279	322.454886307271\\
1280	344.055987478688\\
1281	384.03541630161\\
1282	378.232709001625\\
1283	381.995155103124\\
1284	375.2934365242\\
1285	359.831631686726\\
1286	344.190601465459\\
1287	331.510685383679\\
1288	320.582750192308\\
1289	297.601987995017\\
1290	278.372749002387\\
1291	253.819108629574\\
1292	229.159405603181\\
1293	202.754327644501\\
1294	201.641009548723\\
1295	198.700495597529\\
1296	174.10517437541\\
1297	166.791984861395\\
1298	176.680480413876\\
1299	164.4146112361\\
1300	173.099175056618\\
1301	185.070930429205\\
1302	180.209822381252\\
1303	170.124143343378\\
1304	171.710943983221\\
1305	170.345595091518\\
1306	168.538495509586\\
1307	173.258603756935\\
1308	170.154624751749\\
1309	163.256518562859\\
1310	174.259984230804\\
1311	169.376696030612\\
1312	157.987428313905\\
1313	159.409037581488\\
1314	158.187289714574\\
1315	149.873287148353\\
1316	156.567634918205\\
1317	160.475969284562\\
1318	156.676372400932\\
1319	161.417995110147\\
1320	171.296023041375\\
1321	183.883847131357\\
1322	184.902320787317\\
1323	232.813615131304\\
1324	242.325148167426\\
1325	266.043006393285\\
1326	312.249648560702\\
1327	322.672646727797\\
1328	338.827230612732\\
1329	350.226560726588\\
1330	365.059703174314\\
1331	367.671907491068\\
1332	338.21318654981\\
1333	302.896959461423\\
1334	310.734321736557\\
1335	261.762239026979\\
1336	293.731879683915\\
1337	278.680184265274\\
1338	217.49719271066\\
1339	229.167422618052\\
1340	271.505592648474\\
1341	241.5929762363\\
1342	280.097432682505\\
1343	254.057420411883\\
1344	241.394426971187\\
1345	252.638121708849\\
1346	281.719822927488\\
1347	242.29673320017\\
1348	259.336368382565\\
1349	242.916743508397\\
1350	283.954178793192\\
1351	283.004108943895\\
1352	267.973969580901\\
1353	279.985334630641\\
1354	298.293590238631\\
1355	243.810617015294\\
1356	202.812803132548\\
1357	223.369941989154\\
1358	261.861733535429\\
1359	191.429874155505\\
1360	314.300031799075\\
1361	241.527570054696\\
1362	214.678784363488\\
1363	300.040665365376\\
1364	303.036466927878\\
1365	325.474559001301\\
1366	261.367710779981\\
1367	285.13933039339\\
1368	288.177467954598\\
1369	305.665993933512\\
1370	327.526577569674\\
1371	332.999992504866\\
1372	344.865522331157\\
1373	374.570919194063\\
1374	392.594422412898\\
1375	394.40039995786\\
1376	404.697543191148\\
1377	405.90169401154\\
1378	408.687341796327\\
1379	392.200819186748\\
1380	406.676077268417\\
1381	365.201056154557\\
1382	353.962953671117\\
1383	347.910505660306\\
1384	331.247083612762\\
1385	324.291266295793\\
1386	303.957855726738\\
1387	273.425825042537\\
1388	259.031188539521\\
1389	229.781154090808\\
1390	206.73151965825\\
1391	215.465994952032\\
1392	184.41126359364\\
1393	187.039707548419\\
1394	169.119095591093\\
1395	169.630191122742\\
1396	179.901481397588\\
1397	203.511763315625\\
1398	178.813372298308\\
1399	166.54634375783\\
1400	159.564772860234\\
1401	147.109928750571\\
1402	152.828594770146\\
1403	153.261585318364\\
1404	153.833363922834\\
1405	153.713831267434\\
1406	168.253162358425\\
1407	162.396862844488\\
1408	167.414637255577\\
1409	160.967853833042\\
1410	185.094686428353\\
1411	166.69892861392\\
1412	157.668134201909\\
1413	176.745369151185\\
1414	151.248310770362\\
1415	166.938078873136\\
1416	167.328588530046\\
1417	189.812918319553\\
1418	213.660783939661\\
1419	231.974973202636\\
1420	235.019970281676\\
1421	281.07730295811\\
1422	313.66078955927\\
1423	340.829515927466\\
1424	347.192528653478\\
1425	370.824591485584\\
1426	378.528664380091\\
1427	343.83137292459\\
1428	349.211776557794\\
1429	316.394606743924\\
1430	316.321601413852\\
1431	271.043804550393\\
1432	286.084238992197\\
1433	254.13372328918\\
1434	225.559403370886\\
1435	248.822242154844\\
1436	219.800945571114\\
1437	229.92161873171\\
1438	240.599679852084\\
1439	294.470860693614\\
1440	264.911968879526\\
1441	258.407133633845\\
1442	326.190987968641\\
1443	301.646575763437\\
1444	262.389870966873\\
1445	305.387073753634\\
1446	254.030148071398\\
1447	302.121698619239\\
1448	333.451463341695\\
1449	279.982661064348\\
1450	323.675899392544\\
1451	282.181379086644\\
1452	303.118245944351\\
1453	239.178769960334\\
1454	235.32831609009\\
1455	327.476142022147\\
1456	256.633235024858\\
1457	319.620795037562\\
1458	334.24276186409\\
1459	331.878913631111\\
1460	333.487353004403\\
1461	257.979944099214\\
1462	358.414308607448\\
1463	358.3624485377\\
1464	369.894587979547\\
1465	357.187955905386\\
1466	347.805306957572\\
1467	366.568435031533\\
1468	359.206722845972\\
1469	394.697298676297\\
1470	415.412270362544\\
1471	361.723434963268\\
1472	427.278196431047\\
1473	424.984028690814\\
1474	431.317851249386\\
1475	432.133810838299\\
1476	368.725583720637\\
1477	361.815197878652\\
1478	352.316507177623\\
1479	359.619794385408\\
1480	381.916810492559\\
1481	347.682380753404\\
1482	320.58859390255\\
1483	314.226556279312\\
1484	278.161846827215\\
1485	250.288651529901\\
1486	234.054208054042\\
1487	216.670004928935\\
1488	212.387975514771\\
1489	217.159199690072\\
1490	211.309432619846\\
1491	192.708928193785\\
1492	223.926061565998\\
1493	197.92352872861\\
1494	210.255140988806\\
1495	189.15796052052\\
1496	176.973949359665\\
1497	184.069057911857\\
1498	175.76327352702\\
1499	156.261174005792\\
1500	146.584221733213\\
1501	205.445988550554\\
1502	158.669975138749\\
1503	169.18043721138\\
1504	142.406472838107\\
1505	209.938746182703\\
1506	187.945603347851\\
1507	187.400072946871\\
1508	240.52832206434\\
1509	173.320158162147\\
1510	200.212771029687\\
1511	200.607063454238\\
1512	203.227720024583\\
1513	218.875151942539\\
1514	245.83542910458\\
1515	230.040699838561\\
1516	276.025261318719\\
1517	314.480233885827\\
1518	333.485418781228\\
1519	360.230289647958\\
1520	360.637762254838\\
1521	365.813646325308\\
1522	388.057928055562\\
1523	354.21023719873\\
1524	354.393225286332\\
1525	337.675331551361\\
1526	335.627405618983\\
1527	304.590362133383\\
1528	313.218842256623\\
1529	283.256511559451\\
1530	304.992258882454\\
1531	227.273513807927\\
1532	295.468630336478\\
1533	303.147067501929\\
1534	226.590296735477\\
1535	305.904664210813\\
1536	325.530794701174\\
1537	278.408062980175\\
1538	280.197994752273\\
1539	286.031419386347\\
1540	351.264328869664\\
1541	285.969433417551\\
1542	332.464981770589\\
1543	338.737018011687\\
1544	272.533172413657\\
1545	293.921293034678\\
1546	327.582933847134\\
1547	326.579455350336\\
1548	298.402923226383\\
1549	310.823885230491\\
1550	272.042623755716\\
1551	261.978071497606\\
1552	342.157874184332\\
1553	261.055762448116\\
1554	266.764906512553\\
1555	302.152405834072\\
1556	303.333323311425\\
1557	307.227357856191\\
1558	359.820312091247\\
1559	293.276355435538\\
1560	362.909987719134\\
1561	370.407022436997\\
1562	319.051864714038\\
1563	341.474802753083\\
1564	341.362390715157\\
1565	397.090361007224\\
1566	378.669200530176\\
1567	427.925558111684\\
1568	361.301421244073\\
1569	435.668906898019\\
1570	447.899955295795\\
1571	450.130672288413\\
1572	433.131538667916\\
1573	401.521894854069\\
1574	417.823266032623\\
1575	389.858842126065\\
1576	346.269291838781\\
1577	367.560782199593\\
1578	313.155661326301\\
1579	329.059371334202\\
1580	304.45940851975\\
1581	257.823538945563\\
1582	269.038074325777\\
1583	211.147476527199\\
1584	227.782129403324\\
1585	220.398040269175\\
1586	190.736526823702\\
1587	219.241988188444\\
1588	193.223996979124\\
1589	220.484317718282\\
1590	186.81584054765\\
1591	175.52023556046\\
1592	171.493878235314\\
1593	157.929383268779\\
1594	142.60131776649\\
1595	139.809764271672\\
1596	175.04392041042\\
1597	153.80504948091\\
1598	150.65933715832\\
1599	161.376435618443\\
1600	204.048609404717\\
1601	199.839061363698\\
1602	162.997319638586\\
1603	164.348412629843\\
1604	222.854314133903\\
1605	186.057150284878\\
1606	189.332441941175\\
1607	190.336836615774\\
1608	202.194954698202\\
1609	205.319379685594\\
1610	232.94799573004\\
1611	250.522153183812\\
1612	240.938224144943\\
1613	284.005405209996\\
1614	334.236024092379\\
1615	360.635456964273\\
1616	387.250904908444\\
1617	395.093510637524\\
1618	379.129084127059\\
1619	404.685368789227\\
1620	369.88277115242\\
1621	373.5637343192\\
1622	327.685407808396\\
1623	325.150390548488\\
1624	349.836886198747\\
1625	274.997361619903\\
1626	311.759115471476\\
1627	329.756864613767\\
1628	275.144451366048\\
1629	250.692150187611\\
1630	305.004348021391\\
1631	331.686524054516\\
1632	304.553389859717\\
1633	312.069968942283\\
1634	241.849206747425\\
1635	272.549550310302\\
1636	260.152699915075\\
1637	314.251205611212\\
1638	350.629479349306\\
1639	261.470878566025\\
1640	261.785973445742\\
1641	287.29916999335\\
1642	361.010059970968\\
1643	326.812597674902\\
1644	327.248629136568\\
1645	330.197310823525\\
1646	334.577703819057\\
1647	288.174132782694\\
1648	326.996040477156\\
1649	342.376565378217\\
1650	335.246717721651\\
1651	260.013079854236\\
1652	322.635200508667\\
1653	281.889503352464\\
1654	351.979299346164\\
1655	367.631998287592\\
1656	369.287160406304\\
1657	373.94939617409\\
1658	384.191280584917\\
1659	365.936733160935\\
1660	378.034648934012\\
1661	373.35888316233\\
1662	406.504506953074\\
1663	381.761413790532\\
1664	426.96347712845\\
1665	418.96972167394\\
1666	401.856184933913\\
1667	398.066451043592\\
1668	402.127624661351\\
1669	427.587882987769\\
1670	403.31094824733\\
1671	351.454119610938\\
1672	375.289822478513\\
1673	352.642825443654\\
1674	316.312748356103\\
1675	345.201930212428\\
1676	261.967662693123\\
1677	320.335447331705\\
1678	230.336492769355\\
1679	216.746893457463\\
1680	265.722907940002\\
1681	212.774299381161\\
1682	185.270185811359\\
1683	201.997491045756\\
1684	198.832195108008\\
1685	221.275250321649\\
1686	175.687908321833\\
1687	168.950373688049\\
1688	157.335380887435\\
1689	150.322410123615\\
1690	140.155862217462\\
1691	152.773845259944\\
1692	147.993024965794\\
1693	159.206057327422\\
1694	181.699829478077\\
1695	152.317463093155\\
1696	186.700900979419\\
1697	182.053119295158\\
1698	192.702752439371\\
1699	184.787818723558\\
1700	161.907387532078\\
1701	193.727686634837\\
1702	181.397966406617\\
1703	182.202404963947\\
1704	205.136451915962\\
1705	191.248996153208\\
1706	209.682869932792\\
1707	252.055163921614\\
1708	283.65146196028\\
1709	308.00145232187\\
1710	320.794907891601\\
1711	360.451833799285\\
1712	368.202704421224\\
1713	394.518841039662\\
1714	399.962553433384\\
1715	407.957753992454\\
1716	395.697395022911\\
1717	379.875477587072\\
1718	319.188150903879\\
1719	312.074980741374\\
1720	295.616656601046\\
1721	270.277579476691\\
1722	337.999670187051\\
1723	305.992282515778\\
1724	260.745972795871\\
1725	311.264724181785\\
1726	307.610795624043\\
1727	277.173027151037\\
1728	305.730873597355\\
1729	297.309997745713\\
1730	324.634776656044\\
1731	272.333529344282\\
1732	272.154897285768\\
1733	262.111599223041\\
1734	245.831284833387\\
1735	251.420516131382\\
1736	292.717193948614\\
1737	256.320753007676\\
1738	245.645133666092\\
1739	221.430597015759\\
1740	253.933763593752\\
1741	220.53560206086\\
1742	259.810338134838\\
1743	234.760119597518\\
1744	278.671794596765\\
1745	296.645229702192\\
1746	250.188745435952\\
1747	251.73604738656\\
1748	268.608352469476\\
1749	295.390770251473\\
1750	304.414988407476\\
1751	309.974356986986\\
1752	334.669319516734\\
1753	315.329452986046\\
1754	349.326267141094\\
1755	340.240622181465\\
1756	373.67240068754\\
1757	332.067012880365\\
1758	367.630549536775\\
1759	385.830771026613\\
1760	388.576240960434\\
1761	351.083189049563\\
1762	397.883173632588\\
1763	388.498345781783\\
1764	387.314555805625\\
1765	364.680844242803\\
1766	384.285808838799\\
1767	375.12890713063\\
1768	343.046927112886\\
1769	364.288499907835\\
1770	347.795401905448\\
1771	274.033036091108\\
1772	333.065494833414\\
1773	290.137095949285\\
1774	273.505690210482\\
1775	227.567801540996\\
1776	233.495662757804\\
1777	184.928116542568\\
1778	253.777497352481\\
1779	222.457308413272\\
1780	227.391061164119\\
1781	201.116978337556\\
1782	169.174090915288\\
1783	160.058183608758\\
1784	156.375631563752\\
1785	144.241539084746\\
1786	144.682275105201\\
1787	154.472955230003\\
1788	135.713774112819\\
1789	169.418576293386\\
1790	179.255497686524\\
1791	164.527040447344\\
1792	187.632169812174\\
1793	165.364422135283\\
1794	181.136667027137\\
1795	167.652750904497\\
1796	157.069187843002\\
1797	179.122755534531\\
1798	156.527967564977\\
1799	160.98347088746\\
1800	166.456294030487\\
1801	197.934353976354\\
1802	210.399603788929\\
1803	214.207912116745\\
1804	240.92671873893\\
1805	271.902723925538\\
1806	313.474050429963\\
1807	348.864507745774\\
1808	368.819753874656\\
1809	401.770284165626\\
1810	407.801044696093\\
1811	354.577038635677\\
1812	387.441681202372\\
1813	359.904735777848\\
1814	327.082850668055\\
1815	301.924656895311\\
1816	298.911952370277\\
1817	249.716755012666\\
1818	231.896472478176\\
1819	232.854669161329\\
1820	220.70688566465\\
1821	230.624328428892\\
1822	219.952988468903\\
1823	296.228509917996\\
1824	275.807926739848\\
1825	227.294577323016\\
1826	241.112619327224\\
1827	274.805762601206\\
1828	291.06675495482\\
1829	321.663297280573\\
1830	249.908739032874\\
1831	311.378803943008\\
1832	277.751058374901\\
1833	228.44213861058\\
1834	246.24504205947\\
1835	314.496081397804\\
1836	205.803175363961\\
1837	212.614531721047\\
1838	274.781564334382\\
1839	295.485132924471\\
1840	274.511544937774\\
1841	281.899790493295\\
1842	312.787163308441\\
1843	221.273384981602\\
1844	263.411495453213\\
1845	299.913891450765\\
1846	316.286058718319\\
1847	286.557138890496\\
1848	311.643235651406\\
1849	348.341031431198\\
1850	373.255500384471\\
1851	352.297207540081\\
1852	373.914625439684\\
1853	392.031668753564\\
1854	390.744799567966\\
1855	392.121050046504\\
1856	388.90355966933\\
1857	377.979080477478\\
1858	383.915360704067\\
1859	373.440596897302\\
1860	400.798122457876\\
1861	394.918699295202\\
1862	402.403264476701\\
1863	346.154354660146\\
1864	350.352268853918\\
1865	346.536574505058\\
1866	312.752516071272\\
1867	283.5842382646\\
1868	256.008553466314\\
1869	239.668467095156\\
1870	220.082992233747\\
1871	257.540457261096\\
1872	232.006223547141\\
1873	231.961876010194\\
1874	183.950292720562\\
1875	221.724222216377\\
1876	186.871404222553\\
1876	114.164321936084\\
1875	146.602870457194\\
1874	106.671308466474\\
1873	163.537874204796\\
1872	168.364232804118\\
1871	185.747928713224\\
1870	142.711494909891\\
1869	155.00239957205\\
1868	173.839091868923\\
1867	196.709969384203\\
1866	234.7248147611\\
1865	269.999596214293\\
1864	274.278424551594\\
1863	278.362951029744\\
1862	298.308219683576\\
1861	318.891897519283\\
1860	324.751051523206\\
1859	280.550424129739\\
1858	296.373381683861\\
1857	288.518567270225\\
1856	298.796534140754\\
1855	309.17149181901\\
1854	307.30837525174\\
1853	314.932776744933\\
1852	295.503347291392\\
1851	270.33503399253\\
1850	293.118743036315\\
1849	268.574872577648\\
1848	224.843466716042\\
1847	199.363459883893\\
1846	247.243960936527\\
1845	221.629314360719\\
1844	175.91644085504\\
1843	137.556411419538\\
1842	222.725351735234\\
1841	200.268026910381\\
1840	200.09128464155\\
1839	223.336374865764\\
1838	193.242463113979\\
1837	136.012653104252\\
1836	116.194345710786\\
1835	226.087818175441\\
1834	156.557721888822\\
1833	131.860516995741\\
1832	204.277465647059\\
1831	228.776188534013\\
1830	168.525312977594\\
1829	235.550667960466\\
1828	218.418533450639\\
1827	212.148401442273\\
1826	173.31440143107\\
1825	144.962959028713\\
1824	189.009220123544\\
1823	208.992832185544\\
1822	145.030435272645\\
1821	159.03496062592\\
1820	149.07726365084\\
1819	156.070871334322\\
1818	157.540694036308\\
1817	172.23969953359\\
1816	230.016197741481\\
1815	225.441866508286\\
1814	256.384253852966\\
1813	290.535961357986\\
1812	316.3217593329\\
1811	267.739280297299\\
1810	335.366039645585\\
1809	321.870087021093\\
1808	282.595341428728\\
1807	270.618305885783\\
1806	230.878610536254\\
1805	188.126295511665\\
1804	151.020392923639\\
1803	121.540480540718\\
1802	116.457319295089\\
1801	109.340363482026\\
1800	85.422973995501\\
1799	83.8882633030359\\
1798	81.8289797290516\\
1797	96.0461801603434\\
1796	83.0588548893498\\
1795	97.3132190495371\\
1794	111.145938354595\\
1793	90.0171594863744\\
1792	107.074061141177\\
1791	95.2530408993429\\
1790	103.406867630295\\
1789	91.1141781346511\\
1788	62.7586326224567\\
1787	86.3802029671272\\
1786	72.7681397986132\\
1785	78.3602320678245\\
1784	79.0008886135712\\
1783	94.7479538271856\\
1782	86.5060609276349\\
1781	127.467986507382\\
1780	148.369546418819\\
1779	144.360618349228\\
1778	182.949397544355\\
1777	105.776305844931\\
1776	165.965160342747\\
1775	156.983326158594\\
1774	195.429903352576\\
1773	215.058306250894\\
1772	252.327515365769\\
1771	201.80862534066\\
1770	274.84838191214\\
1769	283.451231477227\\
1768	275.359970766363\\
1767	291.691437014669\\
1766	308.445890746434\\
1765	293.703449058341\\
1764	310.355692181409\\
1763	313.631638772908\\
1762	298.797839192264\\
1761	262.308473906005\\
1760	305.102328325242\\
1759	299.812208953632\\
1758	279.661872603939\\
1757	253.411188325592\\
1756	278.070795039886\\
1755	265.849776878459\\
1754	254.732651407847\\
1753	233.013095947138\\
1752	264.587207313787\\
1751	230.690553388993\\
1750	219.867653866833\\
1749	213.538547676772\\
1748	175.819249569949\\
1747	147.073837384395\\
1746	170.522316236353\\
1745	211.595354404286\\
1744	198.692562286697\\
1743	151.488573732978\\
1742	184.298191219221\\
1741	144.578304391572\\
1740	175.992234333294\\
1739	145.20705816909\\
1738	167.173550221494\\
1737	166.576653169716\\
1736	207.562592983731\\
1735	165.537203220964\\
1734	156.176585933512\\
1733	185.935473226301\\
1732	197.320217024327\\
1731	181.88017781817\\
1730	243.482454645873\\
1729	217.367177177661\\
1728	224.342915713903\\
1727	188.491449167257\\
1726	206.074207822123\\
1725	215.253326756733\\
1724	163.748152285845\\
1723	202.873381869295\\
1722	239.485864369486\\
1721	176.14017524482\\
1720	202.607982419928\\
1719	212.978541691893\\
1718	223.807078406332\\
1717	303.122441702773\\
1716	312.319399707991\\
1715	329.747130515721\\
1714	322.964888087851\\
1713	326.237337465731\\
1712	270.68725364418\\
1711	285.146455378505\\
1710	241.838266182044\\
1709	227.348405682957\\
1708	200.725242176792\\
1707	178.958056617716\\
1706	131.319697421431\\
1705	121.501214355065\\
1704	136.064302896045\\
1703	113.187614715293\\
1702	108.219837913545\\
1701	121.363744511678\\
1700	90.7751400445524\\
1699	114.253073765159\\
1698	120.00383909833\\
1697	103.573048849335\\
1696	109.029565470112\\
1695	80.5879850855541\\
1694	100.716078287278\\
1693	85.7921374298771\\
1692	77.4116901904265\\
1691	79.863450499158\\
1690	45.6400662129264\\
1689	76.2949710345752\\
1688	79.4122167055276\\
1687	84.5396631691347\\
1686	103.283032648444\\
1685	143.962678324198\\
1684	98.9196303932682\\
1683	129.602681439744\\
1682	101.018943856761\\
1681	131.423357000635\\
1680	192.012238342607\\
1679	132.900570699668\\
1678	149.695269551345\\
1677	248.038623878321\\
1676	181.293559814353\\
1675	260.258218530131\\
1674	236.352024818217\\
1673	273.576866382968\\
1672	285.918572471009\\
1671	270.840954747532\\
1670	316.516336028143\\
1669	334.734384601667\\
1668	318.272174212687\\
1667	308.602478679463\\
1666	309.125245325127\\
1665	326.392315431796\\
1664	352.425655172662\\
1663	288.981551061568\\
1662	328.936799400961\\
1661	298.214311023339\\
1660	303.303035329162\\
1659	284.199029978577\\
1658	296.019782750078\\
1657	292.509747466639\\
1656	287.23462754871\\
1655	288.793917681711\\
1654	268.231728394067\\
1653	197.513014084404\\
1652	237.226624094791\\
1651	176.414967255208\\
1650	263.09106998028\\
1649	257.862946042504\\
1648	255.46381170905\\
1647	211.492719886065\\
1646	251.513834962737\\
1645	254.238297067087\\
1644	251.01852740913\\
1643	239.349610565274\\
1642	283.279311522384\\
1641	203.328473420912\\
1640	165.725820261086\\
1639	157.004079534785\\
1638	258.915388767941\\
1637	241.210693503282\\
1636	190.837435063469\\
1635	197.081477774936\\
1634	155.892028924934\\
1633	232.129991691392\\
1632	216.800275808874\\
1631	220.513229682239\\
1630	220.453191913964\\
1629	162.582451853057\\
1628	198.088773972974\\
1627	235.564043105025\\
1626	238.861486127325\\
1625	206.70387305024\\
1624	277.485356326192\\
1623	260.587088730487\\
1622	265.729136447418\\
1621	312.543975219765\\
1620	308.706299650162\\
1619	344.385646949868\\
1618	311.020720462485\\
1617	328.92837309418\\
1616	315.466823148391\\
1615	293.175583186598\\
1614	255.246524793252\\
1613	215.846704834431\\
1612	170.962329206278\\
1611	180.031402296753\\
1610	158.394818306831\\
1609	129.195622596362\\
1608	130.335254246868\\
1607	109.884990042707\\
1606	117.712783105808\\
1605	113.216295034349\\
1604	148.077664728529\\
1603	84.2643908931393\\
1602	94.187586639597\\
1601	131.236437412397\\
1600	131.291420519612\\
1599	92.898198876964\\
1598	77.1987408226258\\
1597	79.9751119148129\\
1596	90.9776014613367\\
1595	65.8065366881011\\
1594	60.9891597920626\\
1593	91.2751990605946\\
1592	101.731945407558\\
1591	104.184428842906\\
1590	111.521761469873\\
1589	138.052886091113\\
1588	97.2435218976665\\
1587	137.148481562363\\
1586	107.834018128522\\
1585	146.155086244854\\
1584	145.771304476608\\
1583	124.206289564184\\
1582	197.80153848107\\
1581	173.623907404498\\
1580	226.34423410459\\
1579	233.024516195998\\
1578	225.205230879401\\
1577	286.367330298309\\
1576	259.289656281805\\
1575	300.800294228591\\
1574	329.546335466439\\
1573	318.851954547803\\
1572	340.299953481727\\
1571	349.333188119668\\
1570	358.998684514955\\
1569	339.327671084754\\
1568	247.806360305033\\
1567	343.342245197438\\
1566	277.104880001675\\
1565	318.09257171417\\
1564	247.586630624974\\
1563	259.155778153236\\
1562	227.87226363126\\
1561	297.34893642016\\
1560	278.173773431863\\
1559	191.443332864744\\
1558	284.674713090929\\
1557	230.173532761455\\
1556	220.42467082824\\
1555	222.039211633136\\
1554	184.32829915822\\
1553	171.246750880406\\
1552	265.123040295918\\
1551	180.006153486932\\
1550	193.026313052852\\
1549	225.183610957208\\
1548	213.007385933919\\
1547	247.412454073847\\
1546	244.58503587834\\
1545	204.127656367997\\
1544	180.934659479651\\
1543	259.951818142295\\
1542	256.445169281002\\
1541	206.360659511198\\
1540	256.771813956009\\
1539	220.558295800617\\
1538	213.521235401195\\
1537	203.131922950757\\
1536	221.899995398291\\
1535	221.17417017817\\
1534	146.167900920847\\
1533	216.690560999345\\
1532	210.4243411389\\
1531	148.943341486472\\
1530	218.286043836703\\
1529	213.848790225712\\
1528	239.092357881334\\
1527	240.325314564321\\
1526	273.277303561095\\
1525	267.416548754382\\
1524	272.368157629597\\
1523	274.513269691891\\
1522	318.508809695494\\
1521	303.226843879382\\
1520	272.001466512591\\
1519	263.132142875515\\
1518	248.757783273811\\
1517	227.814400456084\\
1516	189.730667031438\\
1515	142.468580585668\\
1514	165.251118046817\\
1513	110.9624018345\\
1512	116.644735845174\\
1511	106.513677760688\\
1510	104.419997955702\\
1509	87.4893730794421\\
1508	138.107333793492\\
1507	110.774188487308\\
1506	114.185157535667\\
1505	124.256470974531\\
1504	60.9195347358803\\
1503	87.2195718592246\\
1502	76.2074384397125\\
1501	104.798492044858\\
1500	65.4381345435401\\
1499	75.210850863437\\
1498	72.6545350880086\\
1497	88.2541534531713\\
1496	91.7487187013984\\
1495	109.861314251235\\
1494	130.5369306451\\
1493	105.027461339442\\
1492	126.862673804025\\
1491	95.7279500361466\\
1490	114.839932620177\\
1489	126.382002098867\\
1488	103.066687927743\\
1487	90.4591970591146\\
1486	134.833817971475\\
1485	139.620965533493\\
1484	193.858058995744\\
1483	222.098306773293\\
1482	231.496468525542\\
1481	246.548415132109\\
1480	275.236272038071\\
1479	270.997689886232\\
1478	268.054335601204\\
1477	279.033677540118\\
1476	282.740044047096\\
1475	335.742002782537\\
1474	338.085834059377\\
1473	333.009180966333\\
1472	344.193471026642\\
1471	250.319511894687\\
1470	329.613144728614\\
1469	310.576387170271\\
1468	267.106180482909\\
1467	284.219921957868\\
1466	263.443136002217\\
1465	279.886552014388\\
1464	280.996053736534\\
1463	283.176361735338\\
1462	258.322457774514\\
1461	177.6929913293\\
1460	256.815295624617\\
1459	254.424002110558\\
1458	261.630869433635\\
1457	243.01304093214\\
1456	175.19725002372\\
1455	252.263243595092\\
1454	163.721505679603\\
1453	160.669558408084\\
1452	227.902199706252\\
1451	202.412052760604\\
1450	238.578068445709\\
1449	191.601582593906\\
1448	254.259509689038\\
1447	215.202353990123\\
1446	166.404735780796\\
1445	238.052923489855\\
1444	196.483253171081\\
1443	222.48384018583\\
1442	233.712175095131\\
1441	195.84331859461\\
1440	193.401483907025\\
1439	210.442633831489\\
1438	172.271226438824\\
1437	148.625322070202\\
1436	132.222384266826\\
1435	181.353784426749\\
1434	148.242535489787\\
1433	185.963888385366\\
1432	218.165558434112\\
1431	194.565101989437\\
1430	245.286961343453\\
1429	240.232464468205\\
1428	281.717297956905\\
1427	260.8761144237\\
1426	305.427227951267\\
1425	298.56385308159\\
1424	278.748749100082\\
1423	270.786962887242\\
1422	241.371169260055\\
1421	204.57866144277\\
1420	149.302607727416\\
1419	146.486397513855\\
1418	118.512521376284\\
1417	107.676602021044\\
1416	80.2209768895994\\
1415	71.847315301601\\
1414	64.989719945138\\
1413	76.1378743158469\\
1412	68.6899168773215\\
1411	80.286280639141\\
1410	87.9560510957244\\
1409	82.8742345288105\\
1408	84.3222658776239\\
1407	77.7739754338292\\
1406	81.6139895507357\\
1405	70.114682258829\\
1404	66.1731829490125\\
1403	53.783279367801\\
1402	69.5121956242091\\
1401	63.5247089389091\\
1400	76.6958358742233\\
1399	73.9849735242473\\
1398	86.2919424936586\\
1397	93.4430900692592\\
1396	83.9156538110803\\
1395	79.7831902416226\\
1394	90.8394302990191\\
1393	106.841189442721\\
1392	89.6464174623907\\
1391	131.991081499938\\
1390	124.286646734102\\
1389	145.631892881281\\
1388	179.936260013558\\
1387	194.178532388435\\
1386	214.729002429142\\
1385	226.269590445682\\
1384	244.286345584576\\
1383	267.531526985522\\
1382	276.580099946924\\
1381	287.525871513463\\
1380	296.05793319674\\
1379	306.980680940013\\
1378	312.931279281299\\
1377	312.694284107516\\
1376	318.337999207147\\
1375	304.658867549048\\
1374	297.894734255645\\
1373	288.633899005589\\
1372	260.869337381832\\
1371	248.50109542833\\
1370	241.356360455981\\
1369	219.150025718615\\
1368	200.439627712207\\
1367	185.188849165458\\
1366	158.100049094749\\
1365	244.863193064007\\
1364	221.511630551223\\
1363	209.015499331135\\
1362	130.476617532742\\
1361	159.666700129897\\
1360	221.659946751136\\
1359	123.257378372117\\
1358	185.414212269177\\
1357	144.183084654453\\
1356	121.261846094221\\
1355	151.274890561119\\
1354	218.73707767445\\
1353	196.207933681445\\
1352	188.172439789504\\
1351	206.647161650145\\
1350	210.004279644741\\
1349	167.388941976907\\
1348	184.374006455946\\
1347	163.802813907897\\
1346	214.993390267318\\
1345	187.224172263454\\
1344	167.914497944188\\
1343	186.325125836053\\
1342	202.814205552205\\
1341	175.234181556508\\
1340	195.156818163656\\
1339	162.23604311074\\
1338	135.503493902177\\
1337	209.576731207245\\
1336	229.434323774944\\
1335	179.893713478769\\
1334	243.919620089108\\
1333	229.14728608713\\
1332	270.569161637808\\
1331	298.164134767252\\
1330	298.821098753497\\
1329	280.229683960737\\
1328	268.158826791595\\
1327	232.982884788478\\
1326	220.723532828983\\
1325	175.904119251888\\
1324	146.569236573058\\
1323	126.560481444625\\
1322	79.6289261373709\\
1321	81.7198001316476\\
1320	69.8699927196645\\
1319	48.3161820157563\\
1318	42.2351989939632\\
1317	40.8591052297084\\
1316	38.0198216503103\\
1315	31.7687858217968\\
1314	41.8498438092761\\
1313	48.1759349773908\\
1312	48.6582924173348\\
1311	46.2835051255745\\
1310	46.3398539316331\\
1309	44.7125969418873\\
1308	48.0473683585488\\
1307	50.0439309275341\\
1306	48.0335544774351\\
1305	50.8693111003289\\
1304	56.8397692758852\\
1303	58.1704067123158\\
1302	49.8761191454959\\
1301	82.3039751415499\\
1300	66.5747382562754\\
1299	68.3419558430063\\
1298	72.7400651114276\\
1297	66.6285236888195\\
1296	79.3409222128222\\
1295	93.2873014871484\\
1294	118.716372806817\\
1293	115.858430081958\\
1292	143.332865364786\\
1291	157.338611312454\\
1290	197.102916879759\\
1289	215.078422628457\\
1288	242.911191958561\\
1287	225.744016157094\\
1286	271.794475886159\\
1285	283.445424171793\\
1284	279.099932857786\\
1283	291.753394075156\\
1282	285.842047090067\\
1281	289.875814998391\\
1280	266.253884609638\\
1279	252.87869520869\\
1278	278.021104919277\\
1277	275.241851774761\\
1276	264.359257864522\\
1275	242.658849640445\\
1274	234.971229334454\\
1273	226.821102926637\\
1272	203.510988431792\\
1271	193.6887224587\\
1270	201.713900304629\\
1269	143.988891621959\\
1268	139.792140485419\\
1267	195.483320668597\\
1266	184.455768049492\\
1265	177.871653530686\\
1264	131.078847716788\\
1263	139.86589487297\\
1262	96.5006737118956\\
1261	164.570377664487\\
1260	174.546925593914\\
1259	175.023319675136\\
1258	190.563111098218\\
1257	185.469655185133\\
1256	139.195476210038\\
1255	181.272178706787\\
1254	135.228413682635\\
1253	145.116242890499\\
1252	198.060834355225\\
1251	191.061690697996\\
1250	175.586873831988\\
1249	183.88369557602\\
1248	171.526921110726\\
1247	170.835298872354\\
1246	168.361816092345\\
1245	127.142909387201\\
1244	137.099909999847\\
1243	160.150079593839\\
1242	162.292809550763\\
1241	152.916672217412\\
1240	173.908733343475\\
1239	206.755145856564\\
1238	211.139211912826\\
1237	233.470355248616\\
1236	216.817941463014\\
1235	279.876861348557\\
1234	284.913172763608\\
1233	270.482483373742\\
1232	266.397043583561\\
1231	252.356384651637\\
1230	221.63424624577\\
1229	177.760428436308\\
1228	146.485534134858\\
1227	110.666823188246\\
1226	113.587321607598\\
1225	89.7472561136438\\
1224	71.7142943017206\\
1223	61.4965183918835\\
1222	62.4414836534093\\
1221	61.7154938663643\\
1220	67.1599910664256\\
1219	60.5227366656189\\
1218	77.9915384062132\\
1217	84.3571419448339\\
1216	79.2850086983729\\
1215	55.1899956668008\\
1214	72.5880224967586\\
1213	65.7905573476114\\
1212	65.6179149351068\\
1211	63.3617269578137\\
1210	56.619833243284\\
1209	63.2706980843679\\
1208	70.0351654268598\\
1207	77.9579968435387\\
1206	78.7413460198784\\
1205	82.4842389976297\\
1204	87.680039102365\\
1203	79.7648717940799\\
1202	77.4186857730555\\
1201	74.0717037136689\\
1200	92.7259460781764\\
1199	108.713500588857\\
1198	120.885591863427\\
1197	135.89071742411\\
1196	149.942134494258\\
1195	166.477532033138\\
1194	197.869367807876\\
1193	220.403095120692\\
1192	238.700461854933\\
1191	257.24355347956\\
1190	244.035493678707\\
1189	267.884515832785\\
1188	279.971432339343\\
1187	265.026760202162\\
1186	275.941913525213\\
1185	267.278396072592\\
1184	269.029234299711\\
1183	255.853764031173\\
1182	241.313470281097\\
1181	224.716592961387\\
1180	235.448082983267\\
1179	232.234334512382\\
1178	207.722786022375\\
1177	193.767700180253\\
1176	180.661009708333\\
1175	184.817881159676\\
1174	174.837761029226\\
1173	169.796754569976\\
1172	161.544399361285\\
1171	124.400133284683\\
1170	111.836396279664\\
1169	141.575054457451\\
1168	127.060883018626\\
1167	119.983710632717\\
1166	110.123277999492\\
1165	90.6198581790997\\
1164	108.893312271566\\
1163	111.410809627316\\
1162	118.720787028508\\
1161	149.084279300965\\
1160	137.641430045996\\
1159	140.559317292635\\
1158	156.151134115607\\
1157	142.926697332499\\
1156	139.539002549293\\
1155	141.722151525402\\
1154	145.163024510975\\
1153	116.472908511437\\
1152	131.672414910653\\
1151	127.888974794141\\
1150	116.311173790166\\
1149	117.667646456577\\
1148	106.980305461147\\
1147	102.925785328894\\
1146	121.441661042099\\
1145	120.648529883024\\
1144	141.869346560703\\
1143	145.590111479021\\
1142	165.726997412118\\
1141	174.785018709878\\
1140	188.17069994974\\
1139	217.194204331248\\
1138	229.518984576627\\
1137	231.463439165911\\
1136	238.369672000138\\
1135	205.837748518088\\
1134	175.807276672987\\
1133	160.102847580547\\
1132	153.767631373657\\
1131	139.204935176004\\
1130	92.8426123128833\\
1129	81.4232607969501\\
1128	86.9258674076868\\
1127	69.6261959401269\\
1126	67.1553839403202\\
1125	61.7828406154005\\
1124	60.8253857414093\\
1123	53.9523115415513\\
1122	76.8918705547171\\
1121	83.713828336338\\
1120	75.2072292460922\\
1119	79.411095703367\\
1118	79.9939663016457\\
1117	58.829496475915\\
1116	64.5101912418061\\
1115	71.751501867581\\
1114	67.2640533023258\\
1113	59.6545843560059\\
1112	70.3634853625324\\
1111	75.5996920770866\\
1110	81.199675558465\\
1109	97.3279466669172\\
1108	102.971261232092\\
1107	90.7406246785887\\
1106	90.5428379661781\\
1105	97.9484072728417\\
1104	116.000120821618\\
1103	120.183263228074\\
1102	138.138844156389\\
1101	182.157787210386\\
1100	172.601139308761\\
1099	191.688413856615\\
1098	233.313672291608\\
1097	242.341508044374\\
1096	245.993999779622\\
1095	256.442295956934\\
1094	277.227050923522\\
1093	286.80438048228\\
1092	297.352633919095\\
1091	300.419241450618\\
1090	292.046324088646\\
1089	300.392489456783\\
1088	303.614014657899\\
1087	270.011837183781\\
1086	294.32372965638\\
1085	294.767665672569\\
1084	247.25143909681\\
1083	263.683406382641\\
1082	245.340401937042\\
1081	241.908984842444\\
1080	244.226905898672\\
1079	232.361312534065\\
1078	245.251221609632\\
1077	225.468765647408\\
1076	160.648483620645\\
1075	201.730291739702\\
1074	208.682908185876\\
1073	136.789067546448\\
1072	120.03591097444\\
1071	167.358950928886\\
1070	151.758503470122\\
1069	187.455783556929\\
1068	168.953362385757\\
1067	191.630743122113\\
1066	130.094137229363\\
1065	150.830309526382\\
1064	226.833614336272\\
1063	212.016844043498\\
1062	145.719733504295\\
1061	220.217144987453\\
1060	227.834209545033\\
1059	224.167542906366\\
1058	194.053979772188\\
1057	167.359144312971\\
1056	173.388926326154\\
1055	197.992766920128\\
1054	175.685052618919\\
1053	167.009715189679\\
1052	197.948194375764\\
1051	204.799720717312\\
1050	217.822335758775\\
1049	178.152328887937\\
1048	243.075993270142\\
1047	231.426872407727\\
1046	281.619425236306\\
1045	291.34124923519\\
1044	284.251911777464\\
1043	300.360188905289\\
1042	324.859918541378\\
1041	315.211249542326\\
1040	276.301422441161\\
1039	285.572510857877\\
1038	229.40983721011\\
1037	227.188606032695\\
1036	199.364560284544\\
1035	144.202917234918\\
1034	143.616693377011\\
1033	113.448090665849\\
1032	96.5456650531983\\
1031	100.223510115878\\
1030	92.1921495458079\\
1029	101.642608299537\\
1028	103.576858582092\\
1027	80.7495203521163\\
1026	122.940030732446\\
1025	77.0612233508022\\
1024	77.377417435504\\
1023	116.650015100904\\
1022	71.89535555551\\
1021	62.7201996898424\\
1020	77.6150430850795\\
1019	86.0515034544474\\
1018	75.3897630770514\\
1017	87.1348383914477\\
1016	84.6417623078383\\
1015	105.687635948406\\
1014	115.065340876138\\
1013	139.26472996769\\
1012	105.673950758278\\
1011	154.74529893781\\
1010	123.658688516401\\
1009	97.8486650405071\\
1008	187.6148523508\\
1007	164.84119521657\\
1006	216.245743293743\\
1005	164.180977593049\\
1004	185.654366542316\\
1003	182.576216379983\\
1002	263.428250714896\\
1001	259.240174272246\\
1000	273.544732977728\\
999	313.043544851683\\
998	312.174065792938\\
997	305.36090259223\\
996	286.911294995359\\
995	290.856178531738\\
994	298.52883941825\\
993	296.568231818739\\
992	260.855875672196\\
991	331.262567313184\\
990	284.94664515002\\
989	329.223869507143\\
988	301.158072236124\\
987	289.833146384056\\
986	295.214516250344\\
985	269.524901864797\\
984	256.994297946772\\
983	270.336695906828\\
982	268.005168536629\\
981	259.874592550526\\
980	257.900275994638\\
979	172.765633922887\\
978	200.975512122235\\
977	266.859522626052\\
976	245.373979946047\\
975	136.659182959282\\
974	246.599052855996\\
973	170.88428519333\\
972	194.380206109317\\
971	263.397043863691\\
970	192.962339235497\\
969	144.585766129477\\
968	197.531413273286\\
967	257.832829044506\\
966	190.768035707394\\
965	249.116633467789\\
964	237.865466555111\\
963	206.532665513441\\
962	239.349012360045\\
961	236.75351015072\\
960	197.432659967439\\
959	212.20342144954\\
958	198.741568934545\\
957	197.845708245668\\
956	216.693700272766\\
955	224.143514899067\\
954	190.002031661598\\
953	213.229176359816\\
952	175.196912576606\\
951	258.573900243558\\
950	292.715301554075\\
949	273.317431678406\\
948	291.460711159988\\
947	277.412651785412\\
946	293.839170187221\\
945	299.747515555154\\
944	317.091120585711\\
943	299.546567296558\\
942	267.119741062342\\
941	242.030223119854\\
940	214.233861273528\\
939	179.719850384867\\
938	175.547839648919\\
937	134.269456660679\\
936	127.856473257093\\
935	126.6512024514\\
934	94.0649661644331\\
933	124.448070249861\\
932	143.203887931506\\
931	155.558286945217\\
930	93.9923399212558\\
929	146.016229380087\\
928	144.276545679136\\
927	89.577393856498\\
926	118.02980847319\\
925	100.553721685113\\
924	75.4939997494843\\
923	94.7181869104771\\
922	63.0131260167112\\
921	81.7528864806642\\
920	90.2803227114798\\
919	101.289987274974\\
918	116.317160888196\\
917	133.943102806766\\
916	133.086005479863\\
915	158.181490201602\\
914	152.354305942897\\
913	188.832948418317\\
912	207.788039377596\\
911	142.507819730541\\
910	182.999549256202\\
909	240.776009316878\\
908	178.734185907235\\
907	210.370582490859\\
906	277.496555170356\\
905	288.115442218693\\
904	301.300213496564\\
903	292.284475672628\\
902	308.24190116381\\
901	320.169916911112\\
900	289.890940471961\\
899	291.2084273069\\
898	290.161925869717\\
897	292.81993083714\\
896	267.834015031521\\
895	344.239207306365\\
894	236.26459085695\\
893	234.391396290469\\
892	322.077466109047\\
891	272.564319845796\\
890	287.16812056886\\
889	285.341793301114\\
888	269.848546902053\\
887	194.223022937219\\
886	271.001302819438\\
885	257.096053749737\\
884	213.46468679243\\
883	263.724266960039\\
882	293.348007797916\\
881	243.418365668236\\
880	170.8191126623\\
879	157.588309274753\\
878	272.361965162911\\
877	262.956298226779\\
876	225.443234049764\\
875	247.27918917052\\
874	173.441103930338\\
873	271.239803120141\\
872	286.874789187667\\
871	288.627720968259\\
870	280.464806819801\\
869	246.427446800292\\
868	258.537234074309\\
867	208.944965714521\\
866	196.395934082624\\
865	161.085169605009\\
864	238.268776530068\\
863	210.080079360468\\
862	159.557201705624\\
861	222.554691071904\\
860	222.945532792887\\
859	199.167429077029\\
858	164.388004633704\\
857	233.626515496864\\
856	271.058370045058\\
855	239.943651587558\\
854	265.322450043409\\
853	327.122355037342\\
852	342.24596765136\\
851	338.042228873494\\
850	270.298655859084\\
849	344.339261594928\\
848	314.473197549385\\
847	273.5161410919\\
846	261.542141479695\\
845	248.825561886883\\
844	206.87596422755\\
843	171.194178018638\\
842	153.337650998987\\
841	154.387137828095\\
840	137.987325870634\\
839	144.052283640032\\
838	123.915425877845\\
837	123.475900441398\\
836	175.065850803284\\
835	141.366343481787\\
834	91.5768505547496\\
833	179.579553532251\\
832	77.3465925881332\\
831	129.060233779652\\
830	95.8303147166887\\
829	81.6654467521865\\
828	53.3022306925413\\
827	104.838120315111\\
826	96.9641676485701\\
825	87.6792560595321\\
824	105.230515598294\\
823	123.131554854622\\
822	133.775638039624\\
821	157.112430777402\\
820	180.605598910543\\
819	138.335100089366\\
818	132.642471124329\\
817	183.166814471865\\
816	196.270625514722\\
815	208.690422540918\\
814	264.639645112547\\
813	179.887916366197\\
812	274.877384177882\\
811	199.188899573772\\
810	225.107626491582\\
809	311.34652960816\\
808	244.621823734632\\
807	315.223942469188\\
806	282.230734960677\\
805	313.823511798302\\
804	320.24386372547\\
803	355.408337082352\\
802	267.168482289845\\
801	295.908721769784\\
800	355.812161749466\\
799	255.829159145915\\
798	299.466267035712\\
797	272.037832506257\\
796	287.745587615301\\
795	236.398400284815\\
794	293.985685381644\\
793	276.291718827314\\
792	285.774067185385\\
791	240.797285132771\\
790	271.830018077169\\
789	184.232674658304\\
788	189.933988341002\\
787	190.85141944189\\
786	254.614088457596\\
785	169.522907528554\\
784	217.544341854555\\
783	236.800241793667\\
782	253.862613379785\\
781	173.90389410557\\
780	264.444913311696\\
779	201.562329834151\\
778	266.052507782771\\
777	233.727716827038\\
776	173.003019406922\\
775	211.976842794738\\
774	238.865552201592\\
773	234.057618342677\\
772	219.446914545229\\
771	156.806298383511\\
770	166.563287722881\\
769	205.592421092338\\
768	205.142052745317\\
767	179.121828750272\\
766	157.831272519088\\
765	179.594637009998\\
764	155.851196753965\\
763	167.90003002575\\
762	168.511832889756\\
761	164.62433619453\\
760	202.14379010864\\
759	245.943934313271\\
758	258.476236746441\\
757	254.117580479097\\
756	285.515338137402\\
755	285.335591174317\\
754	294.39209127438\\
753	304.895109120278\\
752	282.549690920636\\
751	259.737252118295\\
750	253.729544299878\\
749	230.435303237824\\
748	189.054214883631\\
747	139.545277573111\\
746	151.350325763115\\
745	116.727831773344\\
744	103.290372099531\\
743	107.634682080401\\
742	100.110340445215\\
741	100.603814789962\\
740	84.9108348990217\\
739	124.393064497126\\
738	113.391206313237\\
737	129.689786959055\\
736	112.001862607888\\
735	105.439782922316\\
734	100.496660398024\\
733	88.1598885736163\\
732	78.0729632331831\\
731	80.8787215314118\\
730	73.6682363223763\\
729	62.3556747782751\\
728	82.3564529213407\\
727	79.0943113761061\\
726	90.3384151573352\\
725	102.045538687944\\
724	109.532466384394\\
723	91.7601910903634\\
722	101.991151454628\\
721	97.0204919441198\\
720	120.590054530415\\
719	111.123038592349\\
718	126.37599834907\\
717	147.769815341829\\
716	164.237438136718\\
715	179.487442332982\\
714	210.845476910287\\
713	221.07123111453\\
712	251.861778404468\\
711	268.229211679323\\
710	269.14312768638\\
709	271.231194351179\\
708	281.84108232496\\
707	289.086917858298\\
706	279.765209092962\\
705	285.594881999822\\
704	255.372977522389\\
703	275.668819886746\\
702	252.441814810516\\
701	261.751485401724\\
700	249.624250741406\\
699	249.693296179031\\
698	230.619881492904\\
697	239.342431927963\\
696	219.379359460179\\
695	180.650243185357\\
694	187.208775125221\\
693	182.68568478779\\
692	197.47764445432\\
691	129.4605719126\\
690	122.980269105113\\
689	160.655508578498\\
688	169.372697030198\\
687	144.605584680541\\
686	163.611910941718\\
685	143.993654870398\\
684	150.656761928213\\
683	149.841751644759\\
682	133.680611572909\\
681	137.101439205492\\
680	138.933331514341\\
679	139.186256964994\\
678	163.965505039688\\
677	157.036843694065\\
676	164.827679612324\\
675	133.476733029589\\
674	158.315694656624\\
673	147.469299139084\\
672	151.906742511597\\
671	139.411284133541\\
670	129.409169810698\\
669	131.599088333675\\
668	152.778899712473\\
667	146.620578096462\\
666	147.896717575308\\
665	176.176778736415\\
664	181.499408969584\\
663	188.721669230998\\
662	182.588113296533\\
661	222.49063963784\\
660	244.606569988121\\
659	269.848112330164\\
658	276.714578057489\\
657	269.010003290704\\
656	281.225766433841\\
655	255.204525868276\\
654	217.881790905594\\
653	200.725440587254\\
652	180.793617633169\\
651	139.425817224543\\
650	134.990914767519\\
649	99.1818748654588\\
648	96.7769980186901\\
647	83.8640142718347\\
646	87.2819364541751\\
645	77.4002307735227\\
644	101.926644310842\\
643	105.057206755527\\
642	65.1331254131407\\
641	107.649193974579\\
640	91.0360119417025\\
639	97.5597477034557\\
638	81.1935545077563\\
637	79.3871242762508\\
636	74.7187652914376\\
635	79.5410840753726\\
634	65.82156113395\\
633	79.8214325732579\\
632	65.8946511302817\\
631	79.6182608455406\\
630	75.5678819466609\\
629	101.302797346815\\
628	99.3864294448957\\
627	87.0820354244889\\
626	101.371377147183\\
625	95.3442874179583\\
624	106.212071082562\\
623	157.239133186996\\
622	155.882470284595\\
621	163.210055917467\\
620	166.133269069087\\
619	191.942144286474\\
618	231.802407288287\\
617	236.939494479465\\
616	265.765467888006\\
615	280.588672405271\\
614	267.33049792391\\
613	286.955767140275\\
612	290.586277841243\\
611	296.841900251805\\
610	285.830123748577\\
609	296.530467930538\\
608	296.146031710575\\
607	288.586334464898\\
606	278.401834718833\\
605	261.326340608317\\
604	265.479814064863\\
603	250.296503675813\\
602	218.062379872223\\
601	234.15938591608\\
600	208.193315522033\\
599	221.158128254034\\
598	173.304982757463\\
597	218.502879395483\\
596	165.415729152556\\
595	166.692814608416\\
594	192.532724796938\\
593	187.084186268156\\
592	205.024192879272\\
591	191.95743354099\\
590	157.735521991094\\
589	183.059047154348\\
588	164.001318598131\\
587	190.783180042267\\
586	171.402851590831\\
585	172.523645968608\\
584	185.557768446731\\
583	149.845805568036\\
582	172.313928899896\\
581	212.639889689751\\
580	174.504457155418\\
579	209.667247268648\\
578	192.712309732804\\
577	172.546147347594\\
576	190.992687801845\\
575	154.090262214745\\
574	132.833388764006\\
573	154.334848556353\\
572	163.72047919464\\
571	159.913439850148\\
570	195.364409640652\\
569	234.375172832439\\
568	249.475261792043\\
567	261.123649444123\\
566	296.400668012767\\
565	268.099922030287\\
564	329.646772375092\\
563	286.18893526655\\
562	334.323516008598\\
561	320.165983610319\\
560	320.849555917466\\
559	282.472834238232\\
558	267.266326630205\\
557	232.644213958965\\
556	205.729529830962\\
555	159.688774505763\\
554	152.429780563585\\
553	147.253760318251\\
552	144.650017106591\\
551	115.487415574931\\
550	102.426878543818\\
549	132.554495060371\\
548	112.138663568831\\
547	164.457191865383\\
546	100.243458980946\\
545	91.9325757213494\\
544	115.472155315274\\
543	47.9995644815051\\
542	130.516161574979\\
541	87.9546745195662\\
540	100.532070894538\\
539	85.2903508152841\\
538	92.8595845704806\\
537	104.532341793955\\
536	121.817001203224\\
535	140.490292216568\\
534	145.458650591486\\
533	122.339451382466\\
532	160.022201593534\\
531	168.640442849748\\
530	117.9328004169\\
529	159.248935268549\\
528	200.120755043054\\
527	182.756178406278\\
526	137.072505497324\\
525	242.291362534688\\
524	198.13555313416\\
523	259.819933060063\\
522	265.446646416051\\
521	280.889954787579\\
520	239.129768993817\\
519	307.626955983079\\
518	287.919634947614\\
517	289.490300592976\\
516	337.937431819047\\
515	267.626963602789\\
514	349.092099023066\\
513	277.695652348334\\
512	285.044797214203\\
511	276.680091573594\\
510	232.746018319751\\
509	252.337971509262\\
508	286.451703909302\\
507	264.226317708818\\
506	290.468333459322\\
505	292.417205249438\\
504	234.843275970137\\
503	260.613936468235\\
502	280.770454930872\\
501	178.894185623375\\
500	198.691326589457\\
499	248.968208654465\\
498	207.058879016882\\
497	233.375192963074\\
496	193.469664901547\\
495	164.542240055547\\
494	188.102046681707\\
493	219.429166973827\\
492	218.343009684251\\
491	247.345856819997\\
490	233.229073506678\\
489	275.327737153149\\
488	278.020699042734\\
487	225.572397382712\\
486	163.93460754419\\
485	163.081859486933\\
484	213.427781931343\\
483	254.330086205825\\
482	217.616916128324\\
481	159.715384872176\\
480	166.223634821954\\
479	169.785995959072\\
478	224.921392371387\\
477	171.809911277136\\
476	224.895766518145\\
475	208.758169069391\\
474	232.349329524991\\
473	238.99080657248\\
472	184.447132370637\\
471	259.297208460697\\
470	289.298689354976\\
469	284.028775010711\\
468	326.825993741237\\
467	331.432881214127\\
466	338.540499649519\\
465	322.134986109791\\
464	302.16560897475\\
463	283.993054987281\\
462	249.690598119292\\
461	236.241076585799\\
460	195.603270488564\\
459	152.718776670045\\
458	150.158401416507\\
457	145.434788430763\\
456	122.42530512382\\
455	117.906681892239\\
454	109.128688149523\\
453	115.495613846437\\
452	98.8657738980616\\
451	117.689405937502\\
450	136.898572410988\\
449	114.163212782228\\
448	130.527101009257\\
447	123.454647848161\\
446	129.225326788791\\
445	59.1566671846342\\
444	91.3850302066743\\
443	100.361680436882\\
442	72.6920671042094\\
441	106.591339964668\\
440	99.6607122766107\\
439	120.271559403989\\
438	143.546652616369\\
437	150.105451239016\\
436	100.434127229336\\
435	141.94342227342\\
434	160.838706036123\\
433	115.618733868501\\
432	115.216608730934\\
431	130.208564657404\\
430	126.333454626813\\
429	167.63648107841\\
428	159.383032878066\\
427	208.01795790384\\
426	237.497804586249\\
425	205.744591267557\\
424	262.586979896018\\
423	255.682989835025\\
422	245.456746336467\\
421	316.878449889153\\
420	329.426661632462\\
419	343.576072932148\\
418	256.491678303949\\
417	282.025635779354\\
416	276.106137620475\\
415	340.10729190286\\
414	321.711410161383\\
413	324.127344222278\\
412	265.441490157913\\
411	295.52504847773\\
410	286.896777229026\\
409	284.268193401799\\
408	280.980513381483\\
407	239.680274791358\\
406	282.90064346885\\
405	219.762330414244\\
404	278.134007731568\\
403	237.624089262028\\
402	281.574741539374\\
401	216.972524139696\\
400	196.927707466853\\
399	252.015205418207\\
398	283.18809486168\\
397	178.930010361536\\
396	150.819159482839\\
395	242.018241857315\\
394	235.692037704986\\
393	259.485322133058\\
392	190.563321407108\\
391	159.175752073765\\
390	182.168282841591\\
389	203.900104195604\\
388	236.245606021204\\
387	248.238382832356\\
386	245.229155604178\\
385	209.867118867426\\
384	197.227794338998\\
383	226.621385407134\\
382	194.698149483526\\
381	190.928826239972\\
380	228.495577869218\\
379	176.684492607272\\
378	217.407991881603\\
377	238.204020375506\\
376	269.248593667197\\
375	271.070716058333\\
374	229.896173813783\\
373	300.645325630091\\
372	303.786204070741\\
371	330.223647882842\\
370	343.328810424692\\
369	325.470576253529\\
368	317.640845347547\\
367	285.661399281207\\
366	263.889532386724\\
365	243.317030874658\\
364	213.99134589086\\
363	187.822162072087\\
362	174.924624002649\\
361	131.619314962093\\
360	129.818850811782\\
359	121.738031880391\\
358	115.579055662078\\
357	138.842034941336\\
356	130.515243528844\\
355	111.137526468346\\
354	60.9856845691148\\
353	161.944029080399\\
352	148.454059698755\\
351	55.0328355976452\\
350	121.012286237369\\
349	103.012456518005\\
348	47.070104647033\\
347	99.5855240554234\\
346	85.4379077316086\\
345	111.780336470075\\
344	108.989662527508\\
343	132.522573079087\\
342	148.12714576887\\
341	155.399919661793\\
340	156.36638090881\\
339	126.529657717385\\
338	123.872804470352\\
337	188.490376122171\\
336	206.209397712282\\
335	132.397587159818\\
334	225.613670276584\\
333	216.187200968959\\
332	208.3665122914\\
331	182.15956121707\\
330	228.411474167395\\
329	294.883094749885\\
328	255.921743465875\\
327	300.192815966727\\
326	329.030001573729\\
325	314.775759587041\\
324	344.060873231467\\
323	306.641606871118\\
322	347.00197420627\\
321	287.174388673471\\
320	298.892724065031\\
319	295.322089163667\\
318	337.116900334361\\
317	274.196718833445\\
316	240.284137419157\\
315	266.515376974355\\
314	294.465771944963\\
313	277.832366889557\\
312	285.572633768547\\
311	211.67650701473\\
310	260.212327417694\\
309	176.681729265086\\
308	266.967912009193\\
307	159.015512124171\\
306	244.85540201436\\
305	197.846433006641\\
304	200.990853968242\\
303	147.709018042928\\
302	275.133683741686\\
301	271.935433309566\\
300	185.305969680012\\
299	210.225248856327\\
298	255.509793892924\\
297	194.177675751203\\
296	159.437428026039\\
295	236.628132183987\\
294	234.907888635121\\
293	248.32083335712\\
292	227.897601857066\\
291	235.686262428059\\
290	203.874118234981\\
289	222.378607137993\\
288	206.082884671657\\
287	146.354091293206\\
286	144.089174030556\\
285	191.785393320881\\
284	195.011352452082\\
283	161.934979530515\\
282	186.18757028655\\
281	232.054423262053\\
280	188.789421307129\\
279	246.118258088188\\
278	256.20999515471\\
277	303.076351098808\\
276	305.905892512751\\
275	260.412959354443\\
274	324.966714033461\\
273	313.820736955243\\
272	290.91990549835\\
271	277.776068088243\\
270	259.683799358806\\
269	203.527678375082\\
268	199.311614873579\\
267	134.824247306951\\
266	154.424395819858\\
265	113.167156471243\\
264	100.016072703497\\
263	112.178568011353\\
262	102.191888620387\\
261	110.516534660897\\
260	108.689252254181\\
259	107.993441970561\\
258	100.162527213516\\
257	92.7111089356224\\
256	82.0376669595888\\
255	96.2785154499885\\
254	104.549760483104\\
253	80.3207918147825\\
252	80.3121072635246\\
251	87.4787560542224\\
250	64.0645853586524\\
249	63.8187819893463\\
248	68.2388304226565\\
247	87.9543475725784\\
246	101.765015949904\\
245	138.406483412548\\
244	133.885698756847\\
243	119.103524570127\\
242	163.994849281041\\
241	127.965946457598\\
240	141.78950035969\\
239	152.105591134435\\
238	164.949689893365\\
237	198.045132573598\\
236	237.355555035941\\
235	194.213322138457\\
234	220.261897419835\\
233	256.442417141923\\
232	292.8203195049\\
231	302.565613069695\\
230	274.659655243525\\
229	322.020404195758\\
228	322.420943215489\\
227	334.752178179763\\
226	329.50389174484\\
225	336.281118712155\\
224	296.519694882576\\
223	325.962206185219\\
222	288.560382408879\\
221	273.834373125752\\
220	250.868789168426\\
219	272.982050883934\\
218	268.953537768859\\
217	207.587449307795\\
216	204.242390969218\\
215	198.609265392084\\
214	216.960081862605\\
213	235.904553018296\\
212	211.460063517393\\
211	201.409429195719\\
210	150.266945351199\\
209	191.962761038706\\
208	194.293374825275\\
207	134.320383186996\\
206	176.780522537668\\
205	125.315486016332\\
204	194.125771722383\\
203	191.828203083002\\
202	202.47001042897\\
201	198.744680187755\\
200	177.227608449383\\
199	195.415702533661\\
198	218.545190178781\\
197	167.927888050685\\
196	204.268843534654\\
195	164.392142830948\\
194	184.951316098021\\
193	204.149095863042\\
192	144.637736063583\\
191	163.140635094319\\
190	182.72541162563\\
189	153.947878049012\\
188	161.995280631033\\
187	166.765929566494\\
186	192.032052607451\\
185	202.921190363335\\
184	230.234994886504\\
183	219.409420547562\\
182	241.180307124658\\
181	282.688924925198\\
180	276.644207292514\\
179	294.889848115069\\
178	298.907995235636\\
177	285.423124138824\\
176	288.003541315278\\
175	274.875523337194\\
174	239.132052560781\\
173	221.026799402584\\
172	198.45025536898\\
171	134.327342691551\\
170	151.789767821802\\
169	98.7499162191139\\
168	102.719346261968\\
167	93.3123974107132\\
166	89.8962986998812\\
165	111.433919559673\\
164	76.238157163319\\
163	115.569999053502\\
162	74.5048973087328\\
161	115.321600130655\\
160	98.8463798978252\\
159	79.4818597395062\\
158	77.1403642409286\\
157	79.0760489166705\\
156	74.4054014548536\\
155	57.0738487677248\\
154	67.2024967107649\\
153	68.8381549606379\\
152	78.5274768959308\\
151	79.3264834911159\\
150	74.2412021894657\\
149	92.6487784144099\\
148	101.845280005721\\
147	83.0110725106062\\
146	99.7619908385863\\
145	74.7672785030108\\
144	94.5261161959394\\
143	102.486621458487\\
142	133.247628519895\\
141	137.915761358563\\
140	171.957460571104\\
139	183.05219776313\\
138	207.78627155008\\
137	235.436433999268\\
136	254.984336676011\\
135	260.857583528488\\
134	278.265305696608\\
133	283.326724783447\\
132	292.074879393929\\
131	291.334801920169\\
130	282.220065296753\\
129	299.203804387119\\
128	292.912180073204\\
127	282.949474132958\\
126	250.660019507238\\
125	266.792037328469\\
124	247.585635927475\\
123	249.440914008735\\
122	240.01886784913\\
121	217.635658531508\\
120	177.52188039929\\
119	203.669992959115\\
118	217.820486328668\\
117	194.719553023028\\
116	183.080858240808\\
115	146.868799003901\\
114	150.697042914266\\
113	171.324853913073\\
112	182.462822739795\\
111	135.937758237271\\
110	141.057950049739\\
109	180.941030807059\\
108	144.236730272145\\
107	108.933104015296\\
106	165.39897700076\\
105	112.289671615634\\
104	132.850753140997\\
103	113.035120272661\\
102	168.207502353672\\
101	144.2403342825\\
100	129.002456757892\\
99	133.069130345008\\
98	167.097548367746\\
97	185.167731400766\\
96	130.551925143024\\
95	153.153121515614\\
94	148.415355320394\\
93	130.073388970341\\
92	172.621051003648\\
91	164.776518838278\\
90	175.495640548983\\
89	155.341321898157\\
88	188.801767013842\\
87	215.821430043856\\
86	224.557167225843\\
85	199.059917563146\\
84	229.370066424732\\
83	234.382641451139\\
82	280.111761198046\\
81	292.256898481976\\
80	267.63517513008\\
79	262.112170284307\\
78	237.94646814874\\
77	203.868063369079\\
76	168.638046678619\\
75	136.698594687189\\
74	107.716369743887\\
73	105.762988607857\\
72	89.8724670271051\\
71	77.2633147052591\\
70	87.5092812544086\\
69	81.7883307050987\\
68	76.6698652160279\\
67	71.200149891054\\
66	103.326310789676\\
65	96.20251147567\\
64	75.0063876194667\\
63	77.4176606766836\\
62	89.277592554634\\
61	78.9837574973668\\
60	76.460074199585\\
59	78.3144435946171\\
58	67.4156918382221\\
57	78.1280206748882\\
56	73.2436698922464\\
55	85.6671982590902\\
54	91.8359052786056\\
53	107.472513935597\\
52	104.597779160644\\
51	102.189115859401\\
50	88.0947548218553\\
49	107.8336456922\\
48	131.838584935923\\
47	141.404963979267\\
46	128.648005249825\\
45	152.3361170962\\
44	201.917264569236\\
43	212.89665709833\\
42	232.915222627913\\
41	236.578432739495\\
40	258.797268116784\\
39	269.020342673677\\
38	279.288152784832\\
37	303.541189467848\\
36	316.058619442889\\
35	312.639456108653\\
34	331.252866238005\\
33	281.222903379467\\
32	340.439461276874\\
31	287.061577941545\\
30	318.944558684604\\
29	263.965864221834\\
28	269.843577378487\\
27	244.964574575343\\
26	207.947831642545\\
25	278.429040470764\\
24	265.294128990365\\
23	190.278400181436\\
22	164.632620394039\\
21	239.441878305077\\
20	238.985616888815\\
19	144.158821386287\\
18	160.923148110763\\
17	164.452963146015\\
16	212.983570045723\\
15	202.880168936744\\
14	162.810748119077\\
13	139.466539848861\\
12	158.23183311261\\
11	134.039600466038\\
10	113.306021052518\\
9	120.783589118369\\
8	217.439354914008\\
7	175.454560387726\\
6	206.703855149444\\
5	146.789181560574\\
4	179.037590408609\\
3	177.452031924014\\
2	215.788775738233\\
1	204.42146513895\\
0	191.710387620631\\
}--cycle;
\addlegendentry{$\mu\text{ }\pm\text{ 2}\sigma$}

\addplot [color=mycolor2, line width=1.0pt]
  table[row sep=crcr]{%
0	222.949930745009\\
1	236.088401171517\\
2	256.839162242675\\
3	217.252130492471\\
4	213.475177861556\\
5	189.636463703891\\
6	245.086375252774\\
7	217.896225785566\\
8	261.129449687538\\
9	173.279000315493\\
10	164.329527400377\\
11	178.224637111032\\
12	197.405000333737\\
13	175.586605102194\\
14	198.795984671758\\
15	241.070998560549\\
16	252.478259785355\\
17	200.359935114973\\
18	202.023417151627\\
19	187.499226905414\\
20	282.973318234671\\
21	278.635880609878\\
22	211.430429214108\\
23	235.252856419236\\
24	307.135626082182\\
25	319.283170487963\\
26	252.077840501988\\
27	282.298200373272\\
28	306.260314081916\\
29	298.675483489277\\
30	357.465922100574\\
31	325.476497089451\\
32	377.9984283317\\
33	321.196200253803\\
34	374.951374546984\\
35	364.322324430575\\
36	359.150958126876\\
37	355.810834872275\\
38	319.636950487183\\
39	308.52234727998\\
40	295.674329848849\\
41	273.513838621758\\
42	267.126393428323\\
43	252.823998548261\\
44	239.051508514562\\
45	191.246553653026\\
46	167.543688537358\\
47	179.156352156586\\
48	168.814776413269\\
49	141.892424121193\\
50	127.939233033074\\
51	138.001204349245\\
52	140.408379033255\\
53	151.870343950333\\
54	128.263324438162\\
55	118.045561855934\\
56	110.276776503038\\
57	113.729006566568\\
58	101.830373647284\\
59	113.625644839394\\
60	113.77973292277\\
61	113.623886628558\\
62	128.791797975295\\
63	110.515630007399\\
64	109.450298416488\\
65	129.923918842221\\
66	136.590989130995\\
67	108.186407734239\\
68	112.857824259787\\
69	118.681266583066\\
70	126.692338322489\\
71	113.739164602418\\
72	125.664849854965\\
73	142.731939134873\\
74	144.932208378506\\
75	177.040559241155\\
76	207.550860787426\\
77	240.085107272778\\
78	273.727658539643\\
79	296.076047109791\\
80	303.093105677264\\
81	328.052630032384\\
82	315.296876623021\\
83	286.08541955608\\
84	276.027146079144\\
85	244.330125183662\\
86	260.668328373851\\
87	249.126600508497\\
88	227.211498348329\\
89	191.652015267167\\
90	210.656810527637\\
91	197.796307364589\\
92	206.349465216247\\
93	169.129366896492\\
94	181.872283565465\\
95	185.868325579351\\
96	171.826436358793\\
97	225.368818284486\\
98	207.739266996458\\
99	177.14532168719\\
100	178.422203605049\\
101	185.782111699175\\
};
\addlegendentry{$\mu$}

\addplot [color=mycolor3, line width=1.0pt]
  table[row sep=crcr]{%
0	210.73969361064\\
1	221.772317547845\\
2	237.354933635304\\
3	182.988158881895\\
4	194.948429401163\\
5	170.202843922228\\
6	211.879622337739\\
7	188.060691888364\\
8	238.958313975853\\
9	153.599717531027\\
10	151.152223620369\\
11	183.062488650129\\
12	202.961394695745\\
13	187.864213110643\\
14	210.253941291116\\
15	243.863365925082\\
16	217.699909314917\\
17	183.47017423572\\
18	217.432783261512\\
19	213.157377559531\\
20	286.244702177811\\
21	265.420247672137\\
22	217.099461190861\\
23	241.422289922088\\
24	313.592269237459\\
25	300.998318554837\\
26	271.78635951497\\
28	310.297339031357\\
29	300.096562891979\\
30	350.596775014715\\
31	300.180998253246\\
32	361.752214805119\\
33	285.622578036265\\
34	367.145280080584\\
35	382.009240501776\\
36	351.779753837806\\
37	377.395468528159\\
38	298.836746805728\\
39	300.759243030579\\
40	303.068367428329\\
41	303.959848623009\\
42	272.89125503435\\
43	269.378327092216\\
44	275.322510027829\\
45	246.286657202268\\
46	169.09640803496\\
47	189.748642092325\\
48	188.541773938146\\
49	164.836823133139\\
50	122.709417794381\\
51	135.474915696459\\
52	118.880245460346\\
53	144.540803615811\\
54	123.233506108684\\
55	105.34154127224\\
56	132.136576331963\\
57	136.22933598205\\
58	96.6356542100833\\
59	114.002256134043\\
60	109.018508677884\\
61	100.988634438253\\
62	123.889301916284\\
63	105.946903962041\\
64	100.52588539269\\
65	123.752470835767\\
66	123.551824484037\\
67	106.292306762502\\
68	108.19193374589\\
69	111.00964462239\\
70	143.15570798544\\
71	125.915385261167\\
72	135.608898775221\\
73	118.789207610187\\
74	217.683131357746\\
75	242.912916021385\\
76	237.275901063507\\
77	221.585530542754\\
78	290.626668717298\\
79	287.925728277647\\
80	317.481639516995\\
81	310.928936983625\\
82	274.978802066944\\
83	250.080813180439\\
84	255.549845097102\\
85	248.173894966512\\
87	268.36750375837\\
88	253.343737734429\\
89	228.627107137192\\
90	208.514063912798\\
91	198.332550701041\\
92	213.300499446105\\
93	165.5461904206\\
94	176.877660949407\\
95	189.689436919304\\
96	169.372682833451\\
97	235.149957871003\\
98	188.044612440246\\
99	166.232567290021\\
100	162.229568640176\\
101	173.843465468508\\
};
\addlegendentry{system}

\end{axis}

\begin{axis}[%
width=\fwidth,
height=0.44\hwidth,
at={(0\fwidth,0\hwidth)},
scale only axis,
xmin=0,
xmax=100,
xlabel style={font=\color{white!15!black}},
xlabel={sample index [-]},
ymin=0,
ymax=450,
ylabel style={font=\color{white!15!black}},
ylabel={power [kW]},
axis background/.style={fill=white},
xmajorgrids,
ymajorgrids
]

\addplot[area legend, draw=mycolor1, fill=mycolor1, forget plot]
table[row sep=crcr] {%
x	y\\
0	255.783574584263\\
1	269.528186736117\\
2	295.253245855453\\
3	231.305067065095\\
4	239.90743257951\\
5	219.030486843595\\
6	260.995306785089\\
7	235.468162571053\\
8	291.085555179376\\
9	214.637901496024\\
10	227.10860983855\\
11	239.700656245744\\
12	244.004060701385\\
13	231.603448965223\\
14	250.117090039721\\
15	282.006307739361\\
16	276.123978453147\\
17	251.812228502727\\
18	260.445424380905\\
19	254.638397646662\\
20	318.958613952094\\
21	310.769367903506\\
22	263.952927399592\\
23	290.692732205157\\
24	354.129373755921\\
25	376.037662007458\\
26	308.535362789816\\
27	336.211399874177\\
28	358.638431221229\\
29	355.181043083934\\
30	410.743593179421\\
31	375.128056089007\\
32	443.916433635768\\
33	373.610810115906\\
34	425.508724575989\\
35	423.553855096169\\
36	391.610938686104\\
37	405.507739411694\\
38	352.843848125588\\
39	346.101280062944\\
40	338.388896614281\\
41	324.100822901214\\
42	325.225759059733\\
43	313.827460310932\\
44	304.311453510848\\
45	275.602515207903\\
46	257.712832914567\\
47	257.15062959553\\
48	248.828398443543\\
49	227.648079706631\\
50	217.891220584804\\
51	219.427005401174\\
52	216.580606639599\\
53	225.469743510094\\
54	180.829154610447\\
55	170.298877266727\\
56	204.126358077702\\
57	177.891331758938\\
58	178.323402296124\\
59	163.790705092047\\
60	168.711798903048\\
61	153.823886326047\\
62	172.39704709994\\
63	149.499648234983\\
64	160.783182827536\\
65	163.595861653884\\
66	157.269289887416\\
67	164.645879229919\\
68	176.370710883446\\
69	180.93926177773\\
70	181.318049499989\\
71	195.360328724854\\
72	195.752524454278\\
73	207.328707775911\\
74	217.771509515142\\
75	260.025718113543\\
76	276.143009738552\\
77	287.358335405616\\
78	299.232505226342\\
79	338.350210866424\\
80	329.77024354486\\
81	354.155919377162\\
82	352.475720151003\\
83	340.907513496901\\
84	320.115663226448\\
85	304.617389072669\\
86	304.965016938679\\
87	310.582406682249\\
88	302.056489823191\\
89	268.439825953991\\
90	288.496697041819\\
91	268.702205262024\\
92	278.094724610165\\
93	229.522319689211\\
94	236.26002925954\\
95	236.005962070354\\
96	208.139300070248\\
97	276.940416341736\\
98	229.942858537865\\
99	202.800445631371\\
100	200.549004564981\\
101	217.317683127294\\
102	236.358173935452\\
103	195.096920635483\\
104	215.749943147459\\
105	206.592591051561\\
106	256.770845410777\\
107	208.378215514202\\
108	249.8932385854\\
109	265.350684047731\\
110	239.359144983068\\
111	235.183847654933\\
112	276.661071967115\\
113	263.887417194515\\
114	245.281392243228\\
115	250.756144824933\\
116	282.572746493015\\
117	277.914208676284\\
118	310.456365626209\\
119	292.199038326399\\
120	280.283046990658\\
121	326.196838861529\\
122	347.434218844709\\
123	346.525092468934\\
124	332.837879772205\\
125	381.807225248323\\
126	337.006161060146\\
127	379.010186080659\\
128	383.372435000797\\
129	396.696809205469\\
130	361.440491463271\\
131	392.690349236826\\
132	380.451347137812\\
133	353.095207791538\\
134	353.594311457899\\
135	337.777470701185\\
136	338.328895673406\\
137	329.414741029899\\
138	325.802045989566\\
139	294.733153132131\\
140	288.686094648333\\
141	267.644785697562\\
142	259.14578683865\\
143	250.573060003109\\
144	235.352158624672\\
145	220.039656197083\\
146	228.036789382633\\
147	210.277237332475\\
148	230.275904818333\\
149	212.400147817994\\
150	218.035264046033\\
151	163.733190303251\\
152	165.788834005984\\
153	189.947068772505\\
154	155.216983588803\\
155	158.551377245804\\
156	161.580604360472\\
157	174.572234894247\\
158	165.252781724123\\
159	165.949214185376\\
160	163.001931184144\\
161	177.734217102112\\
162	182.6188090713\\
163	177.562575143994\\
164	174.240322196804\\
165	176.226172437863\\
166	183.160261808948\\
167	184.718350552194\\
168	195.249912242488\\
169	200.68291176169\\
170	241.966059138928\\
171	242.538176824913\\
172	270.285465770998\\
173	303.508670448294\\
174	295.200439615612\\
175	363.754208705574\\
176	360.329817865398\\
177	369.213812370621\\
178	362.977907238255\\
179	341.851022241601\\
180	331.696912628554\\
181	346.148959282676\\
182	331.642260316192\\
183	304.955775308014\\
184	332.452677755956\\
185	310.231745025377\\
186	304.369645907378\\
187	274.906897582393\\
188	268.670046563741\\
189	252.447380613221\\
190	270.333774496286\\
191	246.113501016508\\
192	232.957068520746\\
193	299.0837474035\\
194	248.216980907641\\
195	221.805768451429\\
196	261.051587587824\\
197	229.65878165004\\
198	279.269293304133\\
199	257.609438626437\\
200	237.328712635808\\
201	259.629613584545\\
202	261.264963346101\\
203	254.353027247722\\
204	261.873778109193\\
205	227.579217188536\\
206	264.38526818032\\
207	236.35823610727\\
208	284.203068131306\\
209	280.355387233833\\
210	254.812275937678\\
211	287.283085978447\\
212	295.996563556138\\
213	311.165610878625\\
214	300.304427965982\\
215	300.709008524429\\
216	322.348932237257\\
217	325.554712644753\\
218	371.211117962486\\
219	400.256331625724\\
220	357.070515035443\\
221	382.891244436603\\
222	395.008211082677\\
223	434.084516542804\\
224	398.665847621925\\
225	450.830687005488\\
226	420.304363545412\\
227	434.733784313205\\
228	405.3176553946\\
229	424.638887303029\\
230	368.932087607461\\
231	415.38499060746\\
232	387.367036601166\\
233	354.283747157555\\
234	331.996907843815\\
235	332.430331927091\\
236	328.892402101598\\
237	306.710328408735\\
238	289.792730932092\\
239	272.942881645245\\
240	265.830949454622\\
241	255.288062240828\\
242	262.109988371321\\
243	240.22356132229\\
244	236.079554271972\\
245	242.742436767235\\
246	207.926592032198\\
247	226.921431270184\\
248	227.998595427685\\
249	205.987720557077\\
250	176.758032721253\\
251	204.320424297446\\
252	197.227400016219\\
253	172.827064730771\\
254	188.870946916598\\
255	170.873269936786\\
256	176.180848046542\\
257	189.012581003187\\
258	185.772968381206\\
259	178.521419976003\\
260	191.401385136133\\
261	183.809200202628\\
262	190.050236061882\\
263	197.591622259177\\
264	202.245139081638\\
265	222.458742807767\\
266	261.719436292245\\
267	243.251417366487\\
268	308.281595579553\\
269	271.311111252277\\
270	335.851998453744\\
271	333.222831850875\\
272	343.715763712462\\
273	343.457331467444\\
274	370.566923897653\\
275	388.269729935072\\
276	355.508612755103\\
277	402.526892417111\\
278	386.729204986044\\
279	340.376056166217\\
280	325.456281869779\\
281	333.838083063254\\
282	340.232852528303\\
283	286.889527524121\\
284	300.920658756118\\
285	299.186671205595\\
286	273.965200759514\\
287	258.22040832324\\
288	290.208738315714\\
289	318.085874141787\\
290	285.163490661694\\
291	298.247638081526\\
292	287.558062274037\\
293	306.315285452031\\
294	287.922053805873\\
295	299.098002637738\\
296	257.530174678168\\
297	271.241365248246\\
298	304.309646679973\\
299	279.505520165529\\
300	269.278245914861\\
301	322.409219382491\\
302	333.712133262818\\
303	257.322039432212\\
304	290.14251313255\\
305	285.802887879605\\
306	309.29960433265\\
307	264.350188755077\\
308	329.948028573492\\
309	276.753515767735\\
310	353.237743998736\\
311	303.609512217413\\
312	371.932520231004\\
313	365.771258608891\\
314	399.14565035366\\
315	366.804064764117\\
316	371.761823267441\\
317	389.113760820924\\
318	438.748317893649\\
319	407.928506623036\\
320	421.841986559616\\
321	413.626979776856\\
322	457.701742498101\\
323	414.723635110744\\
324	458.563719305711\\
325	416.081730056593\\
326	434.899594172213\\
327	403.143121077628\\
328	373.712021998318\\
329	415.830270923948\\
330	371.191366866587\\
331	349.03083964863\\
332	350.098898710116\\
333	351.668922819117\\
334	353.518659160084\\
335	312.401092857735\\
336	325.864453379047\\
337	318.24276461369\\
338	296.011121796837\\
339	283.465798821749\\
340	287.168570708024\\
341	289.774565666882\\
342	296.135915105412\\
343	282.538132994042\\
344	262.53464705607\\
345	262.31982358156\\
346	277.170712269299\\
347	276.345952670638\\
348	226.584974533219\\
349	233.63997476534\\
350	233.578205713107\\
351	210.597988275722\\
352	223.767126138132\\
353	223.667057383322\\
354	207.370769684859\\
355	207.474609356015\\
356	206.550366054661\\
357	209.301008920111\\
358	202.898407585963\\
359	208.243175201198\\
360	218.090024296183\\
361	237.540023176618\\
362	258.010439889249\\
363	257.084795080806\\
364	310.557622345665\\
365	308.078501230475\\
366	304.628967007961\\
367	375.192479181214\\
368	392.141669474656\\
369	380.049445104496\\
370	392.495647869353\\
371	420.589797228309\\
372	407.340980205312\\
373	399.272162803309\\
374	392.36663787224\\
375	355.519430341453\\
376	390.784688868334\\
377	387.164814621785\\
378	340.543085579861\\
379	307.855253053493\\
380	345.302082481296\\
381	320.256559609945\\
382	302.562587954035\\
383	330.282536195575\\
384	302.450500348236\\
385	302.643278024734\\
386	341.014986737263\\
387	355.246552952634\\
388	306.858268073968\\
389	278.693241695974\\
390	281.828618740157\\
391	269.464350149541\\
392	283.685479126283\\
393	325.619793846467\\
394	313.12569868143\\
395	305.955997770806\\
396	265.83694117251\\
397	289.454487269328\\
398	335.382366852184\\
399	320.537339853335\\
400	278.546715150709\\
401	294.20349779552\\
402	339.429337522077\\
403	310.844830673761\\
404	347.493601330343\\
405	299.491442369286\\
406	372.610592570985\\
407	331.053684035514\\
408	376.137577630495\\
409	392.272362691607\\
410	383.649024979164\\
411	394.780550470943\\
412	375.475729919099\\
413	447.044945625133\\
414	427.031531835263\\
415	452.024203192444\\
416	401.706113171882\\
417	425.325984559758\\
418	403.204557627661\\
419	465.664117438101\\
420	448.675728190845\\
421	441.422467063475\\
422	385.624671563239\\
423	397.338150107822\\
424	396.885751078377\\
425	360.915946440104\\
426	376.514585455942\\
427	357.706796895512\\
428	335.464804330895\\
429	332.314817889977\\
430	321.212112831714\\
431	311.501476787245\\
432	300.825011774067\\
433	295.727584634011\\
434	303.829716509942\\
435	299.997885135711\\
436	277.404911593397\\
437	290.523333034338\\
438	296.458263232439\\
439	261.705186057473\\
440	255.586361693343\\
441	244.491212569902\\
442	232.737439211544\\
443	247.927546213589\\
444	244.507995869815\\
445	213.444497869664\\
446	247.541710419494\\
447	223.88311750352\\
448	207.125627457676\\
449	198.656989821608\\
450	195.270909727445\\
451	190.581538137612\\
452	204.022235771006\\
453	201.155026460296\\
454	201.478915708618\\
455	197.62036377143\\
456	207.409278269266\\
457	224.328875043998\\
458	225.186325416315\\
459	282.663709243725\\
460	302.377451204015\\
461	288.130691800733\\
462	314.965069742356\\
463	401.33804526676\\
464	385.800720017698\\
465	399.427405881961\\
466	403.910276720499\\
467	385.797824259086\\
468	389.69753758424\\
469	378.687077820619\\
470	365.448161229179\\
471	361.420410995592\\
472	328.70318181487\\
473	334.233253243844\\
474	357.742866490033\\
475	321.798887952563\\
476	353.468985149035\\
477	291.481808810503\\
478	336.03366963704\\
479	276.311499719153\\
480	278.073154635138\\
481	273.951266010642\\
482	294.888613282982\\
483	338.595605458022\\
484	294.784164775613\\
485	263.343926181963\\
486	278.272317147454\\
487	301.027801381713\\
488	343.790108884143\\
489	350.629665057075\\
490	299.127966303556\\
491	308.712767395009\\
492	300.400429031219\\
493	293.111697224595\\
494	272.354557962892\\
495	271.595975873108\\
496	277.582376524678\\
497	299.732668372285\\
498	280.872438325427\\
499	307.908022496758\\
500	284.356431854962\\
501	285.143850406394\\
502	363.983145060488\\
503	352.498074132146\\
504	330.983529941623\\
505	391.73507189467\\
506	384.581911592757\\
507	365.29269944191\\
508	386.29675061176\\
509	371.782742982321\\
510	375.097831169263\\
511	406.326505961487\\
512	413.323878474562\\
513	410.225914993418\\
514	454.467909608214\\
515	403.56342875084\\
516	447.819235353001\\
517	415.583149385822\\
518	409.96936228442\\
519	434.840499799022\\
520	383.645539101378\\
521	412.055586082641\\
522	414.83457260587\\
523	396.97566370717\\
524	359.989604134797\\
525	373.08506196851\\
526	333.351321310269\\
527	337.675772249573\\
528	330.108377251041\\
529	313.086998404134\\
530	296.057670889589\\
531	308.750635095474\\
532	304.852003500499\\
533	282.708400629824\\
534	297.203192235364\\
535	286.642361156572\\
536	269.010487379725\\
537	256.963186879497\\
538	257.419767514038\\
539	245.373984219494\\
540	259.619492527609\\
541	235.241070302942\\
542	244.276909451928\\
543	217.922241438474\\
544	223.44876557106\\
545	225.541906876527\\
546	221.318719214078\\
547	220.029080180529\\
548	210.10778309652\\
549	208.34168311411\\
550	219.777438610811\\
551	239.604635155561\\
552	224.727749187264\\
553	253.100719901049\\
554	246.175359945982\\
555	329.149748510616\\
556	348.142729515449\\
557	298.352836776407\\
558	365.632784172574\\
559	386.713943596248\\
560	397.30701423048\\
561	409.688868321983\\
562	376.109008474831\\
563	380.514408287572\\
564	367.757312991518\\
565	397.295165431204\\
566	362.195014622981\\
567	351.220049148176\\
568	345.896064939926\\
569	348.789161149357\\
570	300.131800163087\\
571	266.288636552849\\
572	262.629871143429\\
573	247.122201566099\\
574	228.375728221598\\
575	240.438548282531\\
576	272.495089514616\\
577	238.77941858807\\
578	252.788950565378\\
579	296.091927850802\\
580	234.041113445564\\
581	281.758452029422\\
582	229.738320955185\\
583	216.650540105649\\
584	248.640238968335\\
585	235.953162208592\\
586	239.12425800681\\
587	254.71092207766\\
588	236.899568294763\\
589	253.496740617839\\
590	240.928473742366\\
591	274.865905189277\\
592	284.316660493867\\
593	267.277586015181\\
594	273.253915796178\\
595	256.570156831843\\
596	268.883479573831\\
597	308.522376492144\\
598	277.208575368282\\
599	306.154964381187\\
600	306.144511935506\\
601	338.594998308462\\
602	318.942572086595\\
603	377.425463162358\\
604	357.930655796446\\
605	366.980784790814\\
606	408.255415841623\\
607	393.246898635579\\
608	412.518155190516\\
609	421.182197391703\\
610	380.813078571395\\
611	390.17997604051\\
612	377.164273645041\\
613	379.484682900178\\
614	378.926675516672\\
615	365.641354640711\\
616	347.482990027454\\
617	325.699029825215\\
618	329.897728301695\\
619	300.856480053396\\
620	286.89116404032\\
621	277.444204302054\\
622	272.985314372911\\
623	259.00475497518\\
624	247.914277450841\\
625	226.221233162994\\
626	221.322281800949\\
627	216.309999946598\\
628	215.7681557565\\
629	218.287626696568\\
630	210.757761029439\\
631	198.148440391267\\
632	199.748849692096\\
633	169.251114739645\\
634	156.243366024847\\
635	194.104878573867\\
636	164.353208702682\\
637	154.902880325366\\
638	161.266168701104\\
639	176.580684033221\\
640	162.090349145366\\
641	186.700924569298\\
642	173.871199340431\\
643	182.65519038341\\
644	180.251246597473\\
645	195.39638453165\\
646	191.410119780072\\
647	204.700670300248\\
648	206.55964341604\\
649	215.29140103246\\
650	247.676579168606\\
651	263.833771924256\\
652	301.464990641566\\
653	293.246627121976\\
654	294.033350260613\\
655	348.6551314353\\
656	371.615398957915\\
657	359.260064250804\\
658	363.083493434053\\
659	339.526939806625\\
660	322.409360523502\\
661	309.448292782048\\
662	310.022080636473\\
663	298.489633211596\\
664	290.223544038792\\
665	295.040384658367\\
666	270.033973188945\\
667	266.14288250166\\
668	293.217367430281\\
669	235.70505385581\\
670	226.882729325547\\
671	229.858227786702\\
672	247.608584571747\\
673	231.597973991909\\
674	247.627684981402\\
675	214.209748587392\\
676	266.766292364412\\
677	252.686731759989\\
678	267.454396558829\\
679	229.66932154629\\
680	240.677347979294\\
681	229.654335671541\\
682	231.942110867044\\
683	250.298176833315\\
684	251.104139529011\\
685	247.288719768403\\
686	255.114377557662\\
687	250.039745458825\\
688	264.486278653605\\
689	257.352122667612\\
690	257.770402991423\\
691	254.960785287651\\
692	288.263466100719\\
693	269.637709791324\\
694	287.016578628256\\
695	282.837175454439\\
696	324.886976344043\\
697	348.633925620005\\
698	325.575667555564\\
699	369.71379492537\\
700	338.619685591752\\
701	365.691505408506\\
702	356.607178777218\\
703	398.506667719839\\
704	366.930692154043\\
705	398.83678239679\\
706	373.325817407414\\
707	392.839768698448\\
708	369.789234110554\\
709	355.489738022199\\
710	367.184592156682\\
711	349.218958157264\\
712	340.742857566469\\
713	333.872743223708\\
714	311.204455682886\\
715	292.917221142647\\
716	281.519367158993\\
717	269.280281616128\\
718	261.144530357871\\
719	247.433128014906\\
720	242.661358004521\\
721	230.380126407322\\
722	222.20815897522\\
723	215.330961212482\\
724	226.368911345237\\
725	222.47031234599\\
726	178.950738615705\\
727	172.57223684472\\
728	177.021475962051\\
729	189.157611598283\\
730	161.787199168935\\
731	170.089857463203\\
732	163.038237563608\\
733	172.676658900028\\
734	172.171993103117\\
735	173.816803805005\\
736	168.894543357035\\
737	176.508012508685\\
738	161.997957515234\\
739	172.122196898779\\
740	183.464169015832\\
741	179.611300813337\\
742	185.567824079004\\
743	200.09802183455\\
744	191.250013759978\\
745	208.538537470296\\
746	218.037843360581\\
747	254.826822594218\\
748	278.879680515509\\
749	283.590913126562\\
750	309.415744127516\\
751	342.207576750163\\
752	367.007076347226\\
753	357.578933106719\\
754	361.239819927724\\
755	332.271734901625\\
756	326.748083012132\\
757	329.950653370321\\
758	324.101616162004\\
759	328.705516851512\\
760	302.978511206694\\
761	291.171229119015\\
762	286.469094837803\\
763	272.824837241642\\
764	265.49733821691\\
765	273.188149767185\\
766	256.813959842738\\
767	258.872229121804\\
768	293.393425354751\\
769	285.434220592202\\
770	226.968772161878\\
771	227.636250754375\\
772	281.873749066653\\
773	303.268174521218\\
774	295.20974613972\\
775	273.416806169057\\
776	250.573902495729\\
777	293.075643385651\\
778	314.204617274952\\
779	270.719634684866\\
780	316.339533216967\\
781	262.57615794095\\
782	310.866210422179\\
783	300.368023719412\\
784	294.698847667396\\
785	267.075853103431\\
786	328.64179536302\\
787	299.600617929047\\
788	294.678012625389\\
789	288.576167788934\\
790	347.739229685771\\
791	348.254932090386\\
792	375.2288584363\\
793	374.358346202683\\
794	391.710597901296\\
795	357.945379216036\\
796	399.75578786277\\
797	387.123788219946\\
798	414.066725047\\
799	391.101148372479\\
800	482.557601539245\\
801	428.133401836463\\
802	404.327976694098\\
803	472.14651157831\\
804	444.150924765623\\
805	430.70043414491\\
806	406.337156423064\\
807	427.948081157715\\
808	386.840005609162\\
809	423.208542524079\\
810	387.128962120114\\
811	361.750037688838\\
812	378.067051268419\\
813	345.040629494194\\
814	357.924170478863\\
815	337.921861051972\\
816	315.618231739811\\
817	309.813998879978\\
818	293.469126770082\\
819	279.364694736515\\
820	288.692454317233\\
821	277.601279238722\\
822	284.74307187387\\
823	268.603321619483\\
824	267.435694104549\\
825	255.203775575897\\
826	254.595029631155\\
827	263.229598979155\\
828	234.581938274547\\
829	229.450116500375\\
830	237.4048756477\\
831	234.166164453589\\
832	221.660033650947\\
833	253.255738942192\\
834	225.910589154607\\
835	222.84473984475\\
836	236.698886188022\\
837	225.683311249194\\
838	228.784788290463\\
839	230.601543669288\\
840	243.628412703133\\
841	263.945524907964\\
842	255.76088211354\\
843	350.220458936861\\
844	297.057672582881\\
845	333.769067531212\\
846	357.657924017426\\
847	448.101575855877\\
848	407.062450431681\\
849	406.000060643288\\
850	438.9465213319\\
851	380.624242664015\\
852	414.012312149857\\
853	416.959161552111\\
854	405.10893342874\\
855	366.537230118289\\
856	364.383874885962\\
857	376.413991066795\\
858	346.251845921794\\
859	323.021366511821\\
860	341.188556050048\\
861	354.711174624595\\
862	303.560410476107\\
863	315.577378093334\\
864	353.916755198095\\
865	282.887829331895\\
866	304.330464527365\\
867	293.620368179981\\
868	343.804618169227\\
869	328.774878404133\\
870	354.678393443784\\
871	352.912577865452\\
872	342.926947473554\\
873	335.640804659636\\
874	277.343326128336\\
875	331.08537765751\\
876	313.788148675952\\
877	323.205812778916\\
878	330.824098313132\\
879	273.138926334895\\
880	300.921812587741\\
881	312.905654655233\\
882	352.509261304685\\
883	360.357686068448\\
884	308.194357502983\\
885	333.665549827224\\
886	344.497494557088\\
887	301.003040781532\\
888	365.613698956045\\
889	379.643986103989\\
890	388.71676110933\\
891	377.844142480204\\
892	432.90753645582\\
893	369.214714827161\\
894	390.883381742461\\
895	444.58256364805\\
896	395.708780110314\\
897	422.257403177414\\
898	405.70252998644\\
899	401.516623460057\\
900	396.11061356122\\
901	419.136716609197\\
902	419.810595681846\\
903	391.777614649254\\
904	410.355239017573\\
905	391.464814303114\\
906	378.558143114428\\
907	337.580861456187\\
908	318.792583398937\\
909	334.608478564494\\
910	314.798131522101\\
911	283.299203344667\\
912	292.926349854323\\
913	283.644192729246\\
914	278.531329435525\\
915	258.702907673003\\
916	246.945050983569\\
917	246.953699261857\\
918	262.228273812162\\
919	218.736669263609\\
920	251.263030411735\\
921	196.966534387578\\
922	188.001036555082\\
923	207.196955980901\\
924	183.756392436006\\
925	216.339086266575\\
926	218.277968496149\\
927	182.100538382188\\
928	204.934307692818\\
929	197.387310588333\\
930	185.271724210792\\
931	194.045001170003\\
932	188.248696815032\\
933	185.49790010782\\
934	202.963522988966\\
935	207.597269023544\\
936	202.179167119023\\
937	209.070018207376\\
938	250.118827255603\\
939	255.051161608293\\
940	306.452639835301\\
941	298.439551836476\\
942	306.137213007078\\
943	364.200127038287\\
944	380.455153398435\\
945	374.741582787531\\
946	369.713110326839\\
947	393.568506844455\\
948	409.844781166747\\
949	417.347909460028\\
950	419.080234441478\\
951	404.782734633419\\
952	369.715401139827\\
953	345.276837327308\\
954	346.42266899922\\
955	353.666487554235\\
956	371.985353716667\\
957	332.759487323987\\
958	314.828458498439\\
959	333.046370780532\\
960	303.577373061737\\
961	343.564889662347\\
962	338.424119737987\\
963	294.249004989679\\
964	310.031635018073\\
965	324.16857999588\\
966	278.416635165452\\
967	333.628268009425\\
968	281.62469103126\\
969	261.175798909653\\
970	293.807070376727\\
971	319.698681647762\\
972	281.883447896462\\
973	262.314135281332\\
974	306.071376116037\\
975	253.40959026908\\
976	316.295250758197\\
977	319.600838738021\\
978	279.418117070923\\
979	261.083692327249\\
980	328.500979520783\\
981	325.330547850233\\
982	337.563797318098\\
983	343.047203205209\\
984	345.189882247271\\
985	355.421126327279\\
986	399.201813024862\\
987	383.004833125099\\
988	390.380600264644\\
989	422.086360182033\\
990	388.313152579476\\
991	444.778033001659\\
992	389.121626390332\\
993	424.859317858275\\
994	415.878737569136\\
995	398.733861787874\\
996	393.243769345937\\
997	402.325431424635\\
998	413.141376557296\\
999	416.742270935999\\
1000	383.493401803502\\
1001	375.322521386704\\
1002	375.675053317169\\
1003	327.868048172632\\
1004	328.51724317186\\
1005	317.590214840657\\
1006	324.324939550323\\
1007	294.703090852834\\
1008	289.64929776026\\
1009	263.530734477953\\
1010	248.046346635047\\
1011	256.972332682635\\
1012	237.055939838638\\
1013	245.568888584829\\
1014	214.44326932774\\
1015	212.200269634019\\
1016	185.040368264177\\
1017	188.93930732127\\
1018	190.838907120305\\
1019	205.896090253105\\
1020	200.265120261842\\
1021	162.268473765913\\
1022	176.359696441688\\
1023	190.403280986203\\
1024	160.153455239467\\
1025	178.261354092251\\
1026	170.852031195767\\
1027	181.093317393209\\
1028	177.099626813931\\
1029	178.232231445927\\
1030	196.479594691727\\
1031	192.465717996224\\
1032	200.738582159285\\
1033	219.306800955014\\
1034	257.053552780554\\
1035	248.739411370562\\
1036	301.392847830274\\
1037	291.330931133746\\
1038	289.580876317956\\
1039	355.046642688727\\
1040	366.210661365773\\
1041	338.32544624989\\
1042	376.199718771214\\
1043	395.85017538442\\
1044	367.635108242837\\
1045	361.028954532137\\
1046	384.124740757551\\
1047	345.258765746472\\
1048	334.887178753012\\
1049	309.260769266079\\
1050	327.154326819534\\
1051	339.734191572001\\
1052	316.469394235266\\
1053	263.423546219094\\
1054	260.405662705007\\
1055	270.745667165435\\
1056	244.551374757589\\
1057	234.549119057666\\
1058	256.195024665272\\
1059	290.536215309387\\
1060	306.878439768232\\
1061	271.549169372446\\
1062	216.631256767762\\
1063	275.560501380943\\
1064	288.623116935983\\
1065	224.323677325353\\
1066	232.82249283508\\
1067	264.893112230733\\
1068	243.29400664206\\
1069	256.887887154553\\
1070	235.985097404672\\
1071	251.787462534946\\
1072	239.115849826755\\
1073	255.241775668738\\
1074	282.968713410511\\
1075	276.557916908387\\
1076	257.551336634082\\
1077	310.198059761879\\
1078	324.366858066281\\
1079	312.072144945156\\
1080	333.876119919881\\
1081	332.29464406912\\
1082	340.665870590458\\
1083	359.617915849605\\
1084	340.496963878533\\
1085	405.792797521937\\
1086	400.339245903369\\
1087	369.206696051471\\
1088	412.467959352882\\
1089	396.824047143478\\
1090	385.803071765836\\
1091	397.097761201039\\
1092	378.398727617322\\
1093	368.979503018726\\
1094	364.216807710391\\
1095	345.846530598989\\
1096	338.223955458401\\
1097	335.996396130603\\
1098	339.985397706812\\
1099	301.668026799187\\
1100	285.721406004118\\
1101	289.293852624157\\
1102	263.379792252387\\
1103	252.364322303918\\
1104	239.76475477128\\
1105	232.947518861865\\
1106	230.729432043107\\
1107	220.357439924918\\
1108	224.992053988626\\
1109	216.936420087036\\
1110	184.649864360133\\
1111	171.29539532266\\
1112	185.27872652114\\
1113	177.250522902717\\
1114	168.652533702565\\
1115	158.382790379983\\
1116	153.377074413629\\
1117	157.35106450179\\
1118	170.41263921501\\
1119	166.200419060988\\
1120	160.942203394492\\
1121	165.561721113818\\
1122	167.782831567269\\
1123	176.545937030404\\
1124	177.502175414665\\
1125	188.992313726095\\
1126	205.80776902808\\
1127	212.833907215034\\
1128	221.227720816036\\
1129	221.921357190836\\
1130	233.128483365552\\
1131	295.677075062251\\
1132	285.544033775803\\
1133	295.3888666185\\
1134	307.351571979765\\
1135	328.682807133428\\
1136	344.897953092983\\
1137	338.587550742645\\
1138	347.833338164052\\
1139	324.281691730593\\
1140	308.635508916331\\
1141	302.732877388147\\
1142	302.206353604214\\
1143	294.186637219992\\
1144	293.018254652608\\
1145	273.48377761907\\
1146	276.241287366364\\
1147	255.30364496583\\
1148	250.091238811266\\
1149	251.859737910136\\
1150	233.733835383306\\
1151	245.853787630539\\
1152	251.733411054012\\
1153	209.168684639631\\
1154	247.671186430471\\
1155	234.771620951272\\
1156	224.966922602908\\
1157	231.670484696179\\
1158	257.045519256253\\
1159	233.275156645362\\
1160	233.727281476321\\
1161	250.511059743537\\
1162	219.740341213497\\
1163	222.938052379515\\
1164	226.814468151911\\
1165	232.827983475538\\
1166	236.590429843844\\
1167	242.64331820709\\
1168	253.116468690183\\
1169	273.662058934085\\
1170	243.248030977122\\
1171	256.200120157031\\
1172	286.404091414053\\
1173	287.904945765599\\
1174	282.669164008725\\
1175	283.422912659752\\
1176	328.511241781464\\
1177	321.950502742497\\
1178	332.617479353026\\
1179	328.760708489164\\
1180	339.446315378394\\
1181	340.049405060446\\
1182	356.683905842086\\
1183	385.974988634845\\
1184	374.3115394082\\
1185	385.64980550708\\
1186	382.969582608319\\
1187	376.196023754571\\
1188	362.111897419337\\
1189	344.839109334794\\
1190	347.999929568856\\
1191	339.206765708732\\
1192	329.924820935589\\
1193	312.826742738749\\
1194	299.7359317687\\
1195	282.483035326781\\
1196	270.315796343645\\
1197	265.721650917626\\
1198	259.269239566821\\
1199	246.847310264356\\
1200	230.714468575621\\
1201	218.158143645359\\
1202	219.898519180083\\
1203	218.6670341059\\
1204	210.543341578965\\
1205	236.657357286433\\
1206	173.652817825164\\
1207	166.052843351219\\
1208	166.988539696663\\
1209	164.145456759472\\
1210	150.799872615968\\
1211	155.612707053434\\
1212	155.797450895945\\
1213	153.065839852355\\
1214	156.146320516713\\
1215	155.046507537797\\
1216	155.772618341466\\
1217	172.144126491597\\
1218	173.424866984548\\
1219	167.261412054709\\
1220	170.015473069387\\
1221	181.253947205166\\
1222	193.853408554638\\
1223	203.55282717096\\
1224	212.413700153572\\
1225	223.836594373572\\
1226	248.200365279956\\
1227	274.070129211171\\
1228	290.449903359653\\
1229	289.528973059176\\
1230	301.115656665217\\
1231	338.120433386483\\
1232	338.732169863603\\
1233	339.078870366773\\
1234	345.329038034256\\
1235	340.267940238461\\
1236	330.997230896871\\
1237	319.696929290351\\
1238	307.111242795944\\
1239	305.152655011545\\
1240	279.521401566318\\
1241	270.118288211172\\
1242	276.419613763925\\
1243	296.512112923276\\
1244	242.622727235663\\
1245	229.584258938516\\
1246	264.871990126214\\
1247	259.126107066963\\
1248	248.245627404614\\
1249	275.129829267741\\
1250	232.659415455463\\
1251	256.205009033823\\
1252	259.4142465947\\
1253	203.509435859171\\
1254	208.937557940714\\
1255	249.316549178883\\
1256	208.661835714113\\
1257	260.893739937541\\
1258	273.871681622038\\
1259	261.911823970766\\
1260	255.152370972347\\
1261	256.714800309792\\
1262	213.933701798502\\
1263	248.715057426884\\
1264	241.543632451659\\
1265	269.978890117767\\
1266	273.568237171257\\
1267	282.328760057468\\
1268	243.663082191381\\
1269	270.718810120507\\
1270	304.197934412385\\
1271	280.097482081893\\
1272	293.117384565297\\
1273	325.366715881939\\
1274	328.883185575073\\
1275	326.582635571101\\
1276	380.610734883753\\
1277	363.951931620689\\
1278	372.977038309294\\
1279	343.601204425388\\
1280	361.754219532391\\
1281	383.577641400663\\
1282	380.985504763692\\
1283	373.327203002406\\
1284	364.645826947054\\
1285	349.590492352504\\
1286	337.270832214028\\
1287	340.701762242292\\
1288	323.454238331064\\
1289	306.13358086712\\
1290	298.94009052372\\
1291	287.166947669652\\
1292	268.94703973174\\
1293	255.234718017712\\
1294	256.190656515193\\
1295	250.508328338815\\
1296	240.573749546082\\
1297	233.282893701274\\
1298	232.231494247635\\
1299	222.501068257061\\
1300	233.56065093952\\
1301	225.673627272641\\
1302	190.280416090842\\
1303	177.150371342024\\
1304	177.353919967452\\
1305	174.602955238305\\
1306	180.418062698109\\
1307	181.162764762618\\
1308	175.761313062712\\
1309	179.358746648302\\
1310	196.279851626103\\
1311	183.29531011207\\
1312	179.854209994209\\
1313	182.047636818077\\
1314	188.609870814306\\
1315	201.359792671385\\
1316	198.138793729452\\
1317	207.410152862001\\
1318	212.628120726428\\
1319	221.930681727471\\
1320	248.02272163927\\
1321	249.309358720353\\
1322	248.350481172315\\
1323	291.916724596302\\
1324	276.041223285518\\
1325	290.035539363286\\
1326	317.220060302458\\
1327	347.806284281693\\
1328	346.881589752021\\
1329	342.640712317241\\
1330	345.968423513468\\
1331	345.04967789042\\
1332	334.056552644608\\
1333	325.499550173524\\
1334	323.272295744837\\
1335	312.381427692807\\
1336	321.06582316386\\
1337	328.760032777567\\
1338	274.810733909617\\
1339	270.834420141164\\
1340	299.391792655545\\
1341	270.136650612369\\
1342	300.097913553331\\
1343	266.970748631972\\
1344	243.526586010934\\
1345	266.02374627316\\
1346	272.837903022401\\
1347	232.67768104908\\
1348	253.331238192123\\
1349	242.571333740179\\
1350	268.511828735833\\
1351	261.643192256396\\
1352	258.085009332292\\
1353	259.396737274078\\
1354	283.941861415788\\
1355	235.90392215036\\
1356	236.685449650274\\
1357	245.183543172559\\
1358	268.202050661142\\
1359	226.291196496387\\
1360	298.123866911297\\
1361	245.546865918058\\
1362	241.811578090537\\
1363	285.524617465316\\
1364	291.236707480299\\
1365	319.355182694367\\
1366	269.617481516484\\
1367	296.349291869108\\
1368	292.686184980823\\
1369	312.882054794204\\
1370	328.499236906243\\
1371	338.416899536167\\
1372	345.609786226357\\
1373	377.596965808492\\
1374	390.815708319624\\
1375	389.214585761565\\
1376	408.426066427964\\
1377	403.425579426036\\
1378	407.133594062248\\
1379	381.181987462056\\
1380	402.091514787798\\
1381	355.430972322325\\
1382	349.860114872554\\
1383	343.432827304125\\
1384	328.833360727758\\
1385	341.666496433413\\
1386	333.266513771496\\
1387	307.109060580106\\
1388	298.145849508017\\
1389	277.822164044462\\
1390	261.903087433571\\
1391	263.805665421555\\
1392	238.796222387667\\
1393	239.264855072923\\
1394	226.328387175765\\
1395	215.713770262957\\
1396	215.616455478276\\
1397	217.047588565655\\
1398	184.415106809946\\
1399	176.202865587728\\
1400	168.366141060342\\
1401	156.231854999883\\
1402	163.008392355662\\
1403	161.528229734641\\
1404	155.856675753691\\
1405	150.971245419471\\
1406	164.042392436407\\
1407	150.538460395057\\
1408	154.923547462375\\
1409	157.133476691\\
1410	168.493048655586\\
1411	160.281942616132\\
1412	165.624158585853\\
1413	176.421484789542\\
1414	189.154482088168\\
1415	183.37659425941\\
1416	192.813468136986\\
1417	215.323641171496\\
1418	237.751320024456\\
1419	260.067208684618\\
1420	254.952241612206\\
1421	271.565863060992\\
1422	313.554831687652\\
1423	337.511822279418\\
1424	351.471092521762\\
1425	366.734482970271\\
1426	366.016113057491\\
1427	340.586279888755\\
1428	337.789066303957\\
1429	350.662461195276\\
1430	326.176480352955\\
1431	308.85110868867\\
1432	316.555213582527\\
1433	303.511208636596\\
1434	284.983480416317\\
1435	288.32805857477\\
1436	263.223476723962\\
1437	267.955386213791\\
1438	271.871182404533\\
1439	304.410790185085\\
1440	275.063257789723\\
1441	270.523316248173\\
1442	326.758630241019\\
1443	290.445377379744\\
1444	266.531399236923\\
1445	295.921836783563\\
1446	248.710812332148\\
1447	299.875653906946\\
1448	318.491288597256\\
1449	264.444751075368\\
1450	314.072183268629\\
1451	280.856802856428\\
1452	298.418418207272\\
1453	258.222604547216\\
1454	261.122777103004\\
1455	312.005089103863\\
1456	265.838109059163\\
1457	308.196466031979\\
1458	318.375443843193\\
1459	316.832943362415\\
1460	324.706058583182\\
1461	267.470663802876\\
1462	348.098392179316\\
1463	365.74844781627\\
1464	376.31220369251\\
1465	364.31238338387\\
1466	356.326071472807\\
1467	384.943875878664\\
1468	376.941620497208\\
1469	413.556422378669\\
1470	431.856872283009\\
1471	377.667852187173\\
1472	447.859156834473\\
1473	423.219522666068\\
1474	439.672415241901\\
1475	426.744302684803\\
1476	376.949097242936\\
1477	374.108768957834\\
1478	364.583001473916\\
1479	368.739260133598\\
1480	392.719991593208\\
1481	369.003926316554\\
1482	352.631947223817\\
1483	347.2028688068\\
1484	321.652991752821\\
1485	299.054134288241\\
1486	288.583261138565\\
1487	274.844375992189\\
1488	271.840291862668\\
1489	270.475755186801\\
1490	267.491719425674\\
1491	247.406481608524\\
1492	255.528608837922\\
1493	235.727953770333\\
1494	242.525052802914\\
1495	221.75823009901\\
1496	218.193972341735\\
1497	245.869923542613\\
1498	232.85826905492\\
1499	201.002331338701\\
1500	190.073694024895\\
1501	217.786854686521\\
1502	183.568234573823\\
1503	179.74301141968\\
1504	177.529411780295\\
1505	189.338217277112\\
1506	174.448739480772\\
1507	169.532661405969\\
1508	191.727093568613\\
1509	176.58758349754\\
1510	181.174000840016\\
1511	181.402348096255\\
1512	211.882082627356\\
1513	198.451548690832\\
1514	229.664644733964\\
1515	238.18326778625\\
1516	286.551763959995\\
1517	294.463154749463\\
1518	296.024793896262\\
1519	378.334112971919\\
1520	361.9078980951\\
1521	350.261373340834\\
1522	395.305864316878\\
1523	357.931385388433\\
1524	354.014936625168\\
1525	351.279853775579\\
1526	354.465271007997\\
1527	341.805183928857\\
1528	352.12857322243\\
1529	330.297843437718\\
1530	349.532688572466\\
1531	293.241433816842\\
1532	327.333634102121\\
1533	336.496293791532\\
1534	282.091136202704\\
1535	331.59622916538\\
1536	335.817188726324\\
1537	284.876822092315\\
1538	293.858118516112\\
1539	288.554969149637\\
1540	344.164034428489\\
1541	284.493491443491\\
1542	319.868283107441\\
1543	328.395670999341\\
1544	273.272968082175\\
1545	293.424815327503\\
1546	316.312108915588\\
1547	313.792538975638\\
1548	286.635080571386\\
1549	291.081487206747\\
1550	283.265058368261\\
1551	279.611790097346\\
1552	312.584695148087\\
1553	278.636188841134\\
1554	272.088397182798\\
1555	291.468815501544\\
1556	291.727246622001\\
1557	306.892860853582\\
1558	356.199449192123\\
1559	306.672541380247\\
1560	372.187308341022\\
1561	385.291359734276\\
1562	339.391792959249\\
1563	364.20470755712\\
1564	366.442032081748\\
1565	420.667201208822\\
1566	399.204300379909\\
1567	449.061277250653\\
1568	385.877190420418\\
1569	457.808288978662\\
1570	462.294790139858\\
1571	458.615569947646\\
1572	437.313369514132\\
1573	411.018442468501\\
1574	430.500755167544\\
1575	405.135286081239\\
1576	369.423212970941\\
1577	392.736609424949\\
1578	349.042350457124\\
1579	356.466149336323\\
1580	339.181806951136\\
1581	311.577925331864\\
1582	311.120993405231\\
1583	279.312134574023\\
1584	273.984609106904\\
1585	261.244524541853\\
1586	248.253684795448\\
1587	250.779758148621\\
1588	238.099490678528\\
1589	245.144009156263\\
1590	223.714424836135\\
1591	214.678532667401\\
1592	207.619767202834\\
1593	197.080593024257\\
1594	190.197979397138\\
1595	188.065368308659\\
1596	198.582895326496\\
1597	175.29971827137\\
1598	179.94125181846\\
1599	175.121386470834\\
1600	189.001308735525\\
1601	182.635286214011\\
1602	170.11108085298\\
1603	182.154129201784\\
1604	190.935065667314\\
1605	178.807202074985\\
1606	183.164069942266\\
1607	218.694683222071\\
1608	224.924538766327\\
1609	209.500413062892\\
1610	227.307131790663\\
1611	283.760514664684\\
1612	265.658129647441\\
1613	278.23988847853\\
1614	353.355475027071\\
1615	327.737389835408\\
1616	374.710886694071\\
1617	376.347490742005\\
1618	406.183279675851\\
1619	388.935069836587\\
1620	412.103721257261\\
1621	403.929753080049\\
1622	408.332959233393\\
1623	378.185568514197\\
1624	405.703575328764\\
1625	382.420457471816\\
1626	364.867365016149\\
1627	377.843655499934\\
1628	327.226334034549\\
1629	295.586824275924\\
1630	335.50885335976\\
1631	371.443587071373\\
1632	333.79382204387\\
1633	328.006068431091\\
1634	258.395392346605\\
1635	289.146818683445\\
1636	269.467787092003\\
1637	298.930333243836\\
1638	345.727521169674\\
1639	272.752511974405\\
1640	268.713512752371\\
1641	283.209341064775\\
1642	338.418973936516\\
1643	312.448403922251\\
1644	306.724620281401\\
1645	314.851396095224\\
1646	303.175499220496\\
1647	281.96068315387\\
1648	306.743095734226\\
1649	308.200369449549\\
1650	321.499454878631\\
1651	278.647741619776\\
1652	314.941522366611\\
1653	295.79883823266\\
1654	345.196308281104\\
1655	376.774830701022\\
1656	380.767514748942\\
1657	383.168001556759\\
1658	396.714748530768\\
1659	380.830238500739\\
1660	400.875223343945\\
1661	395.761763717055\\
1662	424.184998249865\\
1663	401.337373961263\\
1664	464.948250329025\\
1665	424.23156358929\\
1666	416.384809865403\\
1667	414.100849037179\\
1668	410.457901874059\\
1669	439.89471830835\\
1670	409.159927913451\\
1671	370.60784514932\\
1672	390.504551500815\\
1673	372.920829964416\\
1674	348.805084333946\\
1675	364.261036933335\\
1676	322.776222749433\\
1677	342.280931035782\\
1678	317.839542064714\\
1679	280.045679013672\\
1680	284.305779362419\\
1681	263.131863804773\\
1682	242.894168767391\\
1683	238.431222875\\
1684	238.678910493351\\
1685	245.530371872193\\
1686	211.576315086719\\
1687	203.095911290038\\
1688	222.394466456221\\
1689	200.132078275378\\
1690	177.526343109306\\
1691	181.526670699858\\
1692	172.330924360839\\
1693	182.434943933235\\
1694	195.716519251134\\
1695	173.815943633286\\
1696	173.36235673284\\
1697	168.826840681486\\
1698	181.331432547673\\
1699	185.563902984301\\
1700	194.117870946342\\
1701	200.275085977061\\
1702	207.079085263378\\
1703	218.182207840274\\
1704	230.336735362406\\
1705	236.307046298253\\
1706	250.918920446517\\
1707	282.077119822123\\
1708	318.740872169898\\
1709	308.069762899615\\
1710	293.375526825098\\
1711	383.493125806519\\
1712	387.528008878903\\
1713	364.478236855493\\
1714	398.436230069142\\
1715	430.671321600811\\
1716	415.277937005686\\
1717	391.280806908298\\
1718	383.809596779229\\
1719	339.868606026962\\
1720	334.391686460991\\
1721	327.187231373119\\
1722	383.998295962408\\
1723	365.112309220068\\
1724	292.469578839132\\
1725	340.449834964624\\
1726	341.689513181395\\
1727	296.459444264037\\
1728	318.694510927293\\
1729	306.009229820049\\
1730	336.137035599707\\
1731	260.535109992521\\
1732	257.324060546121\\
1733	244.723757834792\\
1734	225.603217586879\\
1735	237.349929974603\\
1736	269.81996226572\\
1737	235.491762109695\\
1738	238.374543288163\\
1739	229.211620274267\\
1740	249.586566609305\\
1741	233.53120514743\\
1742	271.437049251814\\
1743	255.109665175978\\
1744	282.175941157239\\
1745	302.942243346886\\
1746	264.008293945729\\
1747	276.175071056675\\
1748	286.586835476666\\
1749	303.523161581241\\
1750	310.836426229211\\
1751	323.923518675022\\
1752	357.75655496123\\
1753	337.569360392613\\
1754	370.244080351769\\
1755	370.707309731363\\
1756	412.069873706301\\
1757	367.152921202164\\
1758	399.853669538308\\
1759	423.879457442319\\
1760	423.772840834554\\
1761	381.337912803587\\
1762	424.759653676252\\
1763	411.067518146476\\
1764	404.140733939741\\
1765	383.571300943524\\
1766	398.009051360215\\
1767	389.243156331916\\
1768	364.780726400816\\
1769	373.481252818142\\
1770	364.465520625833\\
1771	321.184087472961\\
1772	336.519610279016\\
1773	317.225944340758\\
1774	299.057080599308\\
1775	272.857417947239\\
1776	266.71071634916\\
1777	248.031190050092\\
1778	268.773701804798\\
1779	252.897754146255\\
1780	250.578227926195\\
1781	230.397892349082\\
1782	232.904639123542\\
1783	197.401513646131\\
1784	219.897769995415\\
1785	189.251518179152\\
1786	174.239577330796\\
1787	182.842207718915\\
1788	162.097157654202\\
1789	194.778607659735\\
1790	184.569033558131\\
1791	167.197063579555\\
1792	171.29735670674\\
1793	165.811759692898\\
1794	180.774932732894\\
1795	178.055082698499\\
1796	178.651982712439\\
1797	186.058033545195\\
1798	194.16051538803\\
1799	213.018349184683\\
1800	213.285268039107\\
1801	247.835567752028\\
1802	261.2734566556\\
1803	283.09566616241\\
1804	296.976301483331\\
1805	302.301164124727\\
1806	319.424978298507\\
1807	365.29468077024\\
1808	398.327796715678\\
1809	419.226808138699\\
1810	411.732788924738\\
1811	386.373806534188\\
1812	361.500429994918\\
1813	381.203385832945\\
1814	362.938257111701\\
1815	338.88882527735\\
1816	324.603815317387\\
1817	304.695325937673\\
1818	279.026056233226\\
1819	266.906507850338\\
1820	258.478467336253\\
1821	254.00942762657\\
1822	241.912276747136\\
1823	312.500313426141\\
1824	277.834866042536\\
1825	229.108425120499\\
1826	251.684817431762\\
1827	262.831960206549\\
1828	269.311844833502\\
1829	293.416603248554\\
1830	238.950318824702\\
1831	294.09383843069\\
1832	266.707443466284\\
1833	229.029492973712\\
1834	256.776786660893\\
1835	288.450818823963\\
1836	221.149869216424\\
1837	241.963550660007\\
1838	281.69388877047\\
1839	291.775672742303\\
1840	270.022067172218\\
1841	268.614824229738\\
1842	284.270602607187\\
1843	239.91963907576\\
1844	277.321983723173\\
1845	300.887481744869\\
1846	329.775805086503\\
1847	297.912206394187\\
1848	327.944955254537\\
1849	358.611859112693\\
1850	395.554579755124\\
1851	367.088170465822\\
1852	390.455583646265\\
1853	412.379208363748\\
1854	406.20138114588\\
1855	408.979581741515\\
1856	407.944672900328\\
1857	400.806762350693\\
1858	406.396882027021\\
1859	390.882390580298\\
1860	414.435222786917\\
1861	409.006176243139\\
1862	417.999263228364\\
1863	359.228050604837\\
1864	358.09244519837\\
1865	359.758299507419\\
1866	334.337890958247\\
1867	311.715575115705\\
1868	299.135456055687\\
1869	290.644723121898\\
1870	271.423851475309\\
1871	280.788346704821\\
1872	262.305456932078\\
1873	259.386437791489\\
1874	231.23905538534\\
1875	249.624409989933\\
1876	225.543577331676\\
1876	105.971226236177\\
1875	112.134958419428\\
1874	123.354049292351\\
1873	141.832878590965\\
1872	154.376231262375\\
1871	162.983250551396\\
1870	164.690415676877\\
1869	170.694204875961\\
1868	182.362423311136\\
1867	201.384982712916\\
1866	227.681566293952\\
1865	259.828487445442\\
1864	256.919178446649\\
1863	250.093901491903\\
1862	300.310005363949\\
1861	303.096047995207\\
1860	308.999495945559\\
1859	272.733772380353\\
1858	295.292580204359\\
1857	289.794066908457\\
1856	300.662048451883\\
1855	306.277079025058\\
1854	305.233748580129\\
1853	311.975316688359\\
1852	289.575119674658\\
1851	252.503339652068\\
1850	288.736126489183\\
1849	257.210792616758\\
1848	221.573289240696\\
1847	182.929866059657\\
1846	225.652832660669\\
1845	198.250766114339\\
1844	168.178671162094\\
1843	125.132547608482\\
1842	180.426921016458\\
1841	164.097913677022\\
1840	170.880008020977\\
1839	188.06182496389\\
1838	169.788723315616\\
1837	116.826495053123\\
1836	109.209779958181\\
1835	169.861379613164\\
1834	136.803868634003\\
1833	118.548351685821\\
1832	159.199261539583\\
1831	175.969501242086\\
1830	131.875440115504\\
1829	171.23060575722\\
1828	158.688773916454\\
1827	157.461016185717\\
1826	139.660081885585\\
1825	122.356303248847\\
1824	158.093847119471\\
1823	198.70114737165\\
1822	138.074016089142\\
1821	149.015622695666\\
1820	148.258814868359\\
1819	150.048507177487\\
1818	165.924718359772\\
1817	189.575560497287\\
1816	223.258732691072\\
1815	224.926418371805\\
1814	248.046133893248\\
1813	266.519028399254\\
1812	254.091287675788\\
1811	257.011024020664\\
1810	300.622613337912\\
1809	301.529936410181\\
1808	267.630609961627\\
1807	252.73567202485\\
1806	205.86335095196\\
1805	180.760410255596\\
1804	168.254717145704\\
1803	154.658884786757\\
1802	140.6783035528\\
1801	133.435544014324\\
1800	93.7221771642201\\
1799	106.313433314643\\
1798	91.0474018320638\\
1797	73.5844152623938\\
1796	76.0808787784707\\
1795	72.4155478449361\\
1794	76.5295911672546\\
1793	56.5763049561832\\
1792	59.4540067939764\\
1791	62.88090542259\\
1790	77.6241789839131\\
1789	82.1527953988704\\
1788	56.3663660629744\\
1787	80.9822630997856\\
1786	70.2790967221233\\
1785	87.4892710244065\\
1784	107.75971837073\\
1783	86.8596632189523\\
1782	103.946474010103\\
1781	105.222753405039\\
1780	111.782151203152\\
1779	124.642873564826\\
1778	136.444752038909\\
1777	133.868672347612\\
1776	159.663002734341\\
1775	169.014243993729\\
1774	187.971305895366\\
1773	208.472534885159\\
1772	224.325865339242\\
1771	207.854085233776\\
1770	255.472585729103\\
1769	270.55662722005\\
1768	259.972792754072\\
1767	285.651487826052\\
1766	293.563336079423\\
1765	271.259350958001\\
1764	293.231598173696\\
1763	304.053776237203\\
1762	301.708785165855\\
1761	257.00755838007\\
1760	308.324688254954\\
1759	305.724939346826\\
1758	285.481597256151\\
1757	246.527445833344\\
1756	291.39255560519\\
1755	253.244354560552\\
1754	253.043061184798\\
1753	221.680709029182\\
1752	248.734241162279\\
1751	214.704931842446\\
1750	195.452349996385\\
1749	191.751018387426\\
1748	165.163807331608\\
1747	153.420929986427\\
1746	147.262520915061\\
1745	189.624404995938\\
1744	173.740938868488\\
1743	144.157714308448\\
1742	163.261801966353\\
1741	123.982849795512\\
1740	140.419180305483\\
1739	120.92269140579\\
1738	133.367342412909\\
1737	125.350363767269\\
1736	155.004522122408\\
1735	124.520440051733\\
1734	114.980358602375\\
1733	136.510040929145\\
1732	139.888826698988\\
1731	138.659862881274\\
1730	208.853675355005\\
1729	191.757792414635\\
1728	198.157710283525\\
1727	184.086283150946\\
1726	202.559730800904\\
1725	199.86345804089\\
1724	170.341948710756\\
1723	217.743505081206\\
1722	246.235980177791\\
1721	197.631082137504\\
1720	208.364673703388\\
1719	208.009669084363\\
1718	248.689769345034\\
1717	274.285414925565\\
1716	294.292433862459\\
1715	297.864076634078\\
1714	279.72265325873\\
1713	260.344555375426\\
1712	255.233873867385\\
1711	273.468168566459\\
1710	177.5196740577\\
1709	188.074473918445\\
1708	190.487365355339\\
1707	162.850577644686\\
1706	143.341947383705\\
1705	129.489196168725\\
1704	114.985840588274\\
1703	112.375946718294\\
1702	100.164252876739\\
1701	90.3174967170521\\
1700	85.2226399998857\\
1699	67.5192288986491\\
1698	76.5913728391971\\
1697	59.1888742916164\\
1696	64.6015285753434\\
1695	66.4205867845807\\
1694	80.718297151151\\
1693	73.7381117694802\\
1692	73.1224881867383\\
1691	77.4076020289559\\
1690	49.2886724110907\\
1689	87.4818002157103\\
1688	109.312991706903\\
1687	73.1952976234713\\
1686	78.0822455444225\\
1685	108.983440659237\\
1684	105.507834507086\\
1683	127.288630174763\\
1682	127.066399238215\\
1681	147.439174642635\\
1680	167.802997117113\\
1679	162.000063520289\\
1678	168.981633234169\\
1677	234.390680439952\\
1676	203.612135676784\\
1675	256.442044224701\\
1674	243.570786952948\\
1673	270.944184244095\\
1672	281.543737150956\\
1671	264.231844601372\\
1670	307.474176529353\\
1669	336.241657723006\\
1668	309.691618926365\\
1667	305.172544631986\\
1666	307.177520263872\\
1665	317.911758603645\\
1664	350.1697388154\\
1663	292.374237431193\\
1662	321.563350683965\\
1661	292.593623909574\\
1660	295.108465019911\\
1659	269.122541576081\\
1658	282.44603288193\\
1657	271.679858155648\\
1656	264.281147913254\\
1655	261.72715214729\\
1654	234.700077285087\\
1653	188.902493537536\\
1652	204.680922187018\\
1651	169.279534052142\\
1650	214.30260433212\\
1649	196.074802848348\\
1648	197.275159600073\\
1647	176.569140383316\\
1646	186.423389290728\\
1645	200.616696499521\\
1644	192.321353862329\\
1643	185.85032696395\\
1642	211.194435027294\\
1641	165.852212057128\\
1640	136.536829889991\\
1639	142.664221260244\\
1638	209.191186374911\\
1637	185.912190553507\\
1636	159.74773833363\\
1635	166.818224733267\\
1634	143.117152691878\\
1633	211.638643432852\\
1632	208.879191826446\\
1631	230.340059676468\\
1630	221.896855099192\\
1629	179.180652218764\\
1628	212.603971377323\\
1627	247.675161053062\\
1626	243.273913799134\\
1625	229.665775014373\\
1624	270.223705408008\\
1623	251.409334149831\\
1622	263.261324549401\\
1621	276.325056844452\\
1620	278.225289950474\\
1619	276.837672458169\\
1618	275.957150937823\\
1617	266.85381316666\\
1616	258.559583487407\\
1615	215.073722286899\\
1614	226.74781104854\\
1613	169.308358269886\\
1612	158.962473249467\\
1611	169.854728169336\\
1610	105.184240592057\\
1609	97.380424008198\\
1608	108.942884339055\\
1607	102.966489213139\\
1606	79.4715146263758\\
1605	71.8874606000045\\
1604	76.366577615255\\
1603	68.2745099255547\\
1602	61.357726446106\\
1601	79.742122322739\\
1600	77.5086581750094\\
1599	75.0705579608847\\
1598	64.6780859247491\\
1597	72.6665257514964\\
1596	85.8612619132625\\
1595	68.1478203803425\\
1594	67.8836324799733\\
1593	86.2794612052307\\
1592	88.7000434657987\\
1591	88.6422638801832\\
1590	80.9555961280473\\
1589	110.432131675339\\
1588	103.230533203193\\
1587	129.876494126554\\
1586	129.394503552838\\
1585	152.801311763125\\
1584	159.727248206239\\
1583	157.475542543768\\
1582	204.981585739555\\
1581	201.899766815988\\
1580	236.904681431871\\
1579	244.274359656268\\
1578	238.893727950386\\
1577	283.143399200965\\
1576	263.985846173261\\
1575	298.48491366714\\
1574	327.082080978892\\
1573	306.093795076442\\
1572	329.509491143057\\
1571	346.011444728891\\
1570	351.120215481136\\
1569	329.954206108589\\
1568	253.916723476353\\
1567	334.441919324827\\
1566	274.77652438235\\
1565	306.507340765558\\
1564	253.792283122573\\
1563	256.958852437061\\
1562	231.982604177492\\
1561	271.655074214551\\
1560	253.016895673926\\
1559	184.922819194955\\
1558	242.366709483776\\
1557	202.438287734426\\
1556	186.177920938157\\
1555	185.703568143706\\
1554	161.668222775136\\
1553	168.231442111311\\
1552	193.193926279943\\
1551	167.573117118097\\
1550	169.472310862297\\
1549	168.248205115507\\
1548	172.706794353378\\
1547	195.403393242964\\
1546	194.169584436805\\
1545	157.457526487222\\
1544	156.346422798882\\
1543	206.094022841232\\
1542	196.532651965979\\
1541	161.59850477593\\
1540	217.60842422777\\
1539	179.7710542951\\
1538	185.698255452851\\
1537	178.229766848775\\
1536	204.801191318966\\
1535	211.396894319601\\
1534	165.681065445449\\
1533	220.410480242218\\
1532	199.837641905526\\
1531	172.746667060323\\
1530	223.615410767544\\
1529	214.007650374738\\
1528	229.169254132982\\
1527	228.72517415346\\
1526	241.057735689667\\
1525	232.643457611857\\
1524	226.38631885267\\
1523	231.391596613682\\
1522	270.069958558344\\
1521	237.244052262346\\
1520	229.073752607874\\
1519	241.687582499986\\
1518	181.045616385803\\
1517	166.63685534005\\
1516	173.239911390945\\
1515	122.535677848879\\
1514	104.624406754375\\
1513	70.9790555614865\\
1512	99.0027912267135\\
1511	73.4396379326243\\
1510	71.8477548319344\\
1509	68.2901106363267\\
1508	63.9521478967094\\
1507	59.6716476564255\\
1506	65.9769744656597\\
1505	66.8325532388212\\
1504	55.8078919163653\\
1503	64.2512935933695\\
1502	65.65888571063\\
1501	100.149727367622\\
1500	69.6033399740791\\
1499	78.9276122903607\\
1498	84.4823235317428\\
1497	108.822343852857\\
1496	75.3008225571745\\
1495	77.0934304493229\\
1494	94.5373425699676\\
1493	95.1298126032445\\
1492	115.890401212277\\
1491	111.317922637622\\
1490	130.829977605665\\
1489	143.044953460349\\
1488	135.54475021569\\
1487	126.682321468395\\
1486	167.626050842563\\
1485	169.376204344175\\
1484	207.826149904972\\
1483	231.711453249033\\
1482	238.009962913468\\
1481	246.907893036506\\
1480	280.924720370395\\
1479	251.225523904897\\
1478	240.800320045361\\
1477	251.195815290233\\
1476	254.273706316125\\
1475	319.472012704587\\
1474	326.180819923854\\
1473	315.151963016321\\
1472	323.093492654344\\
1471	253.177940931253\\
1470	313.649549239896\\
1469	293.272089827726\\
1468	261.58049908044\\
1467	268.256123181732\\
1466	242.366647260568\\
1465	249.870866921693\\
1464	248.58329588221\\
1463	243.878547710826\\
1462	215.125397775792\\
1461	162.767009762503\\
1460	212.821042197067\\
1459	209.474703388887\\
1458	206.542682538478\\
1457	191.777741958228\\
1456	163.080213485479\\
1455	191.221051106127\\
1454	145.90295569803\\
1453	152.803179169561\\
1452	184.859720595701\\
1451	171.712208194535\\
1450	191.306120981005\\
1449	150.405919174631\\
1448	197.344321094637\\
1447	172.839222336665\\
1446	136.133211037096\\
1445	182.730989334505\\
1444	158.17927469802\\
1443	171.205015947419\\
1442	208.590697320413\\
1441	169.325987598041\\
1440	170.977024888211\\
1439	189.505373293587\\
1438	167.417303923597\\
1437	150.50761835372\\
1436	145.561042489232\\
1435	176.993078030668\\
1434	165.388968847542\\
1433	193.381378241338\\
1432	209.063225885818\\
1431	189.422538113273\\
1430	217.462224127136\\
1429	225.263422653946\\
1428	224.145130218167\\
1427	219.372694955223\\
1426	258.049401988436\\
1425	260.789526645327\\
1424	238.410825558045\\
1423	235.500502509902\\
1422	203.881764299948\\
1421	165.770388096409\\
1420	129.508900687757\\
1419	143.318495215871\\
1418	114.578314285182\\
1417	100.511217489319\\
1416	70.641277855567\\
1415	53.0849971393382\\
1414	75.309648573476\\
1413	46.7378235783942\\
1412	51.2047358125469\\
1411	47.6953887024089\\
1410	43.5425564321203\\
1409	47.3914817341776\\
1408	41.2064508122245\\
1407	34.0239313761127\\
1406	53.6824303768761\\
1405	36.6581075791115\\
1404	44.8456751360561\\
1403	40.0363408217192\\
1402	52.8724962227966\\
1401	36.7728011862366\\
1400	55.6295190037353\\
1399	43.3692353561677\\
1398	58.746655576835\\
1397	70.0458391298741\\
1396	85.2687196538266\\
1395	91.4930164817288\\
1394	99.0619290287862\\
1393	112.538196837288\\
1392	118.845005368522\\
1391	137.372422695175\\
1390	145.026578177787\\
1389	163.545094591096\\
1388	187.881968478255\\
1387	198.857979623974\\
1386	214.9354792536\\
1385	230.25814813011\\
1384	209.128113896342\\
1383	222.288101034753\\
1382	222.387000108099\\
1381	234.152652150572\\
1380	291.624496713439\\
1379	268.764688292053\\
1378	299.299880300438\\
1377	297.189101075139\\
1376	303.436453162745\\
1375	288.47978305584\\
1374	287.040828856958\\
1373	273.658619749368\\
1372	243.49700637515\\
1371	235.909425126572\\
1370	227.94796030756\\
1369	213.162710873649\\
1368	192.240515651927\\
1367	190.595051525039\\
1366	162.244973986181\\
1365	216.29967004577\\
1364	192.547040082608\\
1363	183.070650836506\\
1362	128.229026900538\\
1361	140.442867564406\\
1360	185.348321718745\\
1359	119.496833770974\\
1358	163.14639028093\\
1357	129.066248924406\\
1356	119.518667508204\\
1355	124.683603040322\\
1354	172.395088453916\\
1353	148.744775043729\\
1352	150.270089558447\\
1351	152.676307620331\\
1350	154.788515729587\\
1349	131.87623781105\\
1348	133.29636958555\\
1347	126.52216971566\\
1346	167.891070060928\\
1345	159.574873639988\\
1344	141.071389412456\\
1343	166.083985542071\\
1342	190.973514511323\\
1341	170.506841921417\\
1340	186.537387149038\\
1339	158.47439376309\\
1338	154.803873479121\\
1337	219.342502488685\\
1336	219.556087794814\\
1335	185.499350928729\\
1334	222.825321471578\\
1333	213.309628466437\\
1332	235.924860968581\\
1331	243.91818137917\\
1330	248.51632027674\\
1329	242.111855858992\\
1328	240.831763644618\\
1327	215.287239923036\\
1326	199.73792448366\\
1325	161.939107847914\\
1324	146.058400250083\\
1323	148.053266672771\\
1322	90.4584832199822\\
1321	106.120934420568\\
1320	102.437856807195\\
1319	62.4992794620674\\
1318	66.7546632644835\\
1317	56.3347635569127\\
1316	57.7750252781593\\
1315	53.9897863670871\\
1314	50.2852459815788\\
1313	44.5759389387189\\
1312	40.7453584835418\\
1311	41.9034441945506\\
1310	51.4219136382219\\
1309	28.4730609970333\\
1308	35.4022525596105\\
1307	42.3483173113375\\
1306	43.0885450851038\\
1305	33.9808209030101\\
1304	39.0921050421863\\
1303	38.9668395421327\\
1302	52.3602204034937\\
1301	63.5706381885782\\
1300	70.4148844027566\\
1299	79.3576801371184\\
1298	85.1007321190936\\
1297	98.0339992795878\\
1296	107.45777498082\\
1295	110.030916794053\\
1294	118.849661411188\\
1293	142.808909353493\\
1292	151.65821539313\\
1291	172.627279568033\\
1290	172.813625379658\\
1289	170.855551128898\\
1288	195.77911330556\\
1287	224.003698963428\\
1286	204.933056097215\\
1285	223.441573576037\\
1284	250.462236393258\\
1283	261.057037169915\\
1282	267.353531477194\\
1281	265.513760315078\\
1280	241.275759826123\\
1279	227.744856701107\\
1278	266.433059885573\\
1277	258.981234256075\\
1276	267.632228209739\\
1275	224.33657368158\\
1274	224.299923803186\\
1273	218.457968283338\\
1272	187.133609031626\\
1271	171.589334579353\\
1270	198.191846298455\\
1269	150.039991635079\\
1268	125.033891187868\\
1267	174.085071390252\\
1266	168.876746693693\\
1265	161.869267309036\\
1264	130.811267810685\\
1263	135.784621124168\\
1262	86.1316761518183\\
1261	149.536163608087\\
1260	147.319369525985\\
1259	149.203381159851\\
1258	159.523751964204\\
1257	149.508676586476\\
1256	99.709988854631\\
1255	135.529935209344\\
1254	89.6101872329642\\
1253	92.6393318889999\\
1252	137.017278276989\\
1251	130.452541624541\\
1250	114.709683546456\\
1249	138.842888450493\\
1248	125.938201954195\\
1247	133.067383265459\\
1246	137.426408253951\\
1245	100.015168487598\\
1244	119.931726700289\\
1243	150.217197919648\\
1242	157.231764583348\\
1241	155.09474391866\\
1240	172.563791887365\\
1239	193.595161973823\\
1238	200.82897110958\\
1237	197.096115019831\\
1236	215.301676391054\\
1235	230.673010151385\\
1234	233.964736293816\\
1233	234.366574320413\\
1232	229.35006564221\\
1231	227.966701737081\\
1230	189.312626879447\\
1229	162.41068638539\\
1228	156.130983176436\\
1227	132.837829834231\\
1226	125.294313779497\\
1225	103.090176108412\\
1224	91.4566997537468\\
1223	84.2010667757365\\
1222	68.3357615210915\\
1221	65.2652242994423\\
1220	51.0031768650542\\
1219	53.7060528785708\\
1218	56.9933655102323\\
1217	53.233679721916\\
1216	42.0730399564429\\
1215	32.6713397767144\\
1214	45.5446477624776\\
1213	40.5368884773369\\
1212	42.403850901839\\
1211	35.8899174206944\\
1210	31.0315552804468\\
1209	44.403306336306\\
1208	51.2775706111999\\
1207	47.321395468128\\
1206	47.5586561229614\\
1205	71.0976453034526\\
1204	81.8222356588182\\
1203	82.6703575120307\\
1202	89.0158642281425\\
1201	100.800157669235\\
1200	117.426015304952\\
1199	124.96665726223\\
1198	127.048712106523\\
1197	147.443558447583\\
1196	148.996501343347\\
1195	168.681085905268\\
1194	170.903142089844\\
1193	196.257064673898\\
1192	213.524910540713\\
1191	218.361331832238\\
1190	227.919007398793\\
1189	209.322744725318\\
1188	234.535716983479\\
1187	250.10185070999\\
1186	251.080859633284\\
1185	253.482853470032\\
1184	247.579802017185\\
1183	246.820031888498\\
1182	222.996812235822\\
1181	205.780767774133\\
1180	212.908084369707\\
1179	200.937315495791\\
1178	198.365026203237\\
1177	183.586882406105\\
1176	179.426134551436\\
1175	155.64146044302\\
1174	150.272817879452\\
1173	152.549838372692\\
1172	147.588334896552\\
1171	122.182963180528\\
1170	100.1861859875\\
1169	140.897393093469\\
1168	128.414443244297\\
1167	119.373549625907\\
1166	108.388358940535\\
1165	84.8918987648475\\
1164	96.2009200786588\\
1163	88.059286178645\\
1162	89.9481548998901\\
1161	121.750903890096\\
1160	101.226259323928\\
1159	93.7718664036191\\
1158	105.595488114309\\
1157	82.0310419967169\\
1156	71.1306857335921\\
1155	72.9669228535211\\
1154	81.6020991158499\\
1153	44.0547444352463\\
1152	78.5699765917063\\
1151	78.1753345400849\\
1150	70.9716154359444\\
1149	84.2714114564569\\
1148	81.9129344760898\\
1147	82.2098875794903\\
1146	116.031734980104\\
1145	122.990915638696\\
1144	138.69423922691\\
1143	141.363886620737\\
1142	152.553634409564\\
1141	161.799523005035\\
1140	169.86556602207\\
1139	193.544798323491\\
1138	189.254879098757\\
1137	194.750509051624\\
1136	215.747646052349\\
1135	179.1072115157\\
1134	153.776713914336\\
1133	148.993909824704\\
1132	156.359923458277\\
1131	158.043910574049\\
1130	94.9400204589747\\
1129	89.6636053031872\\
1128	104.523136990314\\
1127	96.2366164483394\\
1126	87.7940769424659\\
1125	74.3284526607223\\
1124	64.9581723044445\\
1123	60.7733247821532\\
1122	58.0556503137263\\
1121	47.1298587177098\\
1120	51.5945998755659\\
1119	54.4210229188177\\
1118	60.3320110856455\\
1117	38.2651482408022\\
1116	37.7450576423473\\
1115	52.6977880369467\\
1114	61.4018072795355\\
1113	64.4909188678489\\
1112	74.944042537517\\
1111	49.5890696032001\\
1110	54.2848460494041\\
1109	78.7070509697899\\
1108	87.8803132923239\\
1107	94.9522210939085\\
1106	101.902225860589\\
1105	116.174383851857\\
1104	128.335074158643\\
1103	140.373939150463\\
1102	157.004432562112\\
1101	177.991202254828\\
1100	183.325580936162\\
1099	186.604859884678\\
1098	224.627044800283\\
1097	229.240733417613\\
1096	221.16753989897\\
1095	217.575170323624\\
1094	256.020102984273\\
1093	258.186403577801\\
1092	269.691451266269\\
1091	287.855357579711\\
1090	271.027858184871\\
1089	289.724027672411\\
1088	299.490616707189\\
1087	259.24594759687\\
1086	293.828621310427\\
1085	296.475442467039\\
1084	231.534832114719\\
1083	254.925587630152\\
1082	237.938509219973\\
1081	230.719060929148\\
1080	230.60446934777\\
1079	209.960290035124\\
1078	222.057166929921\\
1077	209.297453192735\\
1076	145.615605924656\\
1075	167.890313058834\\
1074	178.878888069296\\
1073	137.668776112999\\
1072	124.760032533574\\
1071	145.996202212563\\
1070	127.883765226646\\
1069	147.990539990281\\
1068	141.707233751809\\
1067	152.928481556661\\
1066	112.31358162469\\
1065	120.558822721073\\
1064	170.595372710871\\
1063	155.524029554312\\
1062	108.518612972765\\
1061	153.162079183844\\
1060	172.06246096378\\
1059	175.049611700847\\
1058	152.774324351477\\
1057	128.088022221801\\
1056	141.272919649553\\
1055	168.878350903612\\
1054	159.690338832697\\
1053	163.142996205102\\
1052	199.532724401512\\
1051	217.756070741431\\
1050	217.124694674306\\
1049	196.089374738106\\
1048	227.853746755284\\
1047	232.973707636513\\
1046	261.575466674706\\
1045	254.72849313825\\
1044	254.002943249147\\
1043	275.605898108279\\
1042	264.670076602187\\
1041	231.961740908477\\
1040	236.406561281041\\
1039	254.57931109686\\
1038	180.089468387083\\
1037	173.710571684978\\
1036	193.86229858433\\
1035	144.904397343903\\
1034	136.461338825453\\
1033	119.646406517761\\
1032	99.9135994888589\\
1031	93.8314489738757\\
1030	94.4102615389903\\
1029	80.1392851201287\\
1028	77.601373306156\\
1027	76.5070256961405\\
1026	66.2062309036791\\
1025	71.6500400873106\\
1024	52.0066507244411\\
1023	82.4167403564492\\
1022	71.6278282000802\\
1021	49.9627748958529\\
1020	86.6675525821165\\
1019	102.052880487445\\
1018	82.3867285503701\\
1017	87.9715732623513\\
1016	71.0905132829355\\
1015	93.2716239450044\\
1014	87.8614899717195\\
1013	106.705102073278\\
1012	108.038183619703\\
1011	132.860783023939\\
1010	139.858932206957\\
1009	128.451846342621\\
1008	176.453589170319\\
1007	184.885989199618\\
1006	210.000439698322\\
1005	197.357917531781\\
1004	215.173569154938\\
1003	211.527117821804\\
1002	269.917455996826\\
1001	266.043482769215\\
1000	273.982929044098\\
999	310.822796750236\\
998	305.431515396742\\
997	290.612104604083\\
996	273.88860322353\\
995	279.8046830582\\
994	292.795366101129\\
993	293.256779112407\\
992	265.891434528255\\
991	320.156589012087\\
990	275.073963783059\\
989	308.730025182806\\
988	281.714096033737\\
987	265.968322679618\\
986	275.201946269484\\
985	243.532398350502\\
984	237.323698298366\\
983	234.966994605319\\
982	227.801430958682\\
981	214.588672577828\\
980	208.429734756813\\
979	151.632562256781\\
978	177.519750914363\\
977	203.866992166699\\
976	187.041089842556\\
975	142.726382491078\\
974	174.878798825108\\
973	149.081883996355\\
972	163.747746754166\\
971	189.179716416117\\
970	155.061930307944\\
969	133.734222218523\\
968	155.922199994944\\
967	192.035350296808\\
966	155.252986131991\\
965	198.100859849751\\
964	182.793111180918\\
963	177.202225747664\\
962	203.601469782004\\
961	213.788961080165\\
960	187.047079492591\\
959	215.661222679481\\
958	199.935707956596\\
957	212.43164027886\\
956	235.939936865795\\
955	234.16529902317\\
954	210.301495939902\\
953	220.747433509676\\
952	208.541303460302\\
951	256.026457380404\\
950	271.224346570654\\
949	261.434054663096\\
948	263.307614758962\\
947	242.378043199189\\
946	225.979637610812\\
945	236.706389956202\\
944	261.724409173213\\
943	250.787314241778\\
942	194.349655468552\\
941	164.668946769507\\
940	181.450348377046\\
939	146.808516905493\\
938	113.80371428896\\
937	101.544065088823\\
936	92.4548779391002\\
935	95.9088039415027\\
934	93.1065292773014\\
933	77.6666162488787\\
932	80.2138341877538\\
931	85.2243945961006\\
930	71.1350295014363\\
929	85.3092329497114\\
928	89.8152922664882\\
927	65.2723868796184\\
926	109.380855318274\\
925	104.160359835784\\
924	78.2622547467755\\
923	102.745599146359\\
922	70.0401757279402\\
921	77.0223758017981\\
920	123.746853960908\\
919	74.5606487399613\\
918	112.281245078239\\
917	117.763673020953\\
916	126.931047217442\\
915	140.819365953265\\
914	147.437938420932\\
913	166.123070567459\\
912	176.954242689182\\
911	171.009176390241\\
910	197.462117906183\\
909	226.909304904994\\
908	203.992977231193\\
907	228.716101842835\\
906	269.188957200314\\
905	282.378091974371\\
904	301.690988781221\\
903	286.296585841682\\
902	305.158502875338\\
901	311.732064014998\\
900	280.256897307038\\
899	284.735130171461\\
898	288.027292850717\\
897	296.187237043469\\
896	268.08912968932\\
895	328.556601670718\\
894	257.659663158238\\
893	241.384444192782\\
892	307.128420975091\\
891	256.981396128918\\
890	268.81087778682\\
889	260.354297645212\\
888	241.442271653353\\
887	184.403711118248\\
886	229.771180288521\\
885	217.742531619362\\
884	193.308789340278\\
883	229.024330442661\\
882	227.22405657239\\
881	191.029004322004\\
880	156.543678131068\\
879	142.894469354114\\
878	202.138420470792\\
877	196.138838628804\\
876	190.889255291273\\
875	183.600679639975\\
874	150.791523138997\\
873	208.388479215163\\
872	209.408079664906\\
871	208.927611877076\\
870	214.131706348714\\
869	196.669200577347\\
868	218.399422225981\\
867	182.676832862537\\
866	181.152050972557\\
865	160.760076441971\\
864	231.293649864371\\
863	198.73090024919\\
862	183.050631962601\\
861	228.365010935235\\
860	232.202387712885\\
859	206.021372035928\\
858	205.066897130722\\
857	251.364980039621\\
856	244.893583181354\\
855	237.077691104127\\
854	260.836382733657\\
853	284.239680557867\\
852	290.586973546774\\
851	251.12251741799\\
850	268.10799552066\\
849	285.273986877688\\
848	273.611028199899\\
847	290.687584284068\\
846	243.303113408675\\
845	199.202177956627\\
844	175.931050464784\\
843	204.257118612279\\
842	109.214847067905\\
841	126.058035700614\\
840	115.469223391832\\
839	102.313741953967\\
838	107.648071718508\\
837	99.0044061762076\\
836	103.058790225831\\
835	97.2984057812927\\
834	88.4004316167555\\
833	115.838950341801\\
832	82.2441916287512\\
831	113.353327782232\\
830	118.726917830136\\
829	111.299929228634\\
828	86.9116237593346\\
827	145.343933187636\\
826	128.935552929359\\
825	102.781065835169\\
824	120.917474274584\\
823	109.482311870647\\
822	104.277508328497\\
821	134.321908665853\\
820	149.992045660695\\
819	147.178908989559\\
818	155.055108397313\\
817	187.763096283554\\
816	203.439565752612\\
815	215.63545994926\\
814	242.83997058048\\
813	212.747874146464\\
812	261.387318212154\\
811	238.848425614004\\
810	239.040461682654\\
809	310.245828682476\\
808	259.462541086633\\
807	321.548740289753\\
806	289.593170172176\\
805	318.38560401849\\
804	321.71233147363\\
803	359.332507551535\\
802	286.253408137849\\
801	296.356700322103\\
800	361.0999344493\\
799	271.660058079115\\
798	306.036104896421\\
797	274.063261521589\\
796	289.34845842884\\
795	240.637879800203\\
794	280.162051068875\\
793	261.485275602087\\
792	264.942020744623\\
791	229.851111266171\\
790	237.873140471363\\
789	172.108883780761\\
788	183.346251419944\\
787	179.201374800195\\
786	212.236432669574\\
785	155.870873119026\\
784	189.661856281832\\
783	193.940307684148\\
782	198.293931588818\\
781	156.566649595591\\
780	201.728022588838\\
779	164.352370510159\\
778	197.652444829316\\
777	174.691658469817\\
776	142.057609500789\\
775	167.44457673714\\
774	177.968845075013\\
773	174.929073285124\\
772	160.114140633975\\
771	110.425100323623\\
770	122.850764258481\\
769	163.27384520529\\
768	171.964499746198\\
767	150.418528696518\\
766	152.312809875337\\
765	166.900514214962\\
764	159.497751004219\\
763	165.842442864634\\
762	177.280085354699\\
761	175.413899591881\\
760	202.051625420346\\
759	230.620159717966\\
758	227.225970626457\\
757	227.249539507478\\
756	225.371331823082\\
755	232.537186562117\\
754	247.497984331031\\
753	256.010571021484\\
752	249.143550835494\\
751	226.330363561675\\
750	199.336713096422\\
749	177.276435838263\\
748	175.937395310461\\
747	146.804905176329\\
746	108.138718164792\\
745	111.325326827129\\
744	86.5842312565764\\
743	97.5812588057913\\
742	87.7911056303189\\
741	79.6445316363842\\
740	76.3771842584739\\
739	63.2501534741297\\
738	63.5545516668799\\
737	67.1449212897454\\
736	63.5172362319568\\
735	72.3092339696524\\
734	74.4176624476389\\
733	65.9951407040926\\
732	65.9222890390002\\
731	69.9232089015004\\
730	63.4376972105893\\
729	78.3182823535508\\
728	73.3604572787198\\
727	53.8919534495881\\
726	66.6250786285468\\
725	80.8430132900723\\
724	91.6678008238202\\
723	99.8762699536315\\
722	110.791696763422\\
721	118.064990696285\\
720	130.906662865838\\
719	132.235714786158\\
718	138.203451189579\\
717	165.860710768239\\
716	174.376298266122\\
715	173.040157119722\\
714	187.915926755125\\
713	216.13526964972\\
712	226.845956003935\\
711	227.245734955831\\
710	249.198460372718\\
709	221.523524809501\\
708	247.506861380301\\
707	269.234003224713\\
706	247.7243568284\\
705	273.420945789712\\
704	238.049682982042\\
703	270.735667727244\\
702	239.024147462645\\
701	251.943172681416\\
700	226.415451758925\\
699	248.424782157221\\
698	212.800641148556\\
697	232.205713503858\\
696	207.728917417486\\
695	164.742152511665\\
694	174.249845031643\\
693	158.148189993891\\
692	177.975439591258\\
691	126.089524604993\\
690	131.152077468029\\
689	142.487845325717\\
688	150.747119092655\\
687	140.594818831312\\
686	145.452854371252\\
685	133.577536213201\\
684	136.393257473854\\
683	131.491354190058\\
682	100.456380342624\\
681	100.878713327407\\
680	106.591353235752\\
679	94.3947425239714\\
678	115.454120790877\\
677	96.6274180908831\\
676	110.225063892261\\
675	63.966850931963\\
674	96.4316420183408\\
673	83.5249304076726\\
672	99.7925136865337\\
671	86.5642757556184\\
670	83.2699953786184\\
669	98.3972468571534\\
668	135.756642194382\\
667	128.059078347797\\
666	144.767790232769\\
665	172.350338034705\\
664	177.107363327422\\
663	179.273695605545\\
662	187.487861313325\\
661	200.675720545286\\
660	215.229394545178\\
659	211.614485015643\\
658	255.680238970259\\
657	228.522074837082\\
656	257.548158021138\\
655	231.00405764055\\
654	179.309481215846\\
653	183.758066873368\\
652	188.222358015745\\
651	153.621193707742\\
650	137.436713289116\\
649	106.362043318396\\
648	104.225063189575\\
647	101.655417381245\\
646	90.4893653888959\\
645	88.3504314057679\\
644	72.2810010567093\\
643	65.5815555330739\\
642	62.9507604205098\\
641	77.0196407229448\\
640	51.1889067551554\\
639	63.3997226939398\\
638	49.6700819520392\\
637	53.4580792726923\\
636	63.2878789524895\\
635	86.9429764596978\\
634	49.438405338641\\
633	71.6947789593985\\
632	87.8113096853728\\
631	86.543827360126\\
630	92.6853116241713\\
629	84.1874534561192\\
628	94.2403915449067\\
627	99.7744309431979\\
626	115.853770146922\\
625	119.907946929419\\
624	126.387672687266\\
623	154.280563353867\\
622	168.272974102313\\
621	177.984777615034\\
620	168.782141150961\\
619	187.063451940281\\
618	222.112721537606\\
617	212.045492065157\\
616	231.962764681078\\
615	259.727858089321\\
614	266.547724138057\\
613	270.690293477719\\
612	265.036588427812\\
611	277.219974264181\\
610	262.165805214572\\
609	295.732180908065\\
608	293.484074091645\\
607	278.459243187441\\
606	286.092796533639\\
605	254.118592412698\\
604	248.505634115135\\
603	256.522718924071\\
602	200.897787798042\\
601	225.476161490388\\
600	198.678383845817\\
599	204.381950177367\\
598	166.847224922243\\
597	200.886936394695\\
596	160.312116123804\\
595	150.526839621194\\
594	170.681818133844\\
593	166.750834424133\\
592	178.60314086308\\
591	171.270811610818\\
590	139.966450791448\\
589	147.73728296274\\
588	134.11843214735\\
587	149.755789070953\\
586	137.794812672275\\
585	134.088386312549\\
584	138.524589940166\\
583	108.024359368047\\
582	120.651810127571\\
581	161.147670409372\\
580	116.137950339444\\
579	165.912140995141\\
578	136.057772778377\\
577	123.384585929296\\
576	153.17355156195\\
575	113.208823308971\\
574	101.106846601125\\
573	128.667772301929\\
572	146.419866925748\\
571	152.829427086455\\
570	197.246381652569\\
569	232.76323304706\\
568	234.378280752369\\
567	244.565894169957\\
566	258.031795553333\\
565	263.461892082327\\
564	257.658796552992\\
563	248.518525445665\\
562	269.825603248276\\
561	286.524818403656\\
560	285.29057835224\\
559	267.719197922111\\
558	258.572654620239\\
557	181.085995705815\\
556	205.615704322863\\
555	207.174000193176\\
554	112.196562456553\\
553	129.809632897078\\
552	100.954850860733\\
551	107.641796684463\\
550	98.7767536832884\\
549	87.2432380731064\\
548	91.3581439611197\\
547	98.8962104969935\\
546	86.0891579893319\\
545	87.6256774029651\\
544	98.6262134735475\\
543	78.6800940616899\\
542	122.063886590835\\
541	99.2472790695963\\
540	125.581826277755\\
539	110.392978632036\\
538	114.838954528256\\
537	109.596999300749\\
536	114.490643345289\\
535	115.306250589922\\
534	121.180512924857\\
533	107.987494082432\\
532	143.280784527072\\
531	152.168931977169\\
530	144.933021172764\\
529	177.191353819607\\
528	198.889838368804\\
527	191.189571059512\\
526	187.94598381063\\
525	250.517044154382\\
524	233.620480485773\\
523	279.928320632689\\
522	275.871572041649\\
521	295.66348749523\\
520	247.851016402912\\
519	311.160680474751\\
518	283.498444254389\\
517	275.5281445717\\
516	328.110693448556\\
515	258.297212437717\\
514	334.289136946724\\
513	278.463541819237\\
512	286.220569498037\\
511	278.008567506089\\
510	245.143036803742\\
509	252.257306242532\\
508	273.445909371277\\
507	248.519231405759\\
506	269.023866292813\\
505	268.38999425021\\
504	222.240453305252\\
503	229.477633655037\\
502	233.785801019489\\
501	172.257670196133\\
500	179.091111238975\\
499	193.360451959841\\
498	173.287011808158\\
497	190.469786403423\\
496	161.184364800848\\
495	155.624820534158\\
494	154.53362545003\\
493	177.245270869976\\
492	185.184974889933\\
491	188.72184340656\\
490	180.63730126555\\
489	201.765022888124\\
488	206.804986210756\\
487	181.914897235449\\
486	145.771945971522\\
485	141.294327411622\\
484	167.061259826506\\
483	208.321836411159\\
482	179.509208720252\\
481	153.011072839677\\
480	154.995713890037\\
479	171.471223446454\\
478	211.789950187011\\
477	181.060963411541\\
476	228.298409991106\\
475	213.525688877316\\
474	236.461143032483\\
473	213.052726908255\\
472	194.054141533382\\
471	241.794067221196\\
470	240.550180101963\\
469	250.880443432239\\
468	266.095057155549\\
467	262.693489385193\\
466	276.471211290431\\
465	270.608233789757\\
464	260.698471522134\\
463	269.33312276603\\
462	199.982822966027\\
461	171.427421106474\\
460	189.938080804609\\
459	170.849471826819\\
458	101.935408579236\\
457	111.074154013828\\
456	97.3772406551881\\
455	86.4638129380483\\
454	89.8986590847877\\
453	86.7018780472427\\
452	81.4999125596784\\
451	80.4778186621188\\
450	80.3277304921352\\
449	83.7062107524843\\
448	91.1699062744675\\
447	86.6403993058484\\
446	117.541338507561\\
445	77.9335216757946\\
444	109.708810074073\\
443	122.000502549102\\
442	89.1671009790753\\
441	109.285417657093\\
440	96.8399460637606\\
439	91.6227573418841\\
438	103.633467722748\\
437	124.028642951364\\
436	109.371438469557\\
435	143.363732279804\\
434	162.821295877883\\
433	150.146601354944\\
432	158.193614451756\\
431	171.406357308618\\
430	177.138860669709\\
429	211.867411963844\\
428	206.103151187623\\
427	240.979354512967\\
426	261.022143184723\\
425	229.073168617489\\
424	273.932730292124\\
423	260.892720792224\\
422	240.957918320564\\
421	310.747518857098\\
420	320.978857848021\\
419	332.000678277478\\
418	260.983427054498\\
417	285.076811980809\\
416	271.150181197115\\
415	325.002309200568\\
414	302.395535354566\\
413	309.722220315617\\
412	257.173226969411\\
411	272.523703847586\\
410	263.355482940936\\
409	259.443229903128\\
408	246.450464116749\\
407	218.097920859962\\
406	238.211238897095\\
405	191.078080412161\\
404	226.507508956783\\
403	193.346987354387\\
402	212.511591000384\\
401	173.808656373839\\
400	162.472301484652\\
399	195.310581712688\\
398	205.247767811577\\
397	156.15414361469\\
396	145.058389811652\\
395	180.724955364181\\
394	174.004821764985\\
393	200.392643828455\\
392	153.656615811372\\
391	139.998120788172\\
390	159.069961541\\
389	158.47634547954\\
388	187.449352699681\\
387	205.384076163378\\
386	218.999514700499\\
385	194.676421439917\\
384	192.19815329622\\
383	213.9055987394\\
382	198.664113327795\\
381	204.272611835435\\
380	223.400620923292\\
379	194.556103128058\\
378	223.042013323735\\
377	244.749206635364\\
376	257.812039569162\\
375	237.281089544687\\
374	238.314364890578\\
373	266.649348039355\\
372	268.049451677626\\
371	278.394188647921\\
370	268.703502301366\\
369	253.081183728471\\
368	266.625212370174\\
367	252.899522657239\\
366	172.613846174939\\
365	175.708522256417\\
364	177.838382056684\\
363	135.22852287231\\
362	123.935239259965\\
361	122.402878304249\\
360	90.6851366569849\\
359	93.2666456095168\\
358	87.1790363738284\\
357	89.6516293399394\\
356	85.4632960311818\\
355	86.7818698483903\\
354	72.9378576595679\\
353	99.892618154334\\
352	99.0012307111503\\
351	78.3021855821331\\
350	108.66623077123\\
349	110.586129117599\\
348	78.1391780743225\\
347	141.067317829103\\
346	118.618989521707\\
345	120.009349778409\\
344	105.078824624759\\
343	113.240344835256\\
342	114.747195187652\\
341	127.22977639217\\
340	139.704030779601\\
339	136.533882466304\\
338	149.505157093329\\
337	187.700642066067\\
336	195.643978134323\\
335	181.295718053737\\
334	229.834841688842\\
333	240.226725284897\\
332	236.806867807911\\
331	223.477942294315\\
330	242.897165921307\\
329	301.737608602454\\
328	262.496821579319\\
327	296.53452080727\\
326	326.63292785599\\
325	300.610925091425\\
324	339.875052107987\\
323	294.342376373995\\
322	334.781713886394\\
321	288.634932681223\\
320	298.221229735396\\
319	285.296114990105\\
318	323.11217907262\\
317	275.019449236382\\
316	253.769313029328\\
315	252.511049333776\\
314	275.566445798894\\
313	256.665427738244\\
312	256.573146533601\\
311	196.694700784028\\
310	231.037121721691\\
309	160.358528199105\\
308	217.554925258109\\
307	151.440720510492\\
306	201.230372572365\\
305	176.001338643849\\
304	162.13434396433\\
303	140.984142509991\\
302	210.982642426803\\
301	203.611552882254\\
300	159.002495739125\\
299	163.875664623815\\
298	184.441762295219\\
297	152.355901456443\\
296	143.44098085328\\
295	186.716328550705\\
294	173.936948518865\\
293	184.797792067856\\
292	176.679817778166\\
291	187.619470943873\\
290	171.031389866464\\
289	195.056531953279\\
288	184.697798412697\\
287	142.132826521128\\
286	157.404305953713\\
285	195.343933230725\\
284	185.828019313406\\
283	173.503824331242\\
282	212.826952748732\\
281	230.219220674144\\
280	203.442452529151\\
279	234.19361344777\\
278	259.305314924485\\
277	277.366809019223\\
276	252.640391327298\\
275	252.625695717293\\
274	267.940899999392\\
273	234.278768188278\\
272	227.354386355533\\
271	219.94032399083\\
270	221.846277253717\\
269	161.321283159186\\
268	201.435477176042\\
267	132.344010294295\\
266	132.09660929965\\
265	119.622694337144\\
264	96.2586388501733\\
263	94.2483466530841\\
262	89.9215738204728\\
261	83.927377858672\\
260	84.8190043142924\\
259	71.7203110088011\\
258	80.1567812584356\\
257	80.591788760155\\
256	71.9757394311806\\
255	59.26027980828\\
254	88.1099572076363\\
253	69.719853203183\\
252	88.0075306784744\\
251	99.7719189582734\\
250	62.766436507639\\
249	81.7817392195761\\
248	108.26499108308\\
247	103.803424626321\\
246	73.6817997264452\\
245	103.640729931318\\
244	116.593786650145\\
243	122.456151805454\\
242	138.849131922802\\
241	144.947467921727\\
240	155.399189434392\\
239	163.76423009222\\
238	181.150842443064\\
237	202.408839266372\\
236	221.189870909616\\
235	205.415278781868\\
234	222.737166266978\\
233	250.211293344939\\
232	283.227405266592\\
231	301.174136251471\\
230	258.182505391813\\
229	321.784093461688\\
228	304.565176098987\\
227	333.118555898079\\
226	321.8449170899\\
225	337.303561295674\\
224	294.677859674878\\
223	328.21695627117\\
222	290.868462939366\\
221	276.072736884972\\
220	244.711180119195\\
219	276.767932655337\\
218	263.040617509967\\
217	213.008128805131\\
216	209.595280989867\\
215	192.283982523365\\
214	192.612376308842\\
213	208.463836638114\\
212	193.341419657853\\
211	184.03477751922\\
210	144.846215109527\\
209	176.751898878497\\
208	173.301423241165\\
207	122.906914658671\\
206	146.506803514068\\
205	113.821747358039\\
204	151.315936960533\\
203	150.319624716667\\
202	157.928141087522\\
201	155.866344280915\\
200	134.574529560555\\
199	150.083148674701\\
198	159.374846096063\\
197	120.724770500401\\
196	149.268167353835\\
195	114.779356225245\\
194	134.532381392684\\
193	168.100938345023\\
192	117.70497972455\\
191	139.951201639882\\
190	158.914702684962\\
189	140.26013642493\\
188	158.497291842551\\
187	168.577965560389\\
186	192.372126139176\\
185	207.031711645753\\
184	220.739226493776\\
183	202.655198558215\\
182	231.016943556078\\
181	245.900468031856\\
180	234.861007189879\\
179	244.340230072652\\
178	250.001834365625\\
177	261.794533953599\\
176	254.554720294645\\
175	256.781188383057\\
174	193.332084837971\\
173	192.954390528951\\
172	167.603338998022\\
171	129.213576328849\\
170	132.412078198326\\
169	89.3679953368423\\
168	93.1058421001423\\
167	83.8338752539748\\
166	84.619635411908\\
165	71.8656398616228\\
164	69.6119285629895\\
163	69.8090693065299\\
162	73.3615239982447\\
161	65.1703815870429\\
160	54.9704914761117\\
159	58.3590637221856\\
158	57.6734983589335\\
157	63.8925829711087\\
156	53.7491683077567\\
155	38.0967915816613\\
154	53.342936573766\\
153	80.9401083164391\\
152	59.222960902111\\
151	48.4706741968488\\
150	86.9672091327233\\
149	77.9226890079352\\
148	89.575452002165\\
147	95.4972554691785\\
146	106.761849300295\\
145	106.303643427813\\
144	122.588368531305\\
143	123.896420286204\\
142	151.462011713224\\
141	156.226184504948\\
140	178.386371109603\\
139	184.569776258981\\
138	207.671737690666\\
137	207.669720189448\\
136	221.887148088752\\
135	223.489841246322\\
134	242.152659466941\\
133	237.883446445968\\
132	267.501703399347\\
131	278.068428902096\\
130	239.094061140473\\
129	280.147970970023\\
128	272.613411571126\\
127	267.68819236085\\
126	226.430501711018\\
125	264.216293077456\\
124	228.397395652571\\
123	235.649293003996\\
122	236.150778501583\\
121	211.390614168437\\
120	170.148905717681\\
119	184.953780225252\\
118	204.548752801019\\
117	175.114230192987\\
116	179.543951369322\\
115	143.911184856038\\
114	135.008578045689\\
113	159.560426650721\\
112	167.974718170825\\
111	126.195246620947\\
110	131.867160257956\\
109	150.317925830822\\
108	137.274389150875\\
107	91.4664715494584\\
106	136.296572054496\\
105	80.597786118521\\
104	95.9861693092383\\
103	71.9300887827629\\
102	113.097483856594\\
101	93.8751651814691\\
100	75.8614597873961\\
99	81.753800597905\\
98	106.208733174764\\
97	150.48442906942\\
96	93.8865345453279\\
95	124.345095413581\\
94	124.557797872064\\
93	121.958021621988\\
92	171.021013640421\\
91	165.74304572294\\
90	182.297269893366\\
89	164.980978208119\\
88	192.683618573676\\
87	207.013540225625\\
86	200.104182549848\\
85	186.172395248406\\
84	202.659269737045\\
83	216.578012015187\\
82	233.683364192151\\
81	254.048305742245\\
80	218.703908170401\\
79	232.067812785599\\
78	194.05230783353\\
77	181.688673970343\\
76	165.436138924621\\
75	148.811269918058\\
74	92.2049161674654\\
73	101.648152633443\\
72	81.6368548806224\\
71	92.9077496261323\\
70	76.9042902547887\\
69	78.1717261384144\\
68	73.7985671633254\\
67	61.5708647465658\\
66	54.0659512900954\\
65	63.7559559015427\\
64	55.7927077228634\\
63	44.505021160198\\
62	64.21820687082\\
61	51.995519671737\\
60	60.9143527136429\\
59	60.9660969091407\\
58	68.9069312685465\\
57	76.7193096971519\\
56	91.0230490201465\\
55	58.7523171348097\\
54	66.0696520073461\\
53	85.4765354851476\\
52	94.7917499457661\\
51	101.951187449168\\
50	106.008961072327\\
49	120.512168305655\\
48	135.219947134947\\
47	144.311267779719\\
46	151.406326709592\\
45	168.573782706228\\
44	198.027106907302\\
43	209.530937786004\\
42	223.284665627949\\
41	204.124079220394\\
40	217.971415675577\\
39	232.74241549281\\
38	233.545646763176\\
37	298.703856240102\\
36	288.454399763252\\
35	313.935322325111\\
34	313.916179872009\\
33	262.870097575781\\
32	331.138141705485\\
31	271.985470718383\\
30	310.04512900417\\
29	253.799180861303\\
28	260.817744681177\\
27	236.169370425332\\
26	205.547966321248\\
25	264.04173536679\\
24	248.78024325327\\
23	187.145326258055\\
22	157.969721072798\\
21	208.596262557846\\
20	213.225039458701\\
19	146.818126466141\\
18	155.538695109408\\
17	152.048532581222\\
16	174.637906666037\\
15	179.06041465121\\
14	146.030770854064\\
13	126.31005536978\\
12	135.643889981877\\
11	114.784126719609\\
10	96.8746724807002\\
9	103.133011738733\\
8	176.678820305957\\
7	128.031473089623\\
6	146.51983539108\\
5	110.553559814409\\
4	134.38337666408\\
3	124.079841933382\\
2	179.017254947344\\
1	164.667044362073\\
0	153.275951874902\\
}--cycle;
\addplot [color=mycolor2, line width=1.0pt, forget plot]
  table[row sep=crcr]{%
0	204.529763229583\\
1	217.097615549095\\
2	237.135250401399\\
3	177.692454499239\\
4	187.145404621795\\
5	164.792023329002\\
6	203.757571088084\\
7	181.749817830338\\
8	233.882187742667\\
9	158.885456617379\\
10	161.991641159625\\
11	177.242391482676\\
12	189.823975341631\\
13	178.956752167502\\
14	198.073930446893\\
15	230.533361195286\\
16	225.380942559592\\
17	201.930380541974\\
18	207.992059745156\\
19	200.728262056401\\
20	266.091826705397\\
21	259.682815230676\\
22	210.961324236195\\
23	238.919029231606\\
24	301.454808504595\\
25	320.039698687124\\
26	257.041664555532\\
27	286.190385149755\\
28	309.728087951203\\
29	304.490111972619\\
30	360.394361091796\\
31	323.556763403695\\
32	387.527287670627\\
33	318.240453845843\\
34	369.712452223999\\
35	368.74458871064\\
36	340.032669224678\\
37	352.105797825898\\
38	293.194747444382\\
39	289.421847777877\\
40	278.180156144929\\
41	264.112451060804\\
42	274.255212343841\\
43	261.679199048468\\
44	251.169280209075\\
45	222.088148957066\\
46	204.559579812079\\
47	200.730948687625\\
48	192.024172789245\\
49	174.080124006143\\
50	161.950090828566\\
51	160.689096425171\\
52	155.686178292682\\
53	155.473139497621\\
54	123.449403308896\\
55	114.525597200768\\
56	147.574703548924\\
57	127.305320728045\\
58	123.615166782335\\
59	112.378401000594\\
60	114.813075808346\\
61	102.909702998892\\
62	118.30762698538\\
63	97.0023346975905\\
64	108.287945275199\\
65	113.675908777713\\
66	105.667620588756\\
67	113.108371988242\\
68	125.084639023386\\
69	129.555493958072\\
70	129.111169877389\\
71	144.134039175493\\
72	138.69468966745\\
73	154.488430204677\\
74	154.988212841304\\
75	204.4184940158\\
76	220.789574331586\\
77	234.52350468798\\
78	246.642406529936\\
79	285.209011826012\\
80	274.237075857631\\
81	304.102112559704\\
82	293.079542171577\\
83	278.742762756044\\
84	261.387466481746\\
85	245.394892160537\\
86	252.534599744264\\
87	258.797973453937\\
88	247.370054198434\\
89	216.710402081055\\
90	235.396983467593\\
91	217.222625492482\\
92	224.557869125293\\
93	175.740170655599\\
94	180.408913565802\\
95	180.175528741967\\
96	151.012917307788\\
97	213.712422705578\\
98	168.075795856315\\
99	142.277123114638\\
100	138.205232176189\\
101	155.596424154382\\
};
\addplot [color=mycolor3, line width=1.0pt, forget plot]
  table[row sep=crcr]{%
0	210.73969361064\\
1	221.772317547845\\
2	237.354933635304\\
3	182.988158881895\\
4	194.948429401163\\
5	170.202843922228\\
6	211.879622337739\\
7	188.060691888364\\
8	238.958313975853\\
9	153.599717531027\\
10	151.152223620369\\
11	183.062488650129\\
12	202.961394695745\\
13	187.864213110643\\
14	210.253941291116\\
15	243.863365925082\\
16	217.699909314917\\
17	183.47017423572\\
18	217.432783261512\\
19	213.157377559531\\
20	286.244702177811\\
21	265.420247672137\\
22	217.099461190861\\
23	241.422289922088\\
24	313.592269237459\\
25	300.998318554837\\
26	271.78635951497\\
28	310.297339031357\\
29	300.096562891979\\
30	350.596775014715\\
31	300.180998253246\\
32	361.752214805119\\
33	285.622578036265\\
34	367.145280080584\\
35	382.009240501776\\
36	351.779753837806\\
37	377.395468528159\\
38	298.836746805728\\
39	300.759243030579\\
40	303.068367428329\\
41	303.959848623009\\
42	272.89125503435\\
43	269.378327092216\\
44	275.322510027829\\
45	246.286657202268\\
46	169.09640803496\\
47	189.748642092325\\
48	188.541773938146\\
49	164.836823133139\\
50	122.709417794381\\
51	135.474915696459\\
52	118.880245460346\\
53	144.540803615811\\
54	123.233506108684\\
55	105.34154127224\\
56	132.136576331963\\
57	136.22933598205\\
58	96.6356542100833\\
59	114.002256134043\\
60	109.018508677884\\
61	100.988634438253\\
62	123.889301916284\\
63	105.946903962041\\
64	100.52588539269\\
65	123.752470835767\\
66	123.551824484037\\
67	106.292306762502\\
68	108.19193374589\\
69	111.00964462239\\
70	143.15570798544\\
71	125.915385261167\\
72	135.608898775221\\
73	118.789207610187\\
74	217.683131357746\\
75	242.912916021385\\
76	237.275901063507\\
77	221.585530542754\\
78	290.626668717298\\
79	287.925728277647\\
80	317.481639516995\\
81	310.928936983625\\
82	274.978802066944\\
83	250.080813180439\\
84	255.549845097102\\
85	248.173894966512\\
87	268.36750375837\\
88	253.343737734429\\
89	228.627107137192\\
90	208.514063912798\\
91	198.332550701041\\
92	213.300499446105\\
93	165.5461904206\\
94	176.877660949407\\
95	189.689436919304\\
96	169.372682833451\\
97	235.149957871003\\
98	188.044612440246\\
99	166.232567290021\\
100	162.229568640176\\
101	173.843465468508\\
};
\end{axis}
\end{tikzpicture}%
%	\caption{}
%	\captionsetup{justification=centering}
%	\label{F:OED-acc}
%\end{figure}
