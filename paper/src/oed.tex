\section{Optimal Experiment Design}
\label{S:oed}

\subsection{Batch update: selecting most informative data for periodic model update}

The data from a real system are often noisy and contain outliers. 
It is therefore essential to filter the most \textit{informative} data that best explain the dynamics from the available pool of data.
Further, for both training time and real-time control, the computational complexity of Gaussian Processes is $\bigO(n^3)$, where $n$ is number of training samples. Thus, obtaining the best GP model with least data is highly desired.

The goal of this section is to outline a systematic procedure for optimal experiment design that can be employed to select best $k$ samples from given $n$ observations.


We use the concept of approximate marginalization in \cite{Garnett2013}.

What is the optimal procedure to select data for model training for large data

\subsection{Online update: recommending control strategies for experiment design }
