\begin{abstract}
  Building physics-based models of complex physical systems like buildings and chemical plants is extremely cost and time prohibitive for applications such as real-time optimal control, production planning and supply chain logistics.  Machine learning algorithms can reduce this cost and time complexity, and are, consequently, more scalable for large-scale physical systems. However, there are many practical challenges that must be addressed before employing machine learning for closed-loop control. This paper proposes the use of Gaussian Processes (GP) for learning control-oriented models: (1) We develop methods for the optimal experiment design (OED) of functional tests to learn models of a physical system, subject to stringent operational constraints and limited availability of the system.  Using a Bayesian approach with GP, our methods seek to select the most informative data for optimally updating an existing model.  (2) We also show that black-box GP models can be used for receding horizon optimal control with probabilistic guarantees on constraint satisfaction through chance constraints. % of a real building for power demand curtailment in context of Demand Response.
  (3) We further propose an online method for continously improving the GP model in closed loop with a real-time controller.
  Our methods are demonstrated and validated in a case study of building energy control and Demand Response.
  We show that OED achieves faster learning rate than random sampling, reducing the duration of functional tests by upto 50\%. The GP controller provides the desired curtailment perfectly with maximum 1.7\% prediction error.
\end{abstract}
