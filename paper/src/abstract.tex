\begin{abstract}
Building physics-based models for complex physical systems like buildings, chemical plants or for production planning and supply chain logistics is extremely cost and time prohibitive, especially for real-time optimal control applications. Machine learning algorithms can reduce this cost and time complexity, and are, therefore, more scalable to large-scale physical systems. However, there are many practical challenges that must be addressed before employing machine learning for real-time closed-loop control. This paper proposes the use of Gaussian Processes (GP) for learning control-oriented models. We discuss the use of GP for experiment design for a physical system, i.e.~for selecting the most informative data from the available data or for recommending control strategies for updating an existing model efficiently. We show that this black-box GP model can also be used for finite receding horizon control of a real building for load curtailment in context of Demand Response. We further propose a methodology for online learning when the GP controller is running in a closed-loop.
\todo[inline]{revisit in the end}
\end{abstract}