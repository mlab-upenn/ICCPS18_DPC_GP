\section{Data Predictive Control}
\label{S:dpc}

\subsection{Gaussian Processes for Dynamical Systems}
\label{S:intro-gp:control}

As GPs can be used to model nonlinear functions, they are suitable for modeling dynamical systems.
This is achieved by feeding delayed input and output signals back to the model as regressors \cite{kocijan16modelling}.
In such cases, the model is said to be autoregressive, whose current output depends on its past inputs and outputs.
Specifically, in control systems, it is common to use as the regressors of a dynamical GP
\begin{equation*}
x_t\!=\![y_{t-l_y}, \dots, y_{t-1}, u_{t-l_u}, \dots, u_t, w_{t-l_w}, \dots, w_{t-1}, w_t]
\end{equation*}
where \(t\) denotes the time step, \(u\) the control input, \(w\) the exogenous disturbance input, \(y\) the (past) output.
Here, \(l_y\), \(l_u\), and \(l_w\) are respectively the lags for autoregressive outputs, control inputs, and disturbances.
Note that \(u_t\) and \(w_t\) are the current control and disturbance inputs.
The vector of all autoregressive inputs can be thought of as the current state of the model.
A dynamical GP can then be trained from data in the same way as any other GPs.

When a GP is used for control or optimization, it is usually necessary to simulate the model over a finite number of future steps and predict its multistep-ahead behavior.
Because the output of a GP is a distribution rather than a point estimate, the autoregressive outputs fed to the model beyond the first step are random variables, resulting in more and more complex output distributions as we go further.
Therefore, a multistep simulation of a GP involves the propagation of uncertainty through the model.
There exist several methods for uncertainty propagation in GPs like \emph{Monte-Carlo method} and \emph{zero-variance method} \cite{girard04approximate,kocijan16modelling}. In this paper, we use separate models for each prediction step as described in Sec.~\ref{S:dpc}.

\todo[inline]{closed-loop control with the identified model}
\todo[inline]{training multiple models for each time step}
\todo[inline]{comment on non-convexity, computational complexity}


\subsection{Tracking a reference power signal}

\todo[inline]{optimization problem}

\begin{figure}[h!]
	\centering
	\missingfigure[figwidth=20pc]{}
	\caption{}
	\captionsetup{justification=centering}
	\label{F:}
\end{figure}

\subsection{Optimizing energy usage under operation constraints}

\todo[inline]{optimization problem}

\begin{figure}[h!]
	\centering
	\missingfigure[figwidth=20pc]{}
	\caption{}
	\captionsetup{justification=centering}
	\label{F:}
\end{figure}