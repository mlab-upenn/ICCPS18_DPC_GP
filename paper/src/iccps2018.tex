\documentclass[conference]{IEEEtran}
%\IEEEoverridecommandlockouts
% The preceding line is only needed to identify funding in the first footnote. If that is unneeded, please comment it out.

\usepackage{mathtools}  % for special symbols such as :=
\usepackage{xspace}

\usepackage{achinDefs}

%%%%%%%%%%%%%%%%% Macros used in the paper %%%%%%%%%%%%%%%%%%%%

% Auto-regressive vector
\newcommand{\arvec}[1]{\ensuremath{\boldsymbol{#1}}}  % ^{\leftarrow}
\def\Pbatt{\ensuremath{b}}
\def\Ptotal{\ensuremath{p}}
\def\Pbmax{\ensuremath{\Pbatt_{\mathrm{max}}}}
\def\Pbmin{\ensuremath{\Pbatt_{\mathrm{min}}}}
\def\SOCmax{\ensuremath{s_{\mathrm{max}}}}
\def\SOCmin{\ensuremath{s_{\mathrm{min}}}}

\newcommand{\GaussianDist}[2]{\ensuremath{\operatorname*{\mathcal{N}}\left(#1, #2\right)}}  % Gaussian distribution given mean and variance
\newcommand{\GaussianDistSmall}[2]{\ensuremath{\operatorname*{\mathcal{N}}(#1, #2)}}  % Gaussian distribution given mean and variance

% Typeset predicted variables. Usage:
% \predict{k}{u} --> predicted u at k | t
% \predict[\tau]{k}{u} --> predicted u at k | \tau
\newcommand{\predict}[3][t]{\ensuremath{#3}_{#2 | #1}}

\newcommand{\sigmamax}[1][0]{\ensuremath{\overbar{\sigma}_{#1}}}

\newcommand{\eg}{e.g.,\xspace}
\newcommand{\ie}{i.e.,\xspace}
\newcommand{\etc}{etc.\xspace}
\newcommand{\cf}{cf.~}

% Some commonly used math functions/operators
\DeclareMathOperator*{\argmin}{arg\,min}
\DeclareMathOperator*{\argmax}{arg\,max}
\DeclareMathOperator*{\minimize}{minimize}
\DeclareMathOperator*{\maximize}{maximize}
\DeclareMathOperator{\sgn}{sgn}  % the signum/sign function

\newcommand{\RR}{\mathbb{R}\xspace}  % Real set
\newcommand{\NN}{\mathbb{N}\xspace}  % Natural number set
\newcommand{\QQ}{\mathbb{Q}\xspace}  % Rational number set
\newcommand{\CC}{\mathbb{C}\xspace}  % Complex number set
\newcommand{\ZZ}{\mathbb{Z}\xspace}  % Integer number set
\newcommand{\EE}{\mathbb{E}\xspace}  % Expectation
\newcommand{\PP}{\mathbb{P}\xspace}  % Probability
\newcommand{\BB}{\mathbb{B}\xspace}  % Logic set {true, false}

% some fixes
%\linespread{0.915}
%\setlength{\textfloatsep}{4pt}


% make algo small
%\makeatletter
%\renewcommand{\ALG@beginalgorithmic}{\small}
%\makeatother

\begin{document}

\title{Learning and Control using Gaussian Processes \\
	\vspace{0.1cm}
	\Large \textbf{Towards bridging machine learning and controls for physical systems}}


\author{\IEEEauthorblockN{Achin Jain$^{\star,\S}$, Truong X. Nghiem$^{\star,\S,\ddag}$, Manfred Morari$^{\S}$, Rahul Mangharam$^{\S}$}
\IEEEauthorblockA{\textit{$^{\S}$University of Pennsylvania, Philadelphia, PA 19104, USA} \\
\textit{$^{\ddag}$Northern Arizona University, Flagstaff, AZ 86011, USA} \\
$^{\star}$\textit{equal contribution}}
%\and
%\IEEEauthorblockN{2\textsuperscript{nd} Given Name Surname}
%\IEEEauthorblockA{\textit{dept. name of organization (of Aff.)} \\
%\textit{name of organization (of Aff.)}\\
%City, Country \\
%email address}
%\and
%\IEEEauthorblockN{3\textsuperscript{rd} Given Name Surname}
%\IEEEauthorblockA{\textit{dept. name of organization (of Aff.)} \\
%\textit{name of organization (of Aff.)}\\
%City, Country \\
%email address}
%\and
%\IEEEauthorblockN{4\textsuperscript{th} Given Name Surname}
%\IEEEauthorblockA{\textit{dept. name of organization (of Aff.)} \\
%\textit{name of organization (of Aff.)}\\
%City, Country \\
%email address}
}

\maketitle

%% ABSTRACT
\begin{abstract}

\begin{enumerate}
	\item Role of active learning/optimal design
	\item Challenges associated with OED
	\item Machine learning for controls/DPC
	\item Updating the models online while running a DPC controller
\end{enumerate}

\end{abstract}

\begin{IEEEkeywords}
Machine learning, Gaussian Processes, Optimal Experiment Design, Receding Horizon Control, Active Learning
\end{IEEEkeywords}

%% SECTIONS
\section{Introduction}

\todo[inline]{shorten and rephrase the CDC part}
\begin{figure*}[h]
	\centering
	\missingfigure[figwidth=42pc]{Paper organization and contributions}
	\caption{Paper organization and contributions}
	\captionsetup{justification=centering}
	\label{F:intro}
\end{figure*}


\todo[inline]{Need a discussion on related work, including RT papers, GP papers \protect\cite{nghiemetal16gp}, especially for buildings.}

Machine learning and control theory are two foundational but disjoint communities. Machine learning requires data to produce models, and control systems require models to provide stability, safety or other performance guarantees. Machine learning is widely used for regression or classification, but thus far data-driven models have not been suitable for closed-loop control of physical plants. The challenge now, with using data-driven approaches, is to close the loop for real-time control and decision making.

Consider a multivariable dynamical system subject to external disturbances. The first and foremost requirement for making any decision is to obtain the underlying control-oriented predictive model of the system. With a reasonable forecast of the external disturbances, these models should predict the state of the system in the future and thus a predictive controller based on Model Predictive Control (MPC) can act preemptively to provide a desired behavior. In particular, MPC has been proven to be very powerful for multivariable systems in the presence of input and output constraints, and forecast of the disturbances. The caveat is that MPC requires a reasonably accurate physical representation of the system. This makes MPC unsuitable for control of complex plants such as natural gas processing, oil refineries, boilers, manufacturing plants, and buildings where the user expertise, time, and associated sensor costs required to develop a model are very high \cite{Sturzenegger2016,vzavcekova2014}.

There are two main reasons for model complexity. 
(1) The prime contributor is the change in model properties over time. Even if the model is identified once via an expensive route, as the model changes with time, the system identification must be repeated to update the model. Thus, model adaptability or adaptive control is desirable for such systems. 
(2) A secondary reason is the model heterogeneity which further prohibits the use of model-based control. For example, unlike the automobile or the aircraft industry, each building is designed and used in a different way. Therefore, this modeling process must be repeated for every new building. 
Due to aforementioned reasons, the control strategies in such systems are often limited to fuzzy logic rules that are based on best practices. 

The question now is, can we employ data-driven techniques to reduce the cost of modeling, and still exploit the benefits that MPC has to offer? We therefore look for automatic and data-driven approaches to control that are also adaptive, scalable and interpretable. We solve this problem with \textit{Data Predictive Control (DPC)} by bridging the gap between Machine Learning and Predictive Control.

\subsection{Challenges in bridging machine learning and controls}
\label{SS:practical_challenges}

The central idea behind DPC is to obtain control-oriented models using machine learning or black-box modeling, and formulate the control problem in a way that receding horizon control (RHC) can still be applied and the optimization problem can be solved efficiently.

It is important to note that the standard machine learning regression used for prediction is fundamentally different from using machine learning for control synthesis. In the former, all the inputs to the model (also called regressors or features) are known, while in the latter some of the inputs that are the control variables must be optimized in real-time for desired performance. 

We next discuss the practical challenges in using machine learning algorithms for control.

\noindent \textbf{(1) Data quality and quantity:} Most of the historical data that is available from complex systems like buildings are based on some rule-based controllers. Therefore, the data may not be sufficient to explain the relationship between the inputs and the outputs. To obtain richer data with enough excitation in the inputs, new experiments must be done either by sampling the inputs randomly or by a procedure for optimal experiment design (OED) \cite{Emery1998,Fedorov2010}. This paper proposes the use of GP i.e.~the estimate of variance in GP predictions to recommend control strategies for OED.

\noindent \textbf{(2) Computational complexity:} Depending upon the learning algorithm, the output from a learned model is a non-linear, non-convex and sometimes non-differentiable (eg.~Random Forests \cite{Friedman2001}) function of the inputs with no closed-form expression. Using such models for control synthesis where some of the inputs must be optimized can lead to computationally intractable optimization. Our previous work on DPC uses an adaptation of Random Forests which overcomes this problem by separation of variables to derive a linearized input-output mapping at each time step \cite{JainACC2017,JainCDC2017}.
This paper uses Gaussian Processes (GP) where the output mean and variance are analytical functions of the inputs, albeit non-convex.

\noindent \textbf{(3) Performance guarantees and robustness:} A desired characteristic for closed-loop control is to provide performance guarantees. This becomes hard when a black-box is used to replace a physical model. However, it is possible to provide probabilistic guarantees with a learning algorithm based on Gaussian Processes. DPC based on Gaussian Processes allows us to define chance constraints or account for model uncertainty in the cost while solving the optimization problem. This helps bound the performance errors with high confidence. Handling disturbance uncertainties or robustness to sensor failures in the DPC framework is part of our on-going work and is thus excluded from this paper.

\noindent \textbf{(4) Model adaptability:} It is often the case that the model properties change with time, and thus, the learned model must also be updated when required. The traditional mode of system identification, done repeatedly, can be time and cost prohibitive. In this paper, we discuss the active learning for GP models that allows to update the model from one season to another.

\noindent \textbf{(5) Interpretability of control decisions:} Besides the accuracy of synthesizing control strategies with machine learning in the loop, we are also interested in solutions that are interpretable and trustworthy. Thus, the DPC recommendations should have traceability so they can be verified to be stable and safe. This direction of research also forms part of our on-going work.

\subsection{Overcoming practical challenges}
To address these challenges, we can take two different approaches based on what level machine learning is used to learn the models.

\noindent \textbf{(1) Mix of black-box and physics-based models:} Here, we use machine learning only to learn the dynamics of a sub-system or to model uncertainties in the dynamics. An example of former is in the use of machine learning for perception and model-based control for low-level control in self-driving cars \cite{Urmson2008}. Examples on learn uncertainties in the models include \cite{Berkenkamp2015,Desaraju2016}.

\noindent \textbf{(2) Fully black-box models:} The full dynamical model can also be obtained using only machine learning algorithms. This deviates from the traditional notion of system identification where a physics-based structure is assumed to begin with. An example would be fully autonomous control using camera vision where control actions are mapped to raw image pixels \cite{Bojarski2016}. The catch here is that, prior to learning, sufficiently large data could be generated by running the car in simulations.

For the application to building control in context of Demand Response, where there is a massive cost to physical modeling, this paper explores the latter route to bypass the modeling difficulties as summarized in \cite{Sturzenegger2016}. The model-free approach allows to scale this methodology to multi-building campus and in general to any more applications like control of autonomous systems.

\subsection{Contributions}

\todo[inline]{revisit when OED and CONTROL sections are ready}

This paper answers the following questions:
\begin{enumerate}
	\item Optimal experiment design for batch updates
	\item Optimal experiment design for sequential updates in real-time
	\item Data Predictive Control using Gaussian Processes for real-time control
	\item Online learning with a Data Predictive Controller in closed-loop
\end{enumerate}



%%% Local Variables:
%%% mode: latex
%%% TeX-master: "main"
%%% End:

\section{Gaussian Processes}
\label{S:gp}

In this section, we briefly introduce modeling with Gaussian Process (GP) and its applications in control.
More details can be found in \cite{Rasmussen2006} %on GP for machine learning and in
and \cite{Kocijan2016}. % on GP modeling of dynamic systems.

\begin{definition}[\cite{Rasmussen2006}]
A Gaussian Process is a collection of random variables, any finite number of which have a joint Gaussian distribution.
\end{definition}
Consider noisy observations \(y\) of an underlying function \(f: \RR^n \mapsto \RR\) through a Gaussian noise model: \(y = f(x) + \GaussianDist{0}{\sigma_n^2}\), \(x \in \RR^n\).
A GP of \(y\) is fully specified by its mean function \(\mu(x)\) and covariance function \(k(x,x')\),
\begin{align}
\label{E:gp:prior}
\mu(x; \theta) &= \EE [f(x)] \\
k(x,x'; \theta) &= \EE [(f(x)\!-\!\mu(x)) (f(x') \!-\! \mu(x'))] + \sigma_n^2 \delta(x,x') \nonumber
\end{align}
where \(\delta(x,x')\) is the Kronecker delta function.
The hyperparameter vector \(\theta\) parameterizes the mean and covariance functions.
This GP is denoted by \(y \sim \mathcal{GP}(\mu, k; \theta)\).

Given the regression vectors \(X = [x_1, \dots, x_N]\) and the corresponding observed outputs \(Y = [y_1, \dots, y_N]^T\), the distribution of the output \(y_\star\) corresponding to a new input vector \(x_\star\) is a Gaussian distribution \(y_\star | x_\star \sim \GaussianDist{\bar{y}_\star}{\sigma_\star^2}\), with mean and variance given by
\begin{subequations}
\label{E:gp-regression}
\begin{align}
\bar{y}_\star &= g_{\mathrm{m}} (x_{\star}) \coloneqq \mu(x_\star) + K_\star K^{-1} (Y - \mu(X))\\
\sigma_\star^2 &= g_{\mathrm{v}} (x_{\star}) \coloneqq K_{\star \star} - K_\star K^{-1} K_\star^T \text,
\end{align}
\end{subequations}
where \(K_\star = [k(x_\star, x_1), \dots, k(x_\star, x_N)]\), \(K_{\star \star} = k(x_\star, x_\star)\), and $K$ is the covariance matrix with elements \(K_{ij} = k(x_i, x_j)\).

Note that the mean and covariance functions are parameterized by the hyperparameters $\theta$, which can be learned by maximizing the likelihood: \(\argmax_\theta \Pr(Y \vert X, \theta)\).
The covariance function \(k(x,x')\) indicates how correlated the outputs are at \(x\) and \(x'\), with the intuition that the output at an input is influenced more by the outputs of nearby inputs in the training data $\D = (X, Y)$.
In other words, a GP model specifies the structure of the covariance matrix of, or the relationship between, the input variables rather than a fixed structural input--output relationship.
It is therefore highly flexible and can capture complex behavior with fewer parameters.
An example of GP prior and posterior is shown in Fig.~\ref{F:gp:prior:posterior}. We use a constant mean function and a combination of squared exponential kernel and rational quadratic kernel as described in Sec.~\ref{SS:casestudy:gp}.
There exists a wide range of covariance functions and combinations to choose from \cite{Rasmussen2006}. 

GPs offer several advantages over other machine learning algorithms that make them more suitable for identification of dynamical systems.
\begin{enumerate}
\item GPs provide an estimate of uncertainty or confidence in the
  predictions through the predictive variance.  While the predictive mean is often used as the best guess of the output, the full distribution can be used in a meaningful way. For example, we can estimate a 95\% confidence bound for the predictions which can be used to measure control performance.
\item GPs work well with small data sets.  This ability is generally useful for any learning application.
\item GPs allow including prior knowledge of the system behavior by defining priors on the hyperparameters or constructing a particular structure of the covariance function.  This feature enables incorporating domain knowledge into the GP model to improve its accuracy.
\end{enumerate}

\subsection{Gaussian Processes for Dynamical Systems}
\label{SS:intro-gp:control}

GPs can be used for modeling nonlinear dynamical systems, by feeding autoregressive, or time-delayed, input and output signals back to the model as regressors \cite{Kocijan2016}.
Specifically, in control systems, it is common to use an autoregressive GP to model a dynamical system represented by the nonlinear function
\begin{math}
y_{t} = f(x_t)
\end{math}
where
\begin{equation*}
x_{t}\!=\![y_{t-l}, \dots, y_{t-1}, u_{t-m}, \dots, u_t, w_{t-p}, \dots, w_{t-1}, w_t] \text.
\end{equation*}
Here, \(t\) denotes the time step, \(u\) the control input, \(w\) the exogenous disturbance input, \(y\) the (past) output, and \(l\), \(m\), and \(p\) are respectively the lags for autoregressive outputs, control inputs, and disturbances.
Note that \(u_t\) and \(w_t\) are the current control and disturbance inputs.
The vector of all autoregressive inputs can be thought of as the current state of the model.
A dynamical GP can then be trained from data in the same way as any other GPs.

\iffalse
When a GP is used for control or optimization, it is usually necessary to simulate the model over a finite number of future steps and predict its multistep-ahead behavior.
Because the output of a GP is a distribution rather than a point estimate, the autoregressive outputs fed to the model beyond the first step are random variables, resulting in more and more complex output distributions as we go further.
Therefore, a multistep simulation of a GP involves the propagation of uncertainty through the model.
There exist several methods for uncertainty propagation in GPs \cite{girard04approximate,Kocijan2016}. 
We mention here two simulation methods for autoregressive GPs.
\begin{itemize}
	\item The \emph{Monte-Carlo method} obtains samples of the output distribution under input uncertainty, which can be seen as a Gaussian mixture.  This Gaussian mixture becomes more complex in later steps of the simulation, therefore efficient numerical algorithms must be implemented.  This method can achieve good prediction accuracy at the expense of high computational load.  It is also general, \ie it can be used with any covariance functions.
	\item The \emph{zero-variance method} does not propagate uncertainty.  At each step, the autoregressive outputs are replaced by their corresponding expected values.  Obviously, this method will underestimate the variances of the output distributions.  However, its computational simplicity is attractive, especially in optimization applications where the GP must be simulated for many times.  In such cases, if the prediction error caused by not propagating uncertainty is insignificant, the zero-variance method can and should be used.  For more detailed discussions on this topic, see \cite{Kocijan2016,girard04approximate}.
\end{itemize}

\fi

\begin{figure}[!tb]
  \centering
	\setlength\fwidth{0.4\textwidth}
	\setlength\hwidth{0.2\textwidth}	
	% This file was created by matlab2tikz.
%
%The latest updates can be retrieved from
%  http://www.mathworks.com/matlabcentral/fileexchange/22022-matlab2tikz-matlab2tikz
%where you can also make suggestions and rate matlab2tikz.
%
\definecolor{mycolor1}{rgb}{0.97647,0.89804,1.00000}%
%
\begin{tikzpicture}

\begin{axis}[%
width=0.951\fwidth,
height=\hwidth,
at={(0\fwidth,0\hwidth)},
scale only axis,
xmin=3.7,
xmax=9.7,
xlabel style={font=\color{white!15!black}},
xlabel={$\text{Chilled water temp. [}^\text{o}\text{C]}$},
ymin=8.37589565985735,
ymax=409.872922593159,
ylabel style={font=\color{white!15!black}},
ylabel={power [kW]},
axis background/.style={fill=white},
legend style={legend cell align=left, align=left, draw=white!15!black}
]

\addplot[area legend, draw=white!75!gray, fill=white!75!gray]
table[row sep=crcr] {%
x	y\\
3.7	391.623057732555\\
4.01578947368421	391.623057732555\\
4.33157894736842	391.623057732555\\
4.64736842105263	391.623057732555\\
4.96315789473684	391.623057732555\\
5.27894736842105	391.623057732555\\
5.59473684210526	391.623057732555\\
5.91052631578947	391.623057732555\\
6.22631578947368	391.623057732555\\
6.54210526315789	391.623057732555\\
6.8578947368421	391.623057732555\\
7.17368421052632	391.623057732555\\
7.48947368421053	391.623057732555\\
7.80526315789474	391.623057732555\\
8.12105263157895	391.623057732555\\
8.43684210526316	391.623057732555\\
8.75263157894737	391.623057732555\\
9.06842105263158	391.623057732555\\
9.38421052631579	391.623057732555\\
9.7	391.623057732555\\
9.7	26.625760520462\\
9.38421052631579	26.625760520462\\
9.06842105263158	26.625760520462\\
8.75263157894737	26.625760520462\\
8.43684210526316	26.625760520462\\
8.12105263157895	26.625760520462\\
7.80526315789474	26.625760520462\\
7.48947368421053	26.625760520462\\
7.17368421052632	26.625760520462\\
6.8578947368421	26.625760520462\\
6.54210526315789	26.625760520462\\
6.22631578947368	26.625760520462\\
5.91052631578947	26.625760520462\\
5.59473684210526	26.625760520462\\
5.27894736842105	26.625760520462\\
4.96315789473684	26.625760520462\\
4.64736842105263	26.625760520462\\
4.33157894736842	26.625760520462\\
4.01578947368421	26.625760520462\\
3.7	26.625760520462\\
}--cycle;
\addlegendentry{$\text{prior }\mu\text{ }\pm\text{ 2}\sigma$}

\addplot [color=black, line width=1.0pt]
  table[row sep=crcr]{%
3.69999999999999	209.124409126508\\
9.69999999999999	209.124409126508\\
};
\addlegendentry{$\text{prior }\mu$}


\addplot[area legend, draw=mycolor1, fill=mycolor1]
table[row sep=crcr] {%
x	y\\
3.7	173.087265186663\\
4.01578947368421	170.946487305241\\
4.33157894736842	168.871442969416\\
4.64736842105263	166.853382071716\\
4.96315789473684	164.88423519122\\
5.27894736842105	162.957122006465\\
5.59473684210526	161.066750293675\\
5.91052631578947	159.209700398551\\
6.22631578947368	157.384606604534\\
6.54210526315789	155.59225514469\\
6.8578947368421	153.835619661725\\
7.17368421052632	152.119850959456\\
7.48947368421053	150.452231064526\\
7.80526315789474	148.842093771292\\
8.12105263157895	147.300706284425\\
8.43684210526316	145.841100548816\\
8.75263157894737	144.477839615212\\
9.06842105263158	143.226705392763\\
9.38421052631579	142.104300594135\\
9.7	141.127569954911\\
9.7	101.068180525827\\
9.38421052631579	102.858180785079\\
9.06842105263158	104.634427758675\\
8.75263157894737	106.40410900345\\
8.43684210526316	108.173775004104\\
8.12105263157895	109.948906595954\\
7.80526315789474	111.733564246481\\
7.48947368421053	113.530123986469\\
7.17368421052632	115.339092520184\\
6.8578947368421	117.158987568735\\
6.54210526315789	118.986268514527\\
6.22631578947368	120.815305636956\\
5.91052631578947	122.638382277541\\
5.59473684210526	124.445731815086\\
5.27894736842105	126.225619206856\\
4.96315789473684	127.964483657655\\
4.64736842105263	129.647162983633\\
4.33157894736842	131.257219158361\\
4.01578947368421	132.777376249578\\
3.7	134.190065412375\\
}--cycle;
\addlegendentry{$\text{posterior }\mu\text{ }\pm\text{ 2}\sigma$}

\addplot [color=red, dashed, line width=1.0pt]
  table[row sep=crcr]{%
3.69999999999999	153.638665299519\\
4.01578947368421	151.86193177741\\
4.33157894736843	150.064331063888\\
4.64736842105262	148.250272527675\\
5.27894736842106	144.591370606661\\
5.91052631578947	140.924041338046\\
6.22631578947369	139.099956120745\\
6.54210526315791	137.289261829609\\
6.8578947368421	135.49730361523\\
7.17368421052632	133.72947173982\\
7.48947368421054	131.991177525498\\
7.80526315789473	130.287829008886\\
8.12105263157895	128.624806440189\\
8.43684210526317	127.00743777646\\
8.75263157894736	125.440974309331\\
9.06842105263158	123.930566575719\\
9.3842105263158	122.481240689607\\
9.69999999999999	121.097875240369\\
};
\addlegendentry{$\text{posterior }\mu$}

\end{axis}
\end{tikzpicture}%
    \todo[inline]{This figure is impossibe to see!!! You can truncate the length by half (show the result for 1 day, or 12 hours instead). Create two smaller plots side by side, or select colors better.}
  \caption{Example of priors calculated using \eqref{E:gp:prior} and posteriors using \eqref{E:gp-regression} for predicting power consumption of a building for two days. Initially the mean is constant because \(\mu(x)\) is constant, and we observe a high variance. The posterior agrees with the actual power consumption with high confidence.}
  \captionsetup{justification=centering}
  \label{F:gp:prior:posterior}
\end{figure}

%%% Local Variables:
%%% mode: latex
%%% TeX-master: "main"
%%% End:

\section{Optimal Experiment Design}
\label{S:oed}

In this section, we address the practical challenge of ``Data quality and quantity" and also touch upon ``Model adaptability" listed in Sec.~\ref{SS:practical_challenges}.
\\

For practical applications, we come across two kinds of situations:
\begin{enumerate}
	\item \textbf{Insufficient data:} In general, the more data we have, a better model we can learn using machine learning algorithms. When sufficient training data is not available for learning the behavior of the dynamical system, we resort to \textit{optimal experiment design} (OED) or \textit{functional testing}, a method of exciting the inputs of the dynamical system and measuring its response. For example, in the control literature, a popular technique, especially for linear systems, is measuring the \textit{step response} of the system to estimate the time-constants, and further for designing of controllers. In context of buildings as we discuss in Sec.~\ref{S:casestudy}, it is often the case that very limited data from the installed sensors and multimeters are available. Hence, we need to design a mechanism to recommend control strategies to sample new data.
	
	\item \textbf{Computational complexity:} Even if we have sufficient data, it may not be directly suitable for learning because of noisy measurements/outliers or using the entire data set may be not advisable due to computational complexity as is the case with GPs. The solution to this problem lies in \textit{selecting the most informative batch} (from the available data) that best explains the system behavior or dynamics. Another application of this in periodic update of the learned model as the system properties change over time. For example, the same GP model may be not be suitable to control a building in both Summer and Winter season, so we must select the most informative data from year around data. We discuss active learning in detail in Sec.~\ref{S:active}.
\end{enumerate}

For \textit{optimal experiment design} and \textit{selecting the most informative batch} with GP as the learned model, we follow the \textit{information theoretic} approach to estimate how well the training samples explain the behavior underlying physical system.

\subsection{Information theoretic approach to OED}

In this section, we show how the prediction variance \eqref{E:gp-regression} in GPs can be exploited for experiment design.
The goal here is to update the parameters \(\theta\) in the model \(y \sim \mathcal{GP}(\mu(x), k(x); \theta)\) as new samples are observed sequentially. One popular metric of selecting the next sample is the point of Maximum Variance (MV), which is also widely used for Bayesian Optimization using GPs \cite{Snoek2012}. Since, we can calculate the variance in \(y\) for any \(x\), OED based on MV is straight forward to compute. However, another metric which has proven to provide better procedure to update parameters \(\theta\) is the Information Gain (IG) \cite{Krause2008}. 

IG metric selects the sample which adds maximum information to the model, i.e.~reduces the uncertainty in \(\theta\) the most. If we denote the existing data before sampling by \(D\), then the goal is to select \(x\) that maximizes the information gain defined as
\begin{align}
\argmax_x H(\theta|D) - \EE_{y \sim \GaussianDist{\bar{y}(x)}{\sigma^2(x)}}H(\theta|D,x,y),
\label{E:ig:theta}
\end{align}
where, \(H\) is the Shanon's Entropy given by
\begin{align}
H(\theta|D) = -\int p(\theta|D) \log (p(\theta|D))d\theta.
\end{align}
Since \(y|x \sim \GaussianDist{\bar{y}(x)}{\sigma^2(x)}\), we need to take an expectation over \(y\). When the dimension of \(\theta\) is large, computing entropies is typically computationally intractable. Using equivalence of the expressions \(H(\theta) - H(\theta|y) = H(y) - H(y|\theta)\),
we can transform \eqref{E:ig:theta} as
\begin{align}
\argmax_x H(y|x,D) - \EE_{\theta \sim p(\theta|D)}H(y|x,\theta).
\label{E:ig:y}
\end{align}
In this case, the expectation is defined over \(\theta\) the expression \eqref{E:ig:y} is much easier to compute because \(y\) is now single dimension. For further details, we refer the reader to \cite{Houlsby2011}.
The first term in \eqref{E:ig:y} can be calculated by marginalizing over the distribution of \(\theta|D\):
\begin{align}
p(y|x,D) =& \EE_{\theta \sim p(\theta|D)}p(y|x,\theta,D) \nonumber\\
=& \int p(y|x,\theta, D)p(\theta|D)d\theta
\end{align}
for which the exact solution is difficult to compute. We therefore use an approximation described in \cite{Garnett2013}. It is shown that for \(\theta|D \sim \GaussianDist{\bar{\theta}}{\Sigma}\), we can find a linear approximation to \(\bar{y}(x) = a^T(x)\theta+b(x)\) such that
\begin{align}
\EE_{\theta \sim p(\theta|D)}p(y|x,\theta,D) \sim \GaussianDist{a^T\bar{\theta}+b}{\sigma^2+a^T\Sigma a}.
\end{align}
Under the same approximation, the second term in \eqref{E:ig:y} can be written as \(H(y|x,\hat{\theta})\). 
Finally, using the relation for the differential entropy for a Gaussian distribution, the information gain in \eqref{E:ig:y} is approximated as
%\begin{align}
%H(y|x,D) = \frac{1}{2}\log(2\pi e \sigma^2(x)).
%\end{align}
\begin{align}
\text{IG} = \frac{1}{2}\log\left(\frac{\sigma^2(x)+a^T(x)\Sigma a(x)}{\sigma^2(x)}\right).
\end{align}


\subsection{Batch selection: selecting most informative data for periodic model update}

The data from a real system are often noisy and contain outliers. 
It is therefore essential to filter the most \textit{informative} data that best explain the dynamics from the available pool of data.
Further, for both training time and real-time control, the computational complexity of Gaussian Processes is $\bigO(n^3)$, where $n$ is number of training samples. Thus, obtaining the best GP model with least data is highly desired.

The goal of this section is to outline a systematic procedure for optimal experiment design that can be employed to select best $k$ samples from given $n$ observations.

What is the optimal procedure to select data for model training for large data

\begin{figure}[h!]
	\centering
	\missingfigure[figwidth=20pc]{comparison of model accuracy b/w IG, MV, Uniform, PRBS}
	\caption{}
	\captionsetup{justification=centering}
	\label{F:}
\end{figure}

\subsection{Sequential sampling: recommending control strategies for experiment design }

\todo[inline]{extension of last approach for functional testing}
\todo[inline]{useful under operational constraints}

\begin{figure}[h!]
	\centering
	\missingfigure[figwidth=20pc]{BAR PLOT: comparison of model accuracy b/w  IG, MV, Uniform, PRBS}
	\caption{}
	\captionsetup{justification=centering}
	\label{F:}
\end{figure}

\section{Data Predictive Control}
\label{S:dpc}

\todo[inline]{closed-loop control with the identified model}
\todo[inline]{training multiple models for each time step}
\todo[inline]{comment on non-convexity, computational complexity}


\subsection{Tracking a reference power signal}

\todo[inline]{optimization problem}

\begin{figure}[h!]
	\centering
	\missingfigure[figwidth=20pc]{}
	\caption{}
	\captionsetup{justification=centering}
	\label{F:}
\end{figure}

\subsection{Optimizing energy usage under operation constraints}

\todo[inline]{optimization problem}

\begin{figure}[h!]
	\centering
	\missingfigure[figwidth=20pc]{}
	\caption{}
	\captionsetup{justification=centering}
	\label{F:}
\end{figure}
\section{Active Learning / Evolving GP?}
\label{S:active}

\todo[inline]{needs reorganization of OED section}
\todo[inline]{controller in the loop}

\section{Case Study}
\label{S:casestudy}

In January 2014, the east coast electricity grid, managed by PJM, experienced an 86-fold increase in the price of electricity from \$31/MWh to \$2,680/MWh in a matter of 10 minutes.
Similarly, the price spiked 32 times from an average of \$25/MWh to \$800/MWh in July of 2015.
This extreme price volatility has become the new norm in our electric grids.
Building additional peak generation capacity is not environmentally or economically sustainable.
Furthermore, the traditional view of energy efficiency does not address this need for \emph{Energy Flexibility}.
A promising solution lies with Demand Response (DR) from the customer side -- curtailing demand during peak capacity for financial incentives.
However, it is a very hard problem for commercial, industrial and institutional plants -- the largest electricity consumers -- to decide which knobs to turn to achieve the required curtailment, due to the large scale and high complexity of these systems.
Therefore, the problem of energy management during a DR event makes an ideal case for our proposed approach of combining machine learning and control.
In this section, we apply optimal experiment design, receding horizon control based on GPs, and evolving GPs on large scale EnergyPlus models to demonstrate the effectiveness of our approach. % can provide a desired power curtailment as well as a desired thermal comfort.
%DPC builds predictive models of a building based on historical weather, schedule, set-points and electricity consumption data, while also learning from the actions of the building operator. These models are then used for synthesizing recommendations about the control actions that the operator needs to take, during a DR event, to obtain a given load curtailment while providing guarantees on occupant comfort and operations.

\subsection{Building Description}
\label{SS:casestudy:building}
We use two different U.S. Department of Energy's Commercial Reference Buildings (DoE CRB) simulated in EnergyPlus \cite{Deru2011} as the virtual test-bed buildings.
The first is a 6-story hotel consisting of 22 zones with a total area of 120,122 sq.ft, with a peak load of about 400 kW.
The second building is a large 12-story office building consisting of 19 zones with a total area of 498,588 sq.ft. 
Under peak load conditions the office can consume up to 1.4 MW of power. 
Developing a high fidelity physics-based model for these buildings would require massive cost and effort.
Leveraging machine learning algorithms, we can now do both prediction and control with high confidence at a very low cost.

We use the following data to validate our results. We limit ourselves to data which can be measured directly from installed sensors like thermostats, multimeters and weather forecasts, thus making it scalable to any other building or a campus of buildings.

\begin{itemize}
\item \textit{Weather variables \(d^w\):} outside temperature and humidity -- these features are defined in EnergyPlus.
\item \textit{Proxy features \(d^p\):} time of day, day of week -- these features are a good indicator of occupancy and periodic trends.
%\textit{Fixed schedules  \(d^s\):} kitchen cooling set point, corridor cooling set point - these set points follow predefined rules. 
\item \textit{Control variables \(u\):} cooling, supply air temperature and chilled water setpoints -- these will be optimized in the MPC problem. % for Power Tracking Reference Control in Sec.~\ref{SS:power_tracking}.
\item \textit{Output variable \(y\):} total power consumption -- this is the output of interest which we will predict using all the above features in the GP model.
\end{itemize}

The time step for modeling and control is 15 minutes.

\subsection{Gaussian Process Models}
\label{SS:casestudy:gp}

% For MPC, we require a predictive model for each time step in the horizon.
We learn a single GP model of the building and use the \emph{zero-variance method} to predict the outputs \(y\) at the future time steps following the current time step.
For each prediction step $t+\tau$, where $t$ is the current time and \( \tau \ge 0\), %\in \{0,\dots,,N-1\}\),
the output \(y_{t+\tau}\) is a Gaussian random variable given by \eqref{eq:dpc:prediction}.
\begin{gather}
\label{E:gp:casestudy}
y_{t+\tau} \sim \GaussianDist{\bar{y}_{t+\tau} = g_{\mathrm m}(x_{t+\tau})}{\sigma^2_{y, t+\tau} = g_{\mathrm v}(x_{t+\tau})}, \\
x_{t + \tau} = [\bar y_{t+ \tau-l}, \dots, \bar y_{t+ \tau-1}, u_{t+ \tau-m}, \dots, u_{t+ \tau}, \nonumber \\
\qquad\qquad\qquad\qquad  w_{t+ \tau-p}, \dots, w_{t+ \tau-1}, w_{t+ \tau}]\text. \nonumber
\end{gather}
where \(w:=[d^w, d^p, d^s]\). We assume that at time \(t\), \(w_{t+\tau}\) are available \(\forall \tau \) from forecasts or fixed rules as applicable.

As for the mean and covariance functions of the GP, %to define the structure of GP in \eqref{E:gp:prior},
we use a constant mean \(\mu\) and a kernel function \(k(x,x')\) which is a mixture of constant kernel \(k_1(x,x')\), squared exponential kernel \(k_2(x,x')\) and rational quadratic kernel \(k_3(x,x')\) as
\begin{gather}
k_1(x,x')  = k, \nonumber\\
k_2(x,x') = \sigma_{f_2}^2 \exp \left( -\frac{1}{2} \sum_{d=1}^D \frac{(x_d-x_d')}{{\lambda_d^2}}^2 \right),
 \nonumber\\
 k_3(x,x') = \sigma_{f_3}^2  \left( 1+ \frac{1}{2\alpha} \sum_{d=1}^D \frac{(x_d-x_d')}{{\lambda^2}}^2 \right)^{-\alpha},  \nonumber\\
k(x,x') = \left(k_1(x,x') + k_2(x,x')\right)*k_3(x,x').
\end{gather}
Here, \(k_3(x,x')\) is applied to only temporal features like time of day and day of week, while \(k_1(x,x')\) and \(k_2(x,x')\) are applied to all the remaining features as proposed in \cite{nghiemetal16gp}.
We optimize the hyperparameters \(\theta\) % = [\mu, k, \sigma_{f_2}, \lambda_d, \sigma_{f_3}, \alpha, \lambda] \)
of the model in \eqref{E:gp:casestudy} using GPML \cite{Rasmussen2010}.
% After training, the less important features, i.e.~features with high length scales \(\lambda_d\) are removed and the models are trained again. 
%We denote this final selection by \(\theta^\star\).

\subsection{Optimal Experiment Design}

\begin{figure*}[t]
	\centering
	\setlength\fwidth{0.46\textwidth}
	\setlength\hwidth{0.2\textwidth}
	% This file was created by matlab2tikz.
%
%The latest updates can be retrieved from
%  http://www.mathworks.com/matlabcentral/fileexchange/22022-matlab2tikz-matlab2tikz
%where you can also make suggestions and rate matlab2tikz.
%
\definecolor{mycolor1}{rgb}{0.00000,0.44700,0.74100}%
\definecolor{mycolor2}{rgb}{0.85000,0.32500,0.09800}%
\definecolor{mycolor3}{rgb}{0.49400,0.18400,0.55600}%
\definecolor{mycolor4}{rgb}{0.46600,0.67400,0.18800}%
%
\begin{tikzpicture}

\begin{axis}[%
width=0.411\fwidth,
height=\hwidth,
at={(0\fwidth,0\hwidth)},
scale only axis,
unbounded coords=jump,
xmin=0.75,
xmax=4.25,
xtick={1,2,3,4},
xticklabels={{3},{7},{10},{14}},
xlabel style={font=\color{white!15!black}},
xlabel={no of days},
ymin=20,
ymax=65,
ylabel style={font=\color{white!15!black}},
ylabel={RMSE [kW]},
axis background/.style={fill=white},
title style={font=\bfseries},
title={HOTEL},
xlabel style={font=\footnotesize},ylabel style={font=\footnotesize},legend style={font=\footnotesize},ticklabel style={font=\footnotesize},ylabel shift = -5 pt,xlabel shift = -5 pt,title style={font=\normalsize},title style={yshift=1.5ex},legend style={at={(0,1)}}, align=center
]
\addplot [color=mycolor1, line width=1.0pt, mark=diamond, mark options={solid, mycolor1}, forget plot]
  table[row sep=crcr]{%
1	38.3577768196538\\
2	23.5762131941091\\
3	22.2379994094524\\
4	22.0269257238242\\
5	nan\\
};
\addplot [color=mycolor2, line width=1.0pt, mark=diamond, mark options={solid, mycolor2}, forget plot]
  table[row sep=crcr]{%
1	42.5373542684555\\
2	26.3609622756066\\
3	24.4584853322209\\
4	23.5534349154775\\
5	nan\\
};
\addplot [color=mycolor3, dashed, line width=1.0pt, mark=o, mark options={solid, mycolor3}, forget plot]
  table[row sep=crcr]{%
1	60.5953168967942\\
2	33.7867759715775\\
3	25.1122218421038\\
4	23.931187395704\\
5	nan\\
};
\addplot [color=mycolor4, dashed, line width=1.0pt, mark=o, mark options={solid, mycolor4}, forget plot]
  table[row sep=crcr]{%
1	56.9865000401714\\
2	38.0806145496609\\
3	31.0863921959231\\
4	26.9411116227271\\
5	nan\\
};
\end{axis}

\begin{axis}[%
width=0.411\fwidth,
height=\hwidth,
at={(0.54\fwidth,0\hwidth)},
scale only axis,
unbounded coords=jump,
xmin=0.75,
xmax=4.25,
xtick={1,2,3,4},
xticklabels={{3},{7},{10},{14}},
xlabel style={font=\color{white!15!black}},
xlabel={no of days},
ymin=0.5,
ymax=0.95,
ytick={0.5, 0.6, 0.7, 0.8, 0.9},
ylabel style={font=\color{white!15!black}},
ylabel={1-SMSE [\%]},
axis background/.style={fill=white},
title style={font=\bfseries},
title={HOTEL},
legend style={at={(0.5,1.03)}, anchor=south, legend columns=4, legend cell align=left, align=left, fill=none, draw=none},
xlabel style={font=\footnotesize},ylabel style={font=\footnotesize},legend style={font=\footnotesize},ticklabel style={font=\footnotesize},ylabel shift = -5 pt,xlabel shift = -5 pt,title style={font=\normalsize},title style={yshift=1.5ex},legend style={at={(0,1)}}, align=center
]
\addplot [color=mycolor1, line width=1.0pt, mark=diamond, mark options={solid, mycolor1}]
  table[row sep=crcr]{%
1	0.803327193943856\\
2	0.925700556917536\\
3	0.933895825369972\\
4	0.935144735780913\\
5	nan\\
};
\addlegendentry{IG}

\addplot [color=mycolor2, line width=1.0pt, mark=diamond, mark options={solid, mycolor2}]
  table[row sep=crcr]{%
1	0.758132003718017\\
2	0.907111921826193\\
3	0.920035619815826\\
4	0.925844057134119\\
5	nan\\
};
\addlegendentry{MV}

\addplot [color=mycolor3, dashed, line width=1.0pt, mark=o, mark options={solid, mycolor3}]
  table[row sep=crcr]{%
1	0.50918753883645\\
2	0.84740828201528\\
3	0.915703850575905\\
4	0.923446340971381\\
5	nan\\
};
\addlegendentry{Uniform}

\addplot [color=mycolor4, dashed, line width=1.0pt, mark=o, mark options={solid, mycolor4}]
  table[row sep=crcr]{%
1	0.565908354542458\\
2	0.80615912780776\\
3	0.870825109483224\\
4	0.90297838916829\\
5	nan\\
};
\addlegendentry{PRBS}

\end{axis}
\end{tikzpicture}%
	% This file was created by matlab2tikz.
%
%The latest updates can be retrieved from
%  http://www.mathworks.com/matlabcentral/fileexchange/22022-matlab2tikz-matlab2tikz
%where you can also make suggestions and rate matlab2tikz.
%
\definecolor{mycolor1}{rgb}{0.00000,0.44700,0.74100}%
\definecolor{mycolor2}{rgb}{0.85000,0.32500,0.09800}%
\definecolor{mycolor3}{rgb}{0.49400,0.18400,0.55600}%
\definecolor{mycolor4}{rgb}{0.46600,0.67400,0.18800}%
%
\begin{tikzpicture}

\begin{axis}[%
width=0.411\fwidth,
height=\hwidth,
at={(0\fwidth,0\hwidth)},
scale only axis,
unbounded coords=jump,
xmin=0.75,
xmax=5.25,
xtick={1,2,3,4,5},
xticklabels={{3},{7},{10},{14},{21}},
xlabel style={font=\color{white!15!black}},
xlabel={no of days},
ymin=100,
ymax=240,
ylabel style={font=\color{white!15!black}},
ylabel={RMSE [kW]},
axis background/.style={fill=white},
xlabel style={font=\footnotesize},ylabel style={font=\footnotesize},legend style={font=\footnotesize},ticklabel style={font=\footnotesize},ylabel shift = -5 pt,xlabel shift = -5 pt,legend style={at={(0,1)}}, align=center
]
\addplot [color=mycolor1, line width=1.0pt, mark=diamond, mark options={solid, mycolor1}, forget plot]
  table[row sep=crcr]{%
1	202.916706713584\\
2	123.317303637472\\
3	114.6037119272\\
4	113.951810031974\\
5	108.308872828757\\
6	nan\\
};
\addplot [color=mycolor2, line width=1.0pt, mark=diamond, mark options={solid, mycolor2}, forget plot]
  table[row sep=crcr]{%
1	168.449415909196\\
2	123.911101157714\\
3	116.678521620139\\
4	115.852548679535\\
5	108.095648661131\\
6	nan\\
};
\addplot [color=mycolor3, dashed, line width=1.0pt, mark=o, mark options={solid, mycolor3}, forget plot]
  table[row sep=crcr]{%
1	222.545060580002\\
2	135.092384886908\\
3	120.852590121443\\
4	118.158218495779\\
5	106.889101053448\\
6	nan\\
};
\addplot [color=mycolor4, dashed, line width=1.0pt, mark=o, mark options={solid, mycolor4}, forget plot]
  table[row sep=crcr]{%
1	219.422300879522\\
2	130.994515470022\\
3	125.446032559312\\
4	123.126440363686\\
5	124.10982696347\\
6	nan\\
};
\end{axis}

\begin{axis}[%
width=0.411\fwidth,
height=\hwidth,
at={(0.54\fwidth,0\hwidth)},
scale only axis,
unbounded coords=jump,
xmin=0.75,
xmax=5.25,
xtick={1,2,3,4,5},
xticklabels={{3},{7},{10},{14},{21}},
xlabel style={font=\color{white!15!black}},
xlabel={no of days},
ymin=-1.5,
ymax=1,
ylabel style={font=\color{white!15!black}},
ylabel={MSLL},
axis background/.style={fill=white},
legend style={at={(0.5,1.03)}, anchor=south, legend columns=4, legend cell align=left, align=left, fill=none, draw=none},
xlabel style={font=\footnotesize},ylabel style={font=\footnotesize},legend style={font=\footnotesize},ticklabel style={font=\footnotesize},ylabel shift = -5 pt,xlabel shift = -5 pt,legend style={at={(0,1)}}, align=center
]
\addplot [color=mycolor1, line width=1.0pt, mark=diamond, mark options={solid, mycolor1}]
  table[row sep=crcr]{%
1	-0.431251512591452\\
2	-1.11219989870435\\
3	-1.1970167867125\\
4	-1.21292672943036\\
5	-1.2674514749276\\
6	nan\\
};
\addlegendentry{IG}

\addplot [color=mycolor2, line width=1.0pt, mark=diamond, mark options={solid, mycolor2}]
  table[row sep=crcr]{%
1	-0.276374438592905\\
2	-1.0815817558078\\
3	-1.14376474427609\\
4	-1.15149078446885\\
5	-1.22773886044186\\
6	nan\\
};
\addlegendentry{MV}

\addplot [color=mycolor3, dashed, line width=1.0pt, mark=o, mark options={solid, mycolor3}]
  table[row sep=crcr]{%
1	0.759174120362543\\
2	-0.944151441587627\\
3	-1.12386652991089\\
4	-1.15822269745498\\
5	-1.27519098327653\\
6	nan\\
};
\addlegendentry{Uniform}

\addplot [color=mycolor4, dashed, line width=1.0pt, mark=o, mark options={solid, mycolor4}]
  table[row sep=crcr]{%
1	-0.140417197047048\\
2	-1.10320385426257\\
3	-1.14105547856784\\
4	-1.15627502207335\\
5	-1.10813518412733\\
6	nan\\
};
\addlegendentry{PRBS}

\end{axis}
\end{tikzpicture}%
	\caption{Comparison of model accuracies for different experiments: OED based on information gain (IG),  OED based on maximum variance (MV), uniform random sampling (Uniform) and pseudo random binary sampling (PRBS) for two buildings: hotel (left) and office (right). RMSE denotes Root Mean Square Error and SMSE means Standardized Mean Square Error; lower RMSE and higher 1-SMSE indicate better prediction accuracy.}
	\captionsetup{justification=centering}
	\label{F:casestudy:oed}
\end{figure*}

OED is powerful when limited data are available for training. 
To demonstrate this, using Algo.~\ref{A:oed:sequential}, we begin the experiment by assigning \(\GaussianDist{0}{1}\) priors to the kernel hyperparameters. %\(\log (\theta) \) elementwise, except for \(\mu\), since GPML applies gradient descent directly on \(\log \theta\).
For OED, we only consider the one-step-ahead model with \(\tau=0\) in \eqref{E:gp:casestudy}.
The goal at time \(t\) is to determine what should be the optimal cooling set-point \(u_{\mathrm{clg},t}\), supply air temperature set-point \(u_{\mathrm{sat},t}\), and chilled water temperature set-point \(u_{\mathrm{chw},t}\) which, when applied to the building, will require power consumption \(y_t\) such that \((x_t,y_t)\) can be used to learn \(\theta\) as efficiently as possible.
We use the lagged terms of the power consumption, proxy variables, weather variables and their lagged terms to define \(x_t(u_{\mathrm{clg,t}},u_{\mathrm{sat,t}},u_{\mathrm{chw,t}})\).
We assume a practical operational constraint that the chilled water temperature set-point cannot be changed faster than \(0.13^o\mathrm{C}\) every minutes.
Keeping this constraint and thermal comfort constraints into consideration, we consider the following operational constraints:
\begin{gather}
22^o\mathrm{C} \leq u_{\mathrm{clg,t}} \leq  26^o\mathrm{C}, \nonumber \\
12^o\mathrm{C} \leq u_{\mathrm{sat,t}} \leq  14^o\mathrm{C}, \nonumber \\
 3.7^o\mathrm{C} \leq u_{\mathrm{chw,t}} \leq  9.7^o\mathrm{C},\label{E:operation_constraints} \\
| u_{\mathrm{chw},t} - u_{\mathrm{chw},t-1}| \leq  2^o\mathrm{C}. \nonumber
\end{gather}
Finally, we solve the optimization \eqref{E:oed:batch}, subject to the operational constraints \eqref{E:operation_constraints}, every \(15 \mathrm{min}\) to calculate optimal inputs for OED.
% \begin{align}
% \label{E:casestudy:oed}
% \maximize_{u_{\mathrm{clg},t},u_{\mathrm{sat},t},u_{\mathrm{chw},t}} & \ \ \ \frac{1}{2}\log\left(\frac{\sigma^2_{t}(x_t)+a^T(x_t)\Sigma a(x_t)}{\sigma^2_{t}(x_t)}\right) \\
% \st &  \ \ \ \  \text{operations constraints } \eqref{E:operation_constraints}.\nonumber
% \end{align}

The results for experiment design in closed-loop with the EnergyPlus building models described in Sec.~\ref{SS:casestudy:building} are shown in Fig.~\ref{F:casestudy:oed}.
We compare 4 different methods: OED based on maximum information gain (IG), OED based on maximum variance (MV), uniform random sampling (Uniform) and pseudo random binary sampling (PRBS).
The inputs \(u_{\mathrm{clg,t}},u_{\mathrm{sat,t}},u_{\mathrm{chw,t}}\) generated via OED or random sampling are applied to the building every \(15 \mathrm{min}\).
We repeat OED/random sampling continuously for \(14\) days and learn a model at the end of each day using the data generated until that time. 
For example, at the end day of day \(3\) we have \(3\times96\) samples, at day \(7\) we have \(7\times96\) samples and so on. 
As the days progress, we add more training samples and therefore the model accuracy is expected to increase with time. 
This is visible in both metrics Root Mean Square Error (RMSE) and Standardized Mean Square Error (SMSE) for both buildings.

For OED based on information gain as well as maximum variance, the learning rate is much faster than any random sampling.
For the hotel building on the left, the IG method is the best in terms of accuracy. %RMSE and SMSE.
Uniform random sampling and PRBS are far worse in both metrics for the first \(7-10\) days (\(\approx200\) hrs), beyond which we obtain sufficient data that it is hard to distinguish between the %performance and
model accuracies.
For the office building on the right, IG is marginally better than MV in terms of SMSE for all days, while MV shows faster learning rate with lower RMSE. 
Thus for the office building, OED based on IG and MV are comparable. 
With the random sampling, we observe the same trend as before, i.e.~much worse for the first \(\approx200\) hrs of experiment, after which model accuracies are similar for IG, MV, Uniform and PRBS.

It is observed that OED can be used to learn a model very fast. However, when random sampling can be done for sufficiently long time, the two approaches result in similar models. Due to operational constraints, function tests usually cannot be performed for sufficiently long time, in which case even few hours of periodic OED can provide far better models due to its ability to capture more information in the same amount of time.


\subsection{Power Reference Tracking Control}
\label{SS:power_tracking}

This section formulates an MPC approach for the following demand tracking problem.
Consider a building, which responds to various setpoints resulting in power demand variations, and a battery, whose state of charge (SoC) can be measured and whose charge/discharge power can be controlled.
Given a power reference trajectory, for example a curtailed demand trajectory from the nominal energy consumption profile (the \emph{baseline}), our objective is to control the building and the battery to track the reference trajectory as closely as possible without violating the operational constraints.
The building's response to the setpoint changes is modeled by a GP.
The battery helps improve the tracking quality by absorbing the prediction uncertainty of the GP.
An MPC based on the GP model computes the setpoints for the building and the power of the battery to optimally track the reference demand signal.

For simplicity, we assume an ideal lossless battery model
\begin{equation}
\label{eq:battery-model}
s_{t+1} = s_t + T \Pbatt_t
\end{equation}
where \(\Pbatt_t\) is the battery's power at time step \(t\) and \(s\) is the battery's SoC.
Here, \(\Pbatt\) is positive if the battery is charging and negative if discharging.
The battery is subject to power and SoC constraints:
\(\Pbmin \leq \Pbatt_t \leq \Pbmax\), 
and \(\SOCmin \leq s_t \leq\SOCmax\) where \(\SOCmax\) is the fully-charged level and \(\SOCmin\) is the lowest safe discharged level. 

The building and the battery are linked via the power tracking constraint which states that \(p_t = y_t + b_t\) should track the reference \(r_t\) at any time \(t\). Therefore, our objective is to minimize \(\delta_t = r_t - p_t\).
In this way, the battery helps reject the uncertainty of the GP and acts as an energy buffer to increase the tracking capability of the system. 
The controller tries to keep \(\delta_t = 0\), however when exact tracking is impossible, it will maintain the operational safety of the system while keeping \(\delta_t\)  as small as possible.
The bounds on the battery's power and SoC lead to corresponding chance constraints.
We wish to guarantee that at each time step, the power and SoC constraints are satisfied with probability at least \((1 - \epsilon_p)\) and  at least \((1 - \epsilon_s)\), respectively, where \(0 < \epsilon_p, \epsilon_s \leq \frac{1}{2}\) are given constants.
Specifically, for each \(\tau\) in the horizon,
	\begin{gather}
	\label{E:battery_chance}
	\Pr\left( \Pbmin \leq b_{\tau+t}\leq \Pbmax \right) \geq 1 - \epsilon_p  \\
	\label{E:SoC_chance}
	\Pr\left( \SOCmin \leq s_{\tau+t} \leq \SOCmax \right) \geq 1 - \epsilon_s 
	\end{gather}
where \(b_{t+\tau}\) and \(s_{t+\tau}\) are Gaussian random variables whose mean and variance are given by
\begin{gather}
\bar{b}_{t+\tau}= r_{t} - \delta_{t+\tau} - \bar{y}_{t+\tau}, \ \ 
\sigma^2_{b,t+\tau} =  \sigma^2_{y,t+\tau} \text, \label{E:battery_dist} \\
\bar{s}_{t+\tau+1}\!=\! s_t \!+\! T \textstyle\sum_{k=t}^{t+\tau} \predict{k}{\bar{\Pbatt}}, \,
\predict{s,\tau+1}{\sigma^2} \!=\! T^2 \textstyle\sum_{k=t}^{t+\tau} \predict{y,k}{\sigma^2} \text. \label{E:SoC_dist}
\end{gather}
For further details on modeling we refer the reader to our previous work \cite{nghiemetal16gp}.
To track a given reference power signal, we solve the following stochastic optimization problem to optimize \(\delta_{\tau+t},u_{\mathrm{clg},\tau+t},u_{\mathrm{sat},\tau+t},u_{\mathrm{chw},\tau+t} \ \forall \tau \in \{0,\dots,N-1\}\)
\begin{align}
\label{E:casestudy:mpc}
\minimize_{\delta, u} \quad & \sum_{\tau=0}^{N-1} (\delta_{\tau+t})^2 + \lambda \sigma_{y,\tau+t}^2\\
\st \quad & \text{dynamics constraints } \eqref{E:gp:casestudy}, \eqref{E:battery_chance} - \eqref{E:SoC_dist} \nonumber \\
&  \text{operation constraints } \eqref{E:operation_constraints}. \nonumber
\end{align}
The term \( \sigma_{y,\tau+t}^2\) in the objective functions ensures control setpoints where model is more confident.
At time \(t\), we solve for \(u^*_{t},\dots,u^*_{t+N-1} \), apply the first input \(u^*_{t} \) to the building, and proceed to the next time step.

%\textit{\begin{figure}[!tb]
%	\centering
%	\setlength\fwidth{0.44\textwidth}
%	\setlength\hwidth{0.15\textwidth}	
%	\input{figures/control-tracking.tex}
%	\setlength\fwidth{0.44\textwidth}
%	\setlength\hwidth{0.1\textwidth}	
%	% This file was created by matlab2tikz.
%
%The latest updates can be retrieved from
%  http://www.mathworks.com/matlabcentral/fileexchange/22022-matlab2tikz-matlab2tikz
%where you can also make suggestions and rate matlab2tikz.
%
\definecolor{mycolor1}{rgb}{0.97647,0.89804,1.00000}%
\definecolor{mycolor2}{rgb}{0.85000,0.32500,0.09800}%
%
\begin{tikzpicture}

\begin{axis}[%
width=0.951\fwidth,
height=\hwidth,
at={(0\fwidth,0\hwidth)},
scale only axis,
unbounded coords=jump,
xmin=28,
xmax=80,
xtick={40,52,56,64,68},
xticklabels={{10am},{1pm},{2pm},{4pm},{5pm}},
ymin=0,
ymax=150,
ylabel style={font=\color{white!15!black}},
ylabel={error [kW]},
axis background/.style={fill=white},
xmajorgrids,
ymajorgrids,
legend style={at={(0.5,0.97)}, anchor=north, legend columns=2, legend cell align=left, align=left, draw=white},
xlabel style={font=\footnotesize},ylabel style={font=\footnotesize},legend style={font=\footnotesize},ticklabel style={font=\footnotesize},ylabel shift = -5 pt,xlabel shift = -5 pt,
]

\addplot[area legend, draw=mycolor1, fill=mycolor1]
table[row sep=crcr] {%
x	y\\
28	97.6098920017012\\
29	87.0616094836532\\
30	111.133048508513\\
31	110.827159840066\\
32	113.977820092415\\
33	101.432643090638\\
34	116.764197799347\\
35	138.144394491007\\
36	131.003515767036\\
37	118.671301193402\\
38	93.1968936559279\\
39	93.0031230408833\\
40	85.9626603158963\\
41	95.1984454575204\\
42	94.470571016182\\
43	89.2501967117046\\
44	89.2640234573722\\
45	87.9949534522782\\
46	85.0583216254288\\
47	90.7399998693282\\
48	98.6742717459474\\
49	86.6002727932765\\
50	89.5009090083115\\
51	86.297390401661\\
52	84.8458696695925\\
53	85.7598494186104\\
54	82.7837124089043\\
55	80.2021271583231\\
56	80.4049505550647\\
57	86.5603710203912\\
58	85.9001968539745\\
59	90.426668177048\\
60	90.5529314299456\\
61	86.881133558221\\
62	90.9749172106499\\
63	85.8906819304997\\
64	81.1619532380134\\
65	79.327344026635\\
66	85.7068812428649\\
67	85.6081513087749\\
68	92.4794687869323\\
69	92.5422429183118\\
70	112.126042020991\\
71	94.7308409901038\\
72	89.8109566961293\\
73	97.4819064918549\\
74	96.354998087439\\
75	105.679970273324\\
76	91.025910924969\\
77	88.5299233517882\\
78	90.5401321803905\\
79	94.2144975581434\\
80	94.0620329496377\\
80	0\\
79	0\\
78	0\\
77	0\\
76	0\\
75	0\\
74	0\\
73	0\\
72	0\\
71	0\\
70	0\\
69	0\\
68	0\\
67	0\\
66	0\\
65	0\\
64	0\\
63	0\\
62	0\\
61	0\\
60	0\\
59	0\\
58	0\\
57	0\\
56	0\\
55	0\\
54	0\\
53	0\\
52	0\\
51	0\\
50	0\\
49	0\\
48	0\\
47	0\\
46	0\\
45	0\\
44	0\\
43	0\\
42	0\\
41	0\\
40	0\\
39	0\\
38	0\\
37	0\\
36	0\\
35	0\\
34	0\\
33	0\\
32	0\\
31	0\\
30	0\\
29	0\\
28	0\\
}--cycle;
\addlegendentry{$\text{2}\sigma$}

\addplot [color=mycolor2, line width=0.8pt]
  table[row sep=crcr]{%
28	nan\\
29	nan\\
30	nan\\
31	nan\\
32	nan\\
33	nan\\
34	nan\\
35	nan\\
36	nan\\
37	nan\\
38	nan\\
39	nan\\
40	nan\\
41	nan\\
42	nan\\
43	nan\\
44	nan\\
45	nan\\
46	nan\\
47	nan\\
48	nan\\
49	nan\\
50	nan\\
51	nan\\
52	11.3581975850125\\
53	10.7741373809408\\
54	1.65491727871677\\
55	5.04857761399467\\
56	10.598225655827\\
57	6.6312841551171\\
58	9.56068127806475\\
59	5.59768266202741\\
60	2.69785814063016\\
61	22.53572314896\\
62	3.58493577176932\\
63	3.7517533650132\\
64	8.51715817269633\\
65	16.6741531475573\\
66	2.37027927008285\\
67	10.9080773359292\\
68	3.48993329599693\\
69	nan\\
70	nan\\
71	nan\\
72	nan\\
73	nan\\
74	nan\\
75	nan\\
76	nan\\
77	nan\\
78	nan\\
79	nan\\
80	nan\\
};
\addlegendentry{prediction error}

\addplot [color=black, dashed, line width=0.8pt, forget plot]
  table[row sep=crcr]{%
52	0\\
52	16.6666666666667\\
52	33.3333333333333\\
52	50\\
52	66.6666666666667\\
52	83.3333333333333\\
52	100\\
52	116.666666666667\\
52	133.333333333333\\
52	150\\
};
\addplot [color=black, dashed, line width=0.8pt, forget plot]
  table[row sep=crcr]{%
56	0\\
56	16.6666666666667\\
56	33.3333333333333\\
56	50\\
56	66.6666666666667\\
56	83.3333333333333\\
56	100\\
56	116.666666666667\\
56	133.333333333333\\
56	150\\
};
\addplot [color=black, dashed, line width=0.8pt, forget plot]
  table[row sep=crcr]{%
64	0\\
64	16.6666666666667\\
64	33.3333333333333\\
64	50\\
64	66.6666666666667\\
64	83.3333333333333\\
64	100\\
64	116.666666666667\\
64	133.333333333333\\
64	150\\
};
\addplot [color=black, dashed, line width=0.8pt, forget plot]
  table[row sep=crcr]{%
68	0\\
68	16.6666666666667\\
68	33.3333333333333\\
68	50\\
68	66.6666666666667\\
68	83.3333333333333\\
68	100\\
68	116.666666666667\\
68	133.333333333333\\
68	150\\
};
\end{axis}
\end{tikzpicture}%
%	\setlength\fwidth{0.44\textwidth}
%	\setlength\hwidth{0.15\textwidth}	
%	% This file was created by matlab2tikz.
% Minimal pgfplots version: 1.3
%
%The latest updates can be retrieved from
%  http://www.mathworks.com/matlabcentral/fileexchange/22022-matlab2tikz
%where you can also make suggestions and rate matlab2tikz.
%
\definecolor{mycolor1}{rgb}{0.00000,0.44700,0.74100}%
\definecolor{mycolor2}{rgb}{0.49400,0.18400,0.55600}%
\definecolor{mycolor3}{rgb}{0.85000,0.32500,0.09800}%
%
\begin{tikzpicture}

\begin{axis}[%
width=0.95092\fwidth,
height=\hwidth,
at={(0\fwidth,0\hwidth)},
scale only axis,
xmin=28,
xmax=80,
xtick={40,52,56,64,68},
xticklabels={{10am},{1pm},{2pm},{4pm},{5pm}},
xmajorgrids,
ymin=5,
ymax=30,
ylabel={$\text{temperature [}^\text{o}\text{C]}$},
ymajorgrids,
legend style={at={(0.03,0.5)},anchor=west,legend columns=3,legend cell align=left,align=left,draw=white},
xlabel style={font=\footnotesize},ylabel style={font=\footnotesize},legend style={font=\footnotesize},ticklabel style={font=\footnotesize},ylabel shift = -5 pt,xlabel shift = -5 pt,
]
\addplot [color=mycolor1,solid,line width=0.8pt]
  table[row sep=crcr]{%
28	24\\
29	24\\
30	24\\
31	24\\
32	24\\
33	24\\
34	24\\
35	24\\
36	24\\
37	24\\
38	24\\
39	24\\
40	24\\
41	24\\
42	24\\
43	24\\
44	24\\
45	24\\
46	24\\
47	24\\
48	24\\
49	24\\
50	24\\
51	24\\
52	24\\
53	25.6570389885357\\
54	25.6189241638243\\
55	25.5273183233281\\
56	26.2907434828446\\
57	26.0011751525856\\
58	25.7233647503057\\
59	26.1321505190554\\
60	26.4592493905849\\
61	26.0600582854159\\
62	26.3818901076994\\
63	26.3387895015041\\
64	26.5495008717964\\
65	25.036426622822\\
66	24.5823539452488\\
67	24.7341390445692\\
68	24.9337773610262\\
69	24\\
70	24\\
71	24\\
72	24\\
73	24\\
74	24\\
75	24\\
76	24\\
77	24\\
78	24\\
79	24\\
80	24\\
};
\addlegendentry{cooling};

\addplot [color=mycolor2,solid,line width=0.8pt]
  table[row sep=crcr]{%
28	13\\
29	13\\
30	13\\
31	13\\
32	13\\
33	13\\
34	13\\
35	13\\
36	13\\
37	13\\
38	13\\
39	13\\
40	13\\
41	13\\
42	13\\
43	13\\
44	13\\
45	13\\
46	13\\
47	13\\
48	13\\
49	13\\
50	13\\
51	13\\
52	13\\
53	12.9579299724584\\
54	12.96009971949\\
55	12.9393428489775\\
56	13.2929131142471\\
57	13.3369583036851\\
58	13.2343335683688\\
59	13.3672343695149\\
60	13.3296489578334\\
61	12.9166817527838\\
62	13.6395555981625\\
63	13.6347917036618\\
64	13.85946735685\\
65	13.0415247902453\\
66	12.7447055779831\\
67	12.8872979840577\\
68	12.6821168944513\\
69	13\\
70	13\\
71	13\\
72	13\\
73	13\\
74	13\\
75	13\\
76	13\\
77	13\\
78	13\\
79	13\\
80	13\\
};
\addlegendentry{supply air};

\addplot [color=mycolor3,solid,line width=0.8pt]
  table[row sep=crcr]{%
28	6.7\\
29	6.7\\
30	6.7\\
31	6.7\\
32	6.7\\
33	6.7\\
34	6.7\\
35	6.7\\
36	6.7\\
37	6.7\\
38	6.7\\
39	6.7\\
40	6.7\\
41	6.7\\
42	6.7\\
43	6.7\\
44	6.7\\
45	6.7\\
46	6.7\\
47	6.7\\
48	6.7\\
49	6.7\\
50	6.7\\
51	6.7\\
52	6.7\\
53	6.68106534197597\\
54	6.64905617515537\\
55	6.8190596837857\\
56	7.38419648861323\\
57	7.76125673227166\\
58	7.91071727376612\\
59	8.77174751140269\\
60	8.40217021063824\\
61	7.67931744449659\\
62	8.24957222529671\\
63	8.8268932563618\\
64	8.98071424781767\\
65	7.99324388425106\\
66	7.45317315803806\\
67	6.97761610822872\\
68	6.66788941197385\\
69	6.7\\
70	6.7\\
71	6.7\\
72	6.7\\
73	6.7\\
74	6.7\\
75	6.7\\
76	6.7\\
77	6.7\\
78	6.7\\
79	6.7\\
80	6.7\\
};
\addlegendentry{chilled water};

\addplot [color=black,dashed,line width=0.8pt,forget plot]
  table[row sep=crcr]{%
52	0\\
52	5.55555555555556\\
52	11.1111111111111\\
52	16.6666666666667\\
52	22.2222222222222\\
52	27.7777777777778\\
52	33.3333333333333\\
52	38.8888888888889\\
52	44.4444444444444\\
52	50\\
};
\addplot [color=black,dashed,line width=0.8pt,forget plot]
  table[row sep=crcr]{%
56	0\\
56	5.55555555555556\\
56	11.1111111111111\\
56	16.6666666666667\\
56	22.2222222222222\\
56	27.7777777777778\\
56	33.3333333333333\\
56	38.8888888888889\\
56	44.4444444444444\\
56	50\\
};
\addplot [color=black,dashed,line width=0.8pt,forget plot]
  table[row sep=crcr]{%
64	0\\
64	5.55555555555556\\
64	11.1111111111111\\
64	16.6666666666667\\
64	22.2222222222222\\
64	27.7777777777778\\
64	33.3333333333333\\
64	38.8888888888889\\
64	44.4444444444444\\
64	50\\
};
\addplot [color=black,dashed,line width=0.8pt,forget plot]
  table[row sep=crcr]{%
68	0\\
68	5.55555555555556\\
68	11.1111111111111\\
68	16.6666666666667\\
68	22.2222222222222\\
68	27.7777777777778\\
68	33.3333333333333\\
68	38.8888888888889\\
68	44.4444444444444\\
68	50\\
};
\end{axis}
\end{tikzpicture}%	
%	\caption{jaddjndn.}
%	\captionsetup{justification=centering}
%	\label{F:control:tracking}
%\end{figure}}
\begin{figure}[t!]
	\centering
	\begin{subfigure}
		\centering
		\setlength\fwidth{0.44\textwidth}
		\setlength\hwidth{0.15\textwidth}	
		\input{figures/control-tracking.tex}
		\caption{The reference power signal is closely tracked by GP model providing sustained curtailment of \(90\) kW (with respect to the baseline) during the Demand Response event 2-4pm. Due to \(1\)hr horizon in the control problem, the curtailment starts at 1:15pm, and the controller is further active until 5pm to reduce the effect of kickback.}
		\label{F:control:tracking}
	\end{subfigure}
	\begin{subfigure}
		\centering
		\setlength\fwidth{0.44\textwidth}
		\setlength\hwidth{0.1\textwidth}	
		% This file was created by matlab2tikz.
%
%The latest updates can be retrieved from
%  http://www.mathworks.com/matlabcentral/fileexchange/22022-matlab2tikz-matlab2tikz
%where you can also make suggestions and rate matlab2tikz.
%
\definecolor{mycolor1}{rgb}{0.97647,0.89804,1.00000}%
\definecolor{mycolor2}{rgb}{0.85000,0.32500,0.09800}%
%
\begin{tikzpicture}

\begin{axis}[%
width=0.951\fwidth,
height=\hwidth,
at={(0\fwidth,0\hwidth)},
scale only axis,
unbounded coords=jump,
xmin=28,
xmax=80,
xtick={40,52,56,64,68},
xticklabels={{10am},{1pm},{2pm},{4pm},{5pm}},
ymin=0,
ymax=150,
ylabel style={font=\color{white!15!black}},
ylabel={error [kW]},
axis background/.style={fill=white},
xmajorgrids,
ymajorgrids,
legend style={at={(0.5,0.97)}, anchor=north, legend columns=2, legend cell align=left, align=left, draw=white},
xlabel style={font=\footnotesize},ylabel style={font=\footnotesize},legend style={font=\footnotesize},ticklabel style={font=\footnotesize},ylabel shift = -5 pt,xlabel shift = -5 pt,
]

\addplot[area legend, draw=mycolor1, fill=mycolor1]
table[row sep=crcr] {%
x	y\\
28	97.6098920017012\\
29	87.0616094836532\\
30	111.133048508513\\
31	110.827159840066\\
32	113.977820092415\\
33	101.432643090638\\
34	116.764197799347\\
35	138.144394491007\\
36	131.003515767036\\
37	118.671301193402\\
38	93.1968936559279\\
39	93.0031230408833\\
40	85.9626603158963\\
41	95.1984454575204\\
42	94.470571016182\\
43	89.2501967117046\\
44	89.2640234573722\\
45	87.9949534522782\\
46	85.0583216254288\\
47	90.7399998693282\\
48	98.6742717459474\\
49	86.6002727932765\\
50	89.5009090083115\\
51	86.297390401661\\
52	84.8458696695925\\
53	85.7598494186104\\
54	82.7837124089043\\
55	80.2021271583231\\
56	80.4049505550647\\
57	86.5603710203912\\
58	85.9001968539745\\
59	90.426668177048\\
60	90.5529314299456\\
61	86.881133558221\\
62	90.9749172106499\\
63	85.8906819304997\\
64	81.1619532380134\\
65	79.327344026635\\
66	85.7068812428649\\
67	85.6081513087749\\
68	92.4794687869323\\
69	92.5422429183118\\
70	112.126042020991\\
71	94.7308409901038\\
72	89.8109566961293\\
73	97.4819064918549\\
74	96.354998087439\\
75	105.679970273324\\
76	91.025910924969\\
77	88.5299233517882\\
78	90.5401321803905\\
79	94.2144975581434\\
80	94.0620329496377\\
80	0\\
79	0\\
78	0\\
77	0\\
76	0\\
75	0\\
74	0\\
73	0\\
72	0\\
71	0\\
70	0\\
69	0\\
68	0\\
67	0\\
66	0\\
65	0\\
64	0\\
63	0\\
62	0\\
61	0\\
60	0\\
59	0\\
58	0\\
57	0\\
56	0\\
55	0\\
54	0\\
53	0\\
52	0\\
51	0\\
50	0\\
49	0\\
48	0\\
47	0\\
46	0\\
45	0\\
44	0\\
43	0\\
42	0\\
41	0\\
40	0\\
39	0\\
38	0\\
37	0\\
36	0\\
35	0\\
34	0\\
33	0\\
32	0\\
31	0\\
30	0\\
29	0\\
28	0\\
}--cycle;
\addlegendentry{$\text{2}\sigma$}

\addplot [color=mycolor2, line width=0.8pt]
  table[row sep=crcr]{%
28	nan\\
29	nan\\
30	nan\\
31	nan\\
32	nan\\
33	nan\\
34	nan\\
35	nan\\
36	nan\\
37	nan\\
38	nan\\
39	nan\\
40	nan\\
41	nan\\
42	nan\\
43	nan\\
44	nan\\
45	nan\\
46	nan\\
47	nan\\
48	nan\\
49	nan\\
50	nan\\
51	nan\\
52	11.3581975850125\\
53	10.7741373809408\\
54	1.65491727871677\\
55	5.04857761399467\\
56	10.598225655827\\
57	6.6312841551171\\
58	9.56068127806475\\
59	5.59768266202741\\
60	2.69785814063016\\
61	22.53572314896\\
62	3.58493577176932\\
63	3.7517533650132\\
64	8.51715817269633\\
65	16.6741531475573\\
66	2.37027927008285\\
67	10.9080773359292\\
68	3.48993329599693\\
69	nan\\
70	nan\\
71	nan\\
72	nan\\
73	nan\\
74	nan\\
75	nan\\
76	nan\\
77	nan\\
78	nan\\
79	nan\\
80	nan\\
};
\addlegendentry{prediction error}

\addplot [color=black, dashed, line width=0.8pt, forget plot]
  table[row sep=crcr]{%
52	0\\
52	16.6666666666667\\
52	33.3333333333333\\
52	50\\
52	66.6666666666667\\
52	83.3333333333333\\
52	100\\
52	116.666666666667\\
52	133.333333333333\\
52	150\\
};
\addplot [color=black, dashed, line width=0.8pt, forget plot]
  table[row sep=crcr]{%
56	0\\
56	16.6666666666667\\
56	33.3333333333333\\
56	50\\
56	66.6666666666667\\
56	83.3333333333333\\
56	100\\
56	116.666666666667\\
56	133.333333333333\\
56	150\\
};
\addplot [color=black, dashed, line width=0.8pt, forget plot]
  table[row sep=crcr]{%
64	0\\
64	16.6666666666667\\
64	33.3333333333333\\
64	50\\
64	66.6666666666667\\
64	83.3333333333333\\
64	100\\
64	116.666666666667\\
64	133.333333333333\\
64	150\\
};
\addplot [color=black, dashed, line width=0.8pt, forget plot]
  table[row sep=crcr]{%
68	0\\
68	16.6666666666667\\
68	33.3333333333333\\
68	50\\
68	66.6666666666667\\
68	83.3333333333333\\
68	100\\
68	116.666666666667\\
68	133.333333333333\\
68	150\\
};
\end{axis}
\end{tikzpicture}%
		\caption{The tracking error during the DR event is always less \(22.5\) kW (\(1.7\%\)) and the mean absolute error is \(7.9\) kW (\(0.6\%\)).}
		\label{F:control:error}
	\end{subfigure}
	\begin{subfigure}
		\centering
		\setlength\fwidth{0.44\textwidth}
		\setlength\hwidth{0.15\textwidth}	
		% This file was created by matlab2tikz.
% Minimal pgfplots version: 1.3
%
%The latest updates can be retrieved from
%  http://www.mathworks.com/matlabcentral/fileexchange/22022-matlab2tikz
%where you can also make suggestions and rate matlab2tikz.
%
\definecolor{mycolor1}{rgb}{0.00000,0.44700,0.74100}%
\definecolor{mycolor2}{rgb}{0.49400,0.18400,0.55600}%
\definecolor{mycolor3}{rgb}{0.85000,0.32500,0.09800}%
%
\begin{tikzpicture}

\begin{axis}[%
width=0.95092\fwidth,
height=\hwidth,
at={(0\fwidth,0\hwidth)},
scale only axis,
xmin=28,
xmax=80,
xtick={40,52,56,64,68},
xticklabels={{10am},{1pm},{2pm},{4pm},{5pm}},
xmajorgrids,
ymin=5,
ymax=30,
ylabel={$\text{temperature [}^\text{o}\text{C]}$},
ymajorgrids,
legend style={at={(0.03,0.5)},anchor=west,legend columns=3,legend cell align=left,align=left,draw=white},
xlabel style={font=\footnotesize},ylabel style={font=\footnotesize},legend style={font=\footnotesize},ticklabel style={font=\footnotesize},ylabel shift = -5 pt,xlabel shift = -5 pt,
]
\addplot [color=mycolor1,solid,line width=0.8pt]
  table[row sep=crcr]{%
28	24\\
29	24\\
30	24\\
31	24\\
32	24\\
33	24\\
34	24\\
35	24\\
36	24\\
37	24\\
38	24\\
39	24\\
40	24\\
41	24\\
42	24\\
43	24\\
44	24\\
45	24\\
46	24\\
47	24\\
48	24\\
49	24\\
50	24\\
51	24\\
52	24\\
53	25.6570389885357\\
54	25.6189241638243\\
55	25.5273183233281\\
56	26.2907434828446\\
57	26.0011751525856\\
58	25.7233647503057\\
59	26.1321505190554\\
60	26.4592493905849\\
61	26.0600582854159\\
62	26.3818901076994\\
63	26.3387895015041\\
64	26.5495008717964\\
65	25.036426622822\\
66	24.5823539452488\\
67	24.7341390445692\\
68	24.9337773610262\\
69	24\\
70	24\\
71	24\\
72	24\\
73	24\\
74	24\\
75	24\\
76	24\\
77	24\\
78	24\\
79	24\\
80	24\\
};
\addlegendentry{cooling};

\addplot [color=mycolor2,solid,line width=0.8pt]
  table[row sep=crcr]{%
28	13\\
29	13\\
30	13\\
31	13\\
32	13\\
33	13\\
34	13\\
35	13\\
36	13\\
37	13\\
38	13\\
39	13\\
40	13\\
41	13\\
42	13\\
43	13\\
44	13\\
45	13\\
46	13\\
47	13\\
48	13\\
49	13\\
50	13\\
51	13\\
52	13\\
53	12.9579299724584\\
54	12.96009971949\\
55	12.9393428489775\\
56	13.2929131142471\\
57	13.3369583036851\\
58	13.2343335683688\\
59	13.3672343695149\\
60	13.3296489578334\\
61	12.9166817527838\\
62	13.6395555981625\\
63	13.6347917036618\\
64	13.85946735685\\
65	13.0415247902453\\
66	12.7447055779831\\
67	12.8872979840577\\
68	12.6821168944513\\
69	13\\
70	13\\
71	13\\
72	13\\
73	13\\
74	13\\
75	13\\
76	13\\
77	13\\
78	13\\
79	13\\
80	13\\
};
\addlegendentry{supply air};

\addplot [color=mycolor3,solid,line width=0.8pt]
  table[row sep=crcr]{%
28	6.7\\
29	6.7\\
30	6.7\\
31	6.7\\
32	6.7\\
33	6.7\\
34	6.7\\
35	6.7\\
36	6.7\\
37	6.7\\
38	6.7\\
39	6.7\\
40	6.7\\
41	6.7\\
42	6.7\\
43	6.7\\
44	6.7\\
45	6.7\\
46	6.7\\
47	6.7\\
48	6.7\\
49	6.7\\
50	6.7\\
51	6.7\\
52	6.7\\
53	6.68106534197597\\
54	6.64905617515537\\
55	6.8190596837857\\
56	7.38419648861323\\
57	7.76125673227166\\
58	7.91071727376612\\
59	8.77174751140269\\
60	8.40217021063824\\
61	7.67931744449659\\
62	8.24957222529671\\
63	8.8268932563618\\
64	8.98071424781767\\
65	7.99324388425106\\
66	7.45317315803806\\
67	6.97761610822872\\
68	6.66788941197385\\
69	6.7\\
70	6.7\\
71	6.7\\
72	6.7\\
73	6.7\\
74	6.7\\
75	6.7\\
76	6.7\\
77	6.7\\
78	6.7\\
79	6.7\\
80	6.7\\
};
\addlegendentry{chilled water};

\addplot [color=black,dashed,line width=0.8pt,forget plot]
  table[row sep=crcr]{%
52	0\\
52	5.55555555555556\\
52	11.1111111111111\\
52	16.6666666666667\\
52	22.2222222222222\\
52	27.7777777777778\\
52	33.3333333333333\\
52	38.8888888888889\\
52	44.4444444444444\\
52	50\\
};
\addplot [color=black,dashed,line width=0.8pt,forget plot]
  table[row sep=crcr]{%
56	0\\
56	5.55555555555556\\
56	11.1111111111111\\
56	16.6666666666667\\
56	22.2222222222222\\
56	27.7777777777778\\
56	33.3333333333333\\
56	38.8888888888889\\
56	44.4444444444444\\
56	50\\
};
\addplot [color=black,dashed,line width=0.8pt,forget plot]
  table[row sep=crcr]{%
64	0\\
64	5.55555555555556\\
64	11.1111111111111\\
64	16.6666666666667\\
64	22.2222222222222\\
64	27.7777777777778\\
64	33.3333333333333\\
64	38.8888888888889\\
64	44.4444444444444\\
64	50\\
};
\addplot [color=black,dashed,line width=0.8pt,forget plot]
  table[row sep=crcr]{%
68	0\\
68	5.55555555555556\\
68	11.1111111111111\\
68	16.6666666666667\\
68	22.2222222222222\\
68	27.7777777777778\\
68	33.3333333333333\\
68	38.8888888888889\\
68	44.4444444444444\\
68	50\\
};
\end{axis}
\end{tikzpicture}%	
		\caption{Optimal set points obtained after solving optimization \eqref{E:casestudy:mpc}.}
		\label{F:control:inputs}
	\end{subfigure}
%	\caption{Sustained curtailment using predictive control with Gaussian Processes.}
\end{figure}

The office building has a large HVAC system, so for this building we consider the following Demand Response scenario. 
Due to price volatility, the office receives a request from the aggregator to shed \(90\) kW load between 2-4pm. 
Now, the goal of the operators is to decide setpoints that would guarantee this curtailment while following stringent operation and thermal comfort constraints. 
Rule-based strategies do not guarantee this curtailment and hence pose a huge financial risk. 
Using our data-driven approach for control, we can synthesize optimal setpoint recommendations.
Fig~\ref{F:control:tracking} shows the load shedding between 2-4pm. 
The baseline power consumption indicates the usage if there was no DR event, or in other words if the building would have continued to operate under normal conditions. The reference for tracking differs from baseline by \(90\) kW during 2-4pm.
The mean prediction denoted by \(\mu\) is the output \(\bar{y}_{t}\) which follows the reference signal closely as the input constraints are never active. The actual (system) building power consumption differs only marginally from the reference as shown in Fig.~\ref{F:control:error}. The maximum tracking error during the DR event is \(22.5\) kW (\(1.7\%\)) and the mean absolute error is \(7.9\) kW (\(0.6\%\)). The optimal setpoints are shown in Fig.~\ref{F:control:inputs}. The controller has a prediction horizon of \(1\) hr. It kicks in at 1:15pm increase the cooling set point, chilled water temperature and supply air temperature to meet the requirement of \(90\) kW. After 4pm, we continue to follow the baseline signal for the next one hour to reduce the effect of kickback.
\todo[inline]{Is the tracking error calculated by reference - actual building power? Does it also include the battery power (i.e., reference - (building + battery))? Anyways, 22.5 kW error is not 1.7\% but you must calculate the error as 22.5/90 = 25\% error, i.e., versus the curtailment, not the total power. I think it would be better to plot the HVAC power instead of the total power, to hide the small fraction of 90kW to the total power of 1.4 MW. And to plot the (building + battery) power instead of building-only power.}

\subsection{Active Learning}
\label{SS:casestudy:active}

\subsection{Discussion}

MV and IG are better but the benefits marginal after \(\sim200\) hrs of functional testing.

\todo[inline]{computational complexity}


%%% Local Variables:
%%% mode: latex
%%% TeX-master: "main"
%%% End:


%% CONCLUSION
\section{Conclusion}
Conclusion goes here.

%ACKNOWLEDGMENTS
%\section*{Acknowledgments}
Acknowledgement goes here.

%% BIBLIOGRAPHY
\bibliographystyle{unsrt}
\bibliography{references} 

\end{document}
