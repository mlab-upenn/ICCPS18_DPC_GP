\section{Gaussian Processes}

Gaussian Processes offer several advantages over other machine learning algorithms that make them more suitable for identification of dynamical systems. (1) GP provide an estimate of uncertainty or confidence in the predictions. For example, we can estimate a 95\% confidence bound for the predictions which can be used to measure control performance. (2) GP work for well with small data set, which is in general helpful for any learning application. (3) The model structure of GP allows to include prior knowledge of the system behavior by defining priors on the parameters or using particular structure of covariance functions.


\subsection{Training}

\todo[inline]{GP training}

\subsection{Prediction}

\todo[inline]{GP prediction}

\begin{figure}[h!]
	\centering
	\missingfigure[figwidth=20pc]{show GP prior and posterior side by side for building example}
	\caption{}
	\captionsetup{justification=centering}
	\label{F:gp_prior_posterior}
\end{figure}