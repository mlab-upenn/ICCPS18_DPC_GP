\section{Gaussian Processes}
\label{S:gp}

In this section, we briefly introduce modeling with Gaussian Process (GP) and its applications in control.
More details can be found in \cite{Rasmussen2006} %on GP for machine learning and in
and \cite{Kocijan2016}. % on GP modeling of dynamic systems.

\begin{definition}[\cite{Rasmussen2006}]
A Gaussian Process is a collection of random variables, any finite number of which have a joint Gaussian distribution.
\end{definition}
Consider noisy observations \(y\) of an underlying function \(f: \RR^n \mapsto \RR\) through a Gaussian noise model: \(y = f(x) + \GaussianDist{0}{\sigma_n^2}\), \(x \in \RR^n\).
A GP of \(y\) is fully specified by its mean function \(\mu(x)\) and covariance function \(k(x,x')\),
\begin{align}
\label{E:gp:prior}
\mu(x; \theta) &= \EE [f(x)] \\
k(x,x'; \theta) &= \EE [(f(x)\!-\!\mu(x)) (f(x') \!-\! \mu(x'))] + \sigma_n^2 \delta(x,x') \nonumber
\end{align}
where \(\delta(x,x')\) is the Kronecker delta function.
The hyperparameter vector \(\theta\) parameterizes the mean and covariance functions.
This GP is denoted by \(y \sim \mathcal{GP}(\mu, k; \theta)\).

Given the regression vectors \(X = [x_1, \dots, x_N]\) and the corresponding observed outputs \(Y = [y_1, \dots, y_N]^T\), the distribution of the output \(y_\star\) corresponding to a new input vector \(x_\star\) is a Gaussian distribution \(y_\star | x_\star \sim \GaussianDist{\bar{y}_\star}{\sigma_\star^2}\), with mean and variance given by
\begin{subequations}
\label{E:gp-regression}
\begin{align}
\bar{y}_\star &= g_{\mathrm{m}} (x_{\star}) \coloneqq \mu(x_\star) + K_\star K^{-1} (Y - \mu(X))\\
\sigma_\star^2 &= g_{\mathrm{v}} (x_{\star}) \coloneqq K_{\star \star} - K_\star K^{-1} K_\star^T \text,
\end{align}
\end{subequations}
where \(K_\star = [k(x_\star, x_1), \dots, k(x_\star, x_N)]\), \(K_{\star \star} = k(x_\star, x_\star)\), and $K$ is the covariance matrix with elements \(K_{ij} = k(x_i, x_j)\).

Note that the mean and covariance functions are parameterized by the hyperparameters $\theta$, which can be learned by maximizing the likelihood: \(\argmax_\theta \Pr(Y \vert X, \theta)\).
The covariance function \(k(x,x')\) indicates how correlated the outputs are at \(x\) and \(x'\), with the intuition that the output at an input is influenced more by the outputs of nearby inputs in the training data $\D = (X, Y)$.
In other words, a GP model specifies the structure of the covariance matrix of, or the relationship between, the input variables rather than a fixed structural input--output relationship.
It is therefore highly flexible and can capture complex behavior with fewer parameters.
An example of GP prior and posterior is shown in Fig.~\ref{F:gp:prior:posterior}. We use a constant mean function and a combination of squared exponential kernel and rational quadratic kernel as described in Sec.~\ref{SS:casestudy:gp}.
There exists a wide range of covariance functions and combinations to choose from \cite{Rasmussen2006}. 

GPs offer several advantages over other machine learning algorithms that make them more suitable for identification of dynamical systems.
\begin{enumerate}
\item GPs provide an estimate of uncertainty or confidence in the
  predictions through the predictive variance.  While the predictive mean is often used as the best guess of the output, the full distribution can be used in a meaningful way. For example, we can estimate a 95\% confidence bound for the predictions which can be used to measure control performance.
\item GPs work well with small data sets.  This ability is generally useful for any learning application.
\item GPs allow including prior knowledge of the system behavior by defining priors on the hyperparameters or constructing a particular structure of the covariance function.  This feature enables incorporating domain knowledge into the GP model to improve its accuracy.
\end{enumerate}

\subsection{Gaussian Processes for Dynamical Systems}
\label{SS:intro-gp:control}

GPs can be used for modeling nonlinear dynamical systems, by feeding autoregressive, or time-delayed, input and output signals back to the model as regressors \cite{Kocijan2016}.
Specifically, in control systems, it is common to use an autoregressive GP to model a dynamical system represented by the nonlinear function
\begin{math}
y_{t} = f(x_t)
\end{math}
where
\begin{equation*}
x_{t}\!=\![y_{t-l}, \dots, y_{t-1}, u_{t-m}, \dots, u_t, w_{t-p}, \dots, w_{t-1}, w_t] \text.
\end{equation*}
Here, \(t\) denotes the time step, \(u\) the control input, \(w\) the exogenous disturbance input, \(y\) the (past) output, and \(l\), \(m\), and \(p\) are respectively the lags for autoregressive outputs, control inputs, and disturbances.
Note that \(u_t\) and \(w_t\) are the current control and disturbance inputs.
The vector of all autoregressive inputs can be thought of as the current state of the model.
A dynamical GP can then be trained from data in the same way as any other GPs.

\iffalse
When a GP is used for control or optimization, it is usually necessary to simulate the model over a finite number of future steps and predict its multistep-ahead behavior.
Because the output of a GP is a distribution rather than a point estimate, the autoregressive outputs fed to the model beyond the first step are random variables, resulting in more and more complex output distributions as we go further.
Therefore, a multistep simulation of a GP involves the propagation of uncertainty through the model.
There exist several methods for uncertainty propagation in GPs \cite{girard04approximate,Kocijan2016}. 
We mention here two simulation methods for autoregressive GPs.
\begin{itemize}
	\item The \emph{Monte-Carlo method} obtains samples of the output distribution under input uncertainty, which can be seen as a Gaussian mixture.  This Gaussian mixture becomes more complex in later steps of the simulation, therefore efficient numerical algorithms must be implemented.  This method can achieve good prediction accuracy at the expense of high computational load.  It is also general, \ie it can be used with any covariance functions.
	\item The \emph{zero-variance method} does not propagate uncertainty.  At each step, the autoregressive outputs are replaced by their corresponding expected values.  Obviously, this method will underestimate the variances of the output distributions.  However, its computational simplicity is attractive, especially in optimization applications where the GP must be simulated for many times.  In such cases, if the prediction error caused by not propagating uncertainty is insignificant, the zero-variance method can and should be used.  For more detailed discussions on this topic, see \cite{Kocijan2016,girard04approximate}.
\end{itemize}

\fi

\begin{figure}[!tb]
  \centering
	\setlength\fwidth{0.4\textwidth}
	\setlength\hwidth{0.2\textwidth}	
	% This file was created by matlab2tikz.
%
%The latest updates can be retrieved from
%  http://www.mathworks.com/matlabcentral/fileexchange/22022-matlab2tikz-matlab2tikz
%where you can also make suggestions and rate matlab2tikz.
%
\definecolor{mycolor1}{rgb}{0.97647,0.89804,1.00000}%
%
\begin{tikzpicture}

\begin{axis}[%
width=0.951\fwidth,
height=\hwidth,
at={(0\fwidth,0\hwidth)},
scale only axis,
xmin=3.7,
xmax=9.7,
xlabel style={font=\color{white!15!black}},
xlabel={$\text{Chilled water temp. [}^\text{o}\text{C]}$},
ymin=8.37589565985735,
ymax=409.872922593159,
ylabel style={font=\color{white!15!black}},
ylabel={power [kW]},
axis background/.style={fill=white},
legend style={legend cell align=left, align=left, draw=white!15!black}
]

\addplot[area legend, draw=white!75!gray, fill=white!75!gray]
table[row sep=crcr] {%
x	y\\
3.7	391.623057732555\\
4.01578947368421	391.623057732555\\
4.33157894736842	391.623057732555\\
4.64736842105263	391.623057732555\\
4.96315789473684	391.623057732555\\
5.27894736842105	391.623057732555\\
5.59473684210526	391.623057732555\\
5.91052631578947	391.623057732555\\
6.22631578947368	391.623057732555\\
6.54210526315789	391.623057732555\\
6.8578947368421	391.623057732555\\
7.17368421052632	391.623057732555\\
7.48947368421053	391.623057732555\\
7.80526315789474	391.623057732555\\
8.12105263157895	391.623057732555\\
8.43684210526316	391.623057732555\\
8.75263157894737	391.623057732555\\
9.06842105263158	391.623057732555\\
9.38421052631579	391.623057732555\\
9.7	391.623057732555\\
9.7	26.625760520462\\
9.38421052631579	26.625760520462\\
9.06842105263158	26.625760520462\\
8.75263157894737	26.625760520462\\
8.43684210526316	26.625760520462\\
8.12105263157895	26.625760520462\\
7.80526315789474	26.625760520462\\
7.48947368421053	26.625760520462\\
7.17368421052632	26.625760520462\\
6.8578947368421	26.625760520462\\
6.54210526315789	26.625760520462\\
6.22631578947368	26.625760520462\\
5.91052631578947	26.625760520462\\
5.59473684210526	26.625760520462\\
5.27894736842105	26.625760520462\\
4.96315789473684	26.625760520462\\
4.64736842105263	26.625760520462\\
4.33157894736842	26.625760520462\\
4.01578947368421	26.625760520462\\
3.7	26.625760520462\\
}--cycle;
\addlegendentry{$\text{prior }\mu\text{ }\pm\text{ 2}\sigma$}

\addplot [color=black, line width=1.0pt]
  table[row sep=crcr]{%
3.69999999999999	209.124409126508\\
9.69999999999999	209.124409126508\\
};
\addlegendentry{$\text{prior }\mu$}


\addplot[area legend, draw=mycolor1, fill=mycolor1]
table[row sep=crcr] {%
x	y\\
3.7	173.087265186663\\
4.01578947368421	170.946487305241\\
4.33157894736842	168.871442969416\\
4.64736842105263	166.853382071716\\
4.96315789473684	164.88423519122\\
5.27894736842105	162.957122006465\\
5.59473684210526	161.066750293675\\
5.91052631578947	159.209700398551\\
6.22631578947368	157.384606604534\\
6.54210526315789	155.59225514469\\
6.8578947368421	153.835619661725\\
7.17368421052632	152.119850959456\\
7.48947368421053	150.452231064526\\
7.80526315789474	148.842093771292\\
8.12105263157895	147.300706284425\\
8.43684210526316	145.841100548816\\
8.75263157894737	144.477839615212\\
9.06842105263158	143.226705392763\\
9.38421052631579	142.104300594135\\
9.7	141.127569954911\\
9.7	101.068180525827\\
9.38421052631579	102.858180785079\\
9.06842105263158	104.634427758675\\
8.75263157894737	106.40410900345\\
8.43684210526316	108.173775004104\\
8.12105263157895	109.948906595954\\
7.80526315789474	111.733564246481\\
7.48947368421053	113.530123986469\\
7.17368421052632	115.339092520184\\
6.8578947368421	117.158987568735\\
6.54210526315789	118.986268514527\\
6.22631578947368	120.815305636956\\
5.91052631578947	122.638382277541\\
5.59473684210526	124.445731815086\\
5.27894736842105	126.225619206856\\
4.96315789473684	127.964483657655\\
4.64736842105263	129.647162983633\\
4.33157894736842	131.257219158361\\
4.01578947368421	132.777376249578\\
3.7	134.190065412375\\
}--cycle;
\addlegendentry{$\text{posterior }\mu\text{ }\pm\text{ 2}\sigma$}

\addplot [color=red, dashed, line width=1.0pt]
  table[row sep=crcr]{%
3.69999999999999	153.638665299519\\
4.01578947368421	151.86193177741\\
4.33157894736843	150.064331063888\\
4.64736842105262	148.250272527675\\
5.27894736842106	144.591370606661\\
5.91052631578947	140.924041338046\\
6.22631578947369	139.099956120745\\
6.54210526315791	137.289261829609\\
6.8578947368421	135.49730361523\\
7.17368421052632	133.72947173982\\
7.48947368421054	131.991177525498\\
7.80526315789473	130.287829008886\\
8.12105263157895	128.624806440189\\
8.43684210526317	127.00743777646\\
8.75263157894736	125.440974309331\\
9.06842105263158	123.930566575719\\
9.3842105263158	122.481240689607\\
9.69999999999999	121.097875240369\\
};
\addlegendentry{$\text{posterior }\mu$}

\end{axis}
\end{tikzpicture}%
    \todo[inline]{This figure is impossibe to see!!! You can truncate the length by half (show the result for 1 day, or 12 hours instead). Create two smaller plots side by side, or select colors better.}
  \caption{Example of priors calculated using \eqref{E:gp:prior} and posteriors using \eqref{E:gp-regression} for predicting power consumption of a building for two days. Initially the mean is constant because \(\mu(x)\) is constant, and we observe a high variance. The posterior agrees with the actual power consumption with high confidence.}
  \captionsetup{justification=centering}
  \label{F:gp:prior:posterior}
\end{figure}

%%% Local Variables:
%%% mode: latex
%%% TeX-master: "main"
%%% End:
