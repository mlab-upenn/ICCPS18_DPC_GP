\section{Conclusion}

Learning black-box models for real-time control reduces the cost and time required to model complex physical systems like buildings and chemical plants by an order of magnitude. 
However, many practical issues must be addressed before employing machine learning algorithms on large scale in closed-loop with a physical system.
This paper addresses the various challenges associated in bridging machine learning and controls with application to load curtailment for Demand Response.
We propose a method for optimal experiment design using Gaussian Processes to recommend strategies for functional test (in closed-loop with the plant) when limited data are available. 
We show that under operational constraints, data generated by the proposed OED method based on maximizing information gain or maximizing variance provides much faster learning rate than uniform random sampling or pseudo random binary sampling. 
However, when functional tests are allowed for more than \(200-250\) hours, model accuracies obtained with OED and random sampling are comparable.
We exploit the variance in predictions from GPs to formulate a stochastic optimization problem to design an MPC controller to provide the desired load curtailment with high confidence during a DR event. 
We observe maximum tracking error of \(1.7\%\) and mean absolute error of \(0.6\%\). \todo{Revise these numbers!!!}
Finally, we extend the OED approach to update the GP model as new data is generated by running the controller in a closed-loop with the building, obviating the repetitive need for functional tests as the system properties change with time.

%%% Local Variables:
%%% mode: latex
%%% TeX-master: "main"
%%% End:
