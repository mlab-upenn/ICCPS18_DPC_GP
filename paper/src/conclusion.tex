\section{Conclusion}

Learning black-box models for real-time control reduces the cost and time required to model complex physical systems like buildings and chemical plants \textit{by an order of magnitude}.
This paper addresses the various challenges associated in bridging machine learning and controls with application to load curtailment for Demand Response.
(1) We propose a method for optimal experiment design using Gaussian Processes to recommend strategies for functional test (in closed-loop with the plant) when limited data are available. 
We show that under operational constraints, data generated by the proposed OED method based on maximizing information gain or maximizing variance provides much faster learning rate than uniform random sampling or pseudo random binary sampling. 
OED drastically reduces the duration of required functional tests by upto (50\%), which, in practice, are permitted for only a few hours in a month due to operation constraints.
(2) We exploit the variance in predictions from GPs to formulate a stochastic optimization problem to design an MPC controller to provide the desired load curtailment with perfect tracking and maximum \(1.7\%\) prediction error during a DR event. 
(3) Finally, we extend the OED approach to update the GP model as new data is generated by running the controller in a closed-loop with the building, reducing the repetitive need for functional tests as the system properties change with time.

While we can do functional tests more efficiently, perform closed-loop control with high confidence and update the model online with Gaussian Processes, our future work will focus on scaling the approach to even more complex systems like a network of buildings in a district.